\documentclass[output=paper]{langsci/langscibook} 
\ChapterDOI{10.5281/zenodo.2583812}
\author{Holger Diessel\affiliation{University of Jena}}
\title{Preposed adverbial clauses: Functional adaptation and diachronic inheritance}
\abstract{In the historical literature it is commonly assumed that subordinate clauses are derived from paratactic sentences. However, while this assumption is not implausible for certain types of postposed adverbial clauses, there is no obvious connection between preposed adverbial clauses and parataxis. This paper investigates the diachronic development of preposed adverbial clauses from a cross-linguistic perspective. Drawing on data from a typological and diachronic database, it is shown that preposed adverbial clauses evolve from various diachronic sources that are semantically and structurally similar to the target construction (e.g. \isi{adposition}al phrases, pre- and postnominal \isi{relative clause}s, juxtaposed\is{juxtaposition} sentences). Considering the factors behind these developments, the paper argues that while the occurrence of preposed adverbial clauses can be explained by general cognitive processes of language use, the internal structure of preposed adverbial clauses, notably the position of the \isi{subordinator}, is primarily determined by grammaticalization.}
\shorttitlerunninghead{Preposed adverbial clauses}
\begin{document}
\maketitle 

\section{Introduction}

It is a standard assumption of historical linguistics that syntactic structures often develop from structurally independent elements in discourse \citep{Givón1979}. An oft-cited example is the diachronic development of \isi{subordinate clause}s from paratactic\is{parataxis} sentences. As \citet{Lehmann1988} and others have shown, there is a cline of clause linkage ranging from the combination of two structurally independent sentences in discourse to tightly organized bi-clausal structures in which one clause is syntactically dependent on the other one. Building on this observation, it is commonly assumed that \isi{subordinate clause}s have evolved from independent sentences or \isi{parataxis} (e.g. \citealt{HopperTraugott2003}: 176--184). However, while this assumption appears to be plausible for many postposed \isi{subordinate clause}s, there is no obvious connection between \isi{parataxis} and preposed \isi{subordinate clause}s.

Clause combining in discourse has a backwards orientation. Paratactic\is{parataxis} sentences are usually related to previous sentences, as evidenced by the occurrence of anaphoric\is{anaphora} \isi{pronoun}s and clause linkers that connect the current sentence to participants and propositions of the preceding sentence or discourse \REF{ex:diessel:1}.
 

\ea%1
\label{ex:diessel:1}
\ConnectTail{\textit{John\textsubscript{i}}} was accepted to Harvard. \begin{tikzpicture}[baseline,anchor=base,inner xsep=0pt] \node (Diesseltherefore) {\textit{Therefore,}};\draw[dashed,{Circle[]}-{Triangle[]}] (Diesseltherefore.north) -- (Diesseltherefore.north west);\end{tikzpicture} \textit{\ConnectHead*{he\textsubscript{i}}} moved to Boston.
\z


Like independent sentences, complex sentences are processed with a backwards orientation if the \isi{subordinate clause} follows the main clause (e.g. \textit{John\textsubscript{i}} \textit{moved} \textit{to} \textit{Boston,} \textit{because} \textit{he\textsubscript{i}} \textit{was} \textit{accepted} \textit{to} \textit{Harvard}). However, unlike paratactic\is{parataxis} sentences, preposed \isi{subordinate clause}s have an inherent forward orientation in that \isi{pronoun}s and clause linkers are related to elements of the upcoming main clause \REF{ex:diessel:2}.

\ea%2
    \label{ex:diessel:2} \tikz[remember picture,baseline,anchor=base] \node [inner xsep=0pt] (DiesselEx2Because) {\textit{Because}}; \ConnectTail{\textit{he\textsubscript{i}}} was accepted to \tikz[remember picture,baseline,anchor=base] \node [inner xsep=0pt] (DiesselEx2Harvard) {Harvard};, \ConnectHead{\textit{John\textsubscript{i}}} moved to Boston.
    \tikz[remember picture,overlay] \draw[dashed,{Circle[]}-{Triangle[]}] (DiesselEx2Because.north) -- (DiesselEx2Harvard.north east);
  \z

Considering the projective force of preposed \isi{subordinate clause}s, it is unclear if and how these structures have evolved from clause combining strategies in discourse. It is the purpose of this paper to investigate the diachronic developments of preposed \isi{subordinate clause}s from a cross-linguistic perspective. Specifically, the paper is concerned with the development of preposed \isi{adverbial clause}s. 

Following \citet{Cristofaro2003}, \isi{adverbial clause}s are here defined as part of a biclausal construction consisting of a main clause and a \isi{subordinate clause} in which the event designated by the \isi{subordinate clause} specifies the circumstances under which the event of the main clause takes place. Several typological studies have investigated the positional patterns of \isi{adverbial clause}s (e.g. \citealt{Greenberg1963,Diessel2001}; \citealt{Schmidtke-Bode2009,DiesselHetterle2011,Hetterle2015}); but they are either based on small and biased samples or concentrate on particular adverbial relations (e.g. purpose\is{purpose clause} or cause\is{causal clause}). In the current study, we will be concerned with four general semantic types of \isi{adverbial clause}s (i.e. \isi{adverbial clause}s of time\is{temporal clause}, condition\is{conditional clause}, cause\is{causal clause} and purpose\is{purpose clause}) based on data from a genetically and geographically dispersed convenience sample of 100 languages. The languages come from 85 genera (which maximally include two languages) and six large geographical areas (i.e. Eurasia, Africa, South East Asia and Oceania, Australia and New Guinea, North America, South America) (cf. \citealt{Dryer1992}). The bulk of the data were gathered from reference grammars and other published sources, supplemented by information from native speakers and language specialists.\footnote{A list of languages included in the sample is given in the Appendix.} 

The paper is divided into three parts. The first part describes the cross-linguis\-tic distribution of preposed \isi{adverbial clause}s in the 100 language sample; the second part provides an overview of the main diachronic paths to preposed \isi{adverbial clause}s; and the third part considers the developments described in light of the debate about functional and diachronic explanations for language universals that takes center stage in the present volume.

\section{Cross-linguistic patterns}\label{sec:diessel:2}

Let us begin with some general observations regarding the position of \isi{subordinate clause}s. Subordinate clauses are dependent categories of an associated element. Three basic types of \isi{subordinate clause}s are commonly distinguished: 
(i) \isi{complement clause}s, which are dependent categories of a \isi{complement}-taking verb or predicate, 
(ii) \isi{relative clause}s, which are dependent categories of a noun or noun phrase, and 
(iii) \isi{adverbial clause}s, which may be seen as dependent categories of a main clause or main clause predicate. 

The position of all three types of \isi{subordinate clause}s relative to the associated element correlates with the position of other dependent categories relative to the so-called head, but the correlations are skewed in particular directions \citep{Diessel2001}. As \citet{Greenberg1963} already noted, the order of \isi{relative clause} and noun correlates with that of \isi{object} and verb, but there is a predominance of postnominal \isi{relative clause}s. In VO languages, \isi{relative clause}s are almost always postposed to the associated N(P), but in OV languages we find both prenominal and postnominal relatives (cf. \citealt{Dryer2005_Rel}). 

The order of \isi{complement clause} and verb is similar. As \citet{Schmidtke-BodeDiessel2017} have shown, although \isi{object} \isi{complement clause}s usually serve the same \isi{syntactic function} as \isi{object} NPs, they do not always occur in the same structural position as nominal \isi{object}s. In VO languages, \isi{complement clause}s follow the verb with almost no \isi{exception}, but in many OV languages they are postposed to the main verb, as for instance in \ili{Persian}, \ili{Epena Pedee} and \ili{Supyire}. There is thus a general tendency for both relative and \isi{complement clause}s to follow the associated category, which may be due to the oft-noted trend for long and heavy\is{heaviness} constituents to follow short ones (cf. \citealt{Behaghel1932}).
\newpage
However, \isi{adverbial clause}s are different. Although \isi{adverbial clause}s are long constituents, they often precede the main clause. As \citet{Diessel2001} observed (based on data from a small and biased sample), in VO languages, \isi{adverbial clause}s occur both before and after the associated main clause, but in some OV languages, there is a general tendency to prepose all \isi{adverbial clause}s. This tendency is also evident in the current sample (cf. \tabref{tab:diessel:1}).

\begin{table}
\small
\begin{tabularx}{\textwidth}{lYYYr}
\lsptoprule
&   Languages in which all types of ACs (usually) precede the MC &   Languages in which ACs are commonly pre- and postposed &   Languages in which all types of ACs (usually) follow the MC &   Total\\
\midrule 
VO & - & 40 & - & 40\\
VO/OV & - & 8 & - & 8\\
OV & 31 & 21 & - & 52\\
\midrule
Total & 31 & 69 & - & 100\\
\lspbottomrule
\end{tabularx}

\caption{The order of adverbial clause (AC) and main clause (MC) and the order of verb and object}
\label{tab:diessel:1}
\end{table}

As can be seen, most of the languages of the current sample make common use of both pre- and postposed \isi{adverbial clause}s, but in more than half of all OV languages, \isi{adverbial clause}s are usually preposed to the main clause. In \ili{Japanese}, for instance, there is a very strong tendency to prepose \isi{adverbial clause}s (though in spoken \ili{Japanese}, \isi{adverbial clause}s sometimes follow the main clause as afterthoughts; cf. \citealt{FordMori1994}).

Generalizing across the data in \tabref{tab:diessel:1}, we may say that while the order of \isi{adverbial clause} and main clause correlates with that of \isi{object} and verb, the occurrence of preposed \isi{adverbial clause}s is cross-linguistically predominant. However, on closer inspection we find that the predominance of preposed \isi{adverbial clause}s is mainly due to certain semantic types of \isi{adverbial clause}s that precede the main clause in both VO and OV languages. Consider the data in \tabref{tab:diessel:2}, which show that the positional patterns of \isi{adverbial clause}s correlate with their meaning.  

\begin{table}
\begin{tabularx}{\textwidth}{Xr@{ }rcr@{ }rr} 
\lsptoprule
&   \multicolumn{2}{c}{Preposed} &  Pre- and postposed &   \multicolumn{2}{c}{Postposed} &   Total\\
\midrule
Condition & 94  & [91.3\%] & \begin{tabular}{@{}r@{ }r@{}} 9 & \hphantom{3}[8.7\%] \end{tabular}  & 0  & [0\%] & 103\\
Time      & 119 & [59.8\%] & \begin{tabular}{@{}r@{ }r@{}}68 & [34.2\%] \end{tabular} & 12 & [6.0\%] & 199\\
Cause     & 40  & [38.8\%] & \begin{tabular}{@{}r@{ }r@{}}24 & [21.2\%] \end{tabular} & 49 & [43.4\%] & 113\\
Purpose   & 33  & [28.7\%] & \begin{tabular}{@{}r@{ }r@{}}19 & [16.5\%] \end{tabular} & 63 & [54.8\%] & 115\\
\midrule 
Total & 286 & & \begin{tabular}{@{}r@{ }r@{}}120 & \hphantom{[25.5\%]}\end{tabular} & 124 & & 530\\
\lspbottomrule
\end{tabularx}

\caption{The meaning and position of adverbial clause constructions in a sample of 100 languages}
\label{tab:diessel:2}
\end{table}

Note that the frequencies in \tabref{tab:diessel:2} are based on constructions rather than on languages. Since some languages have multiple \isi{adverbial clause} constructions of the same semantic type, \tabref{tab:diessel:2} includes a larger number of constructions than languages. Note also that this table concerns both \isi{adverbial clause}s that are tied to a specific position by linguistic convention and \isi{adverbial clause}s that are statistically biased to precede or follow the main clause. In the latter case, some of the data in \tabref{tab:diessel:2} are based on \isi{frequency} counts from linguistic corpora, but more often these data are based on field workers’ judgements regarding the position of \isi{adverbial clause}s. While expert judgements are less reliable than \isi{corpus} counts, they provide a reasonable estimate as to how main and \isi{adverbial clause}s are arranged in a particular language.\footnote{Psycholinguistic evidence suggests that while speakers have difficulties to estimate the absolute frequencies\is{frequency} of linguistic elements, their judgements of relative linguistic frequencies\is{frequency} are quite reliable (\citealt{HasherZacks1984}).}

As can be seen, \isi{conditional clause}s typically precede the main clause (cf. \citealt{Greenberg1963}: Universal 14), though in many languages, \isi{conditional clause}s can also be postposed to the main clause. Like \isi{conditional clause}s, \isi{temporal clause}s tend to precede the main clause, but \isi{temporal clause}s follow the main clause more often than conditionals\is{conditional clause}. The position of \isi{temporal clause}s varies with the nature of the temporal\is{temporal clause} link they encode. For instance, \isi{temporal clause}s denoting a prior event, i.e. an event that precedes the one in the main clause, are more often preposed than \isi{temporal clause}s denoting a posterior event. In \ili{English}, for example, \textit{after}- and \textit{since}-clauses denote a prior event and precede the main clause more often than \isi{adverbial clause}s denoting a posterior event such as \textit{before}- and \textit{until}-clauses (cf. \citealt{Diessel2008}). The same tendency has been observed in several other languages of the current sample (e.g. in \ili{German}, \ili{Supyire}, \ili{Abun}, \ili{Nkore Kiga}, \ili{Noon}, and \ili{Taba}). 

Moreover, and this is particularly striking, there is a general tendency to prepose \isi{adverbial clause}s that correspond to \ili{English} \textit{when}-clauses. Like \textit{after} and \textit{since}, \textit{when} can denote a prior event, but it can also indicate a link between events that occur simultaneously \citep{Diessel2008}. However, regardless of the temporal\is{temporal clause} relationship that is expressed by a \textit{when}-clause, there is a tendency for temporal\is{temporal clause} \textit{when}-clauses to precede the main clause. In fact, in a substantial number of languages \textit{when}-clauses are generally preposed to the main clause in the current sample (i.e. \ili{Abun}, \ili{Supyire}, \ili{Yagua}, \ili{Trumai}, \ili{Motuna}). 

Finally, cause\is{causal clause} and \isi{purpose clause}s tend to follow the main clause. \tabref{tab:diessel:2} shows that there are 40 \isi{adverbial clause} constructions of cause\is{causal clause} and 33 \isi{adverbial clause} constructions of purpose\is{purpose clause} that precede the main clause, but most of these constructions occur in languages like \ili{Japanese}, in which all \isi{adverbial clause}s are preposed to the main clause regardless of their meaning. Generalizing across these findings we may conclude that the cross-linguistic tendency to prepose \isi{adverbial clause}s is mainly due to the fact that \isi{conditional clause}s and certain types of \isi{temporal clause}s, notably \textit{when}-clauses, precede the main clause regardless of the order of other syntactic constituents.

Interestingly, a number of studies suggest that the position of \isi{adverbial clause}s does not only correlate with the semantic link between main clauses and \isi{adverbial clause}s, but also with aspects of their internal structure. Of particular importance here is the position of the \isi{subordinator} (cf. \citealt{Diessel2001}; \citealt{Schmidtke-Bode2009,Hetterle2015}). Across languages, \isi{adverbial clause}s are often marked by subordinate \isi{conjunction}s that typically appear at the beginning or end of the \isi{subordinate clause}. \citet{Dryer1992} showed that the position of the \isi{subordinator} correlates with the order of verb and \isi{object}: In VO languages, \isi{adverbial clause}s usually occur with initial \isi{subordinator}s, but in OV languages they often include a final marker. However, the position of the \isi{subordinator} does not only correlate with the order of verb and \isi{object}, it also correlates with the position of the \isi{adverbial clause}. Consider the data in \tabref{tab:diessel:3}, which is restricted to \isi{adverbial clause}s with free subordinating morphemes.\footnote{Since \isi{adverbial clause} constructions that do not include a free subordinating morpheme are disregarded, \tabref{tab:diessel:3} includes only a subset of the \isi{adverbial clause} constructions in \tabref{tab:diessel:2}.} 

\begin{table}
\begin{tabularx}{\textwidth}{X rr  rr rr  r}
\lsptoprule
& & & \multicolumn{2}{c}{Flexible}\\
& \multicolumn{2}{c}{Preposed} &  \multicolumn{2}{c}{(no preference)} &  \multicolumn{2}{c}{Postposed} \\
\cmidrule(lr){2-3}\cmidrule(lr){4-5}\cmidrule(lr){6-7}
&   Initial &   Final &   Initial &   Final &   Initial &   Final & Total\\
\midrule
{Condition} 	& 34 	& 22 	& 5 	& - 	& - 	& - 	& 61\\
{Time} 		& 20 	& 47 	& 43 	& 5 	& 7 	& 3 	& 125\\
{Cause} 	& 2 	& 26 	& 11 	& 6 	& 37 	& 4 	& 86\\
{Purpose} 	& - 	& 20 	& 2 	& 4 	& 38 	& 4 	& 68\\
\midrule
{Total} 	& 56 	& 115 	& 61 	& 15 	& 82 	& 11 	& 340\\
\lspbottomrule
\end{tabularx}

\caption{The position of free subordinators in pre- and postposed adverbial clauses}
\label{tab:diessel:3}
\end{table}
\largerpage
As can be seen,\isi{adverbial clause}s that follow the main clause or that are flexible with regard to their position typically occur with an initial marker. There are languages in which postposed and flexible \isi{adverbial clause}s include a final marker, but this is relatively rare (and mainly found in certain areas, e.g. South America). By contrast, preposed \isi{adverbial clause}s are frequently marked by a final \isi{subordinator}, especially in languages in which all \isi{adverbial clause}s precede the main clause, as for instance in \ili{Amele}, \ili{Burmese}, \ili{Japanese}, \ili{Korafe}, \ili{Korean}, \ili{Santali}, \ili{Slave}, \ili{Turkish}, \ili{Wappo}, \ili{Warao}, and \ili{Menya}. Only \isi{conditional clause}s and temporal\is{temporal clause} \textit{when}-clauses are commonly preposed and often marked       by an initial \isi{subordinator} (in languages in which other semantic types of \isi{adverbial clause}s \largerpage are flexible or postposed to the main clause). 

\section{Diachronic sources}\label{sec:diessel:3}

Having described the positional patterns of \isi{adverbial clause}s (and adverbial \isi{subordinator}s), let us now consider their diachronic \isi{evolution}. Where do preposed \isi{adverbial clause}s come from? In the historical literature, syntactic development is commonly described as a process that leads from a source construction A to a target construction B, but this scenario is not always appropriate to characterize syntactic change (cf. \citealt{Givón1991};  \citealt{VandeVeldeEtAl2013}). Since \isi{subordinate clause}s are complex grammatical units, they are usually related to several other constructions, e.g. other types of \isi{subordinate clause}s, certain types of phrasal constituents and independent sentences. Since all of these constructions can influence the development of a particular \isi{adverbial clause}, it is not always possible to trace \isi{adverbial clause}s to one specific source. However, while the diachronic developments of \isi{adverbial clause}s are (usually) influenced by several constructions, in many cases there is one construction that is so closely related to a certain type of \isi{adverbial clause} that it can be seen as the primary determinant, or source, of that clause. For instance, many postposed \isi{adverbial clause}s are so similar to paratactic\is{parataxis} sentences that it seems reasonable to assume that \isi{parataxis} has a significant impact on the development of (many) postposed \isi{subordinate clause}s. However, while the development from \isi{parataxis} provides a plausible scenario for the rise of (many) postposed \isi{adverbial clause}, it does not explain where preposed \isi{adverbial clause}s come from.

Since preposed \isi{adverbial clause}s are thematically related to the ensuing discourse, there is no obvious connection to \isi{parataxis} unless we assume that preposed \isi{adverbial clause}s are based on postposed \isi{subordinate clause}s that were fronted after they developed from paratactic\is{parataxis} sentences. However, there is no evidence for this scenario. The diachronic developments of \isi{adverbial clause}s have been examined in a large number of studies (e.g. \citealt{Haiman1985,Haspelmath1989,Givón1991,Genetti1991,HarrisCampbell1995,Frajzyngier1996,DisterheftViti2010}), but although fronting appears to provide a plausible scenario for the development of preposed \isi{adverbial clause}s, there is almost no evidence for this scenario in the historical literature. On the contrary, what previous studies suggest is that \isi{adverbial clause}s usually occur in the same position as their diachronic sources. In what follows, we consider four common source constructions for preposed \isi{adverbial clause}s.

First, while preposed \isi{adverbial clause}s are unlikely to have evolved from paratactic\is{parataxis} sentences through fronting, there is one common diachronic path that leads from independent sentences in discourse to complex sentences with preposed \isi{adverbial clause}s. As \citet[39--70]{Haiman1985} observed, in many languages conditional\is{conditional clause} relations are expressed by juxtaposed\is{juxtaposition} clauses that have the same structure as two simple sentences, as in the following examples from \ili{Vietnamese} \REF{ex:diessel:3}, \ili{Mapudungun} \REF{ex:diessel:4} and \ili{Wambaya} \REF{ex:diessel:5}.

\ea\label{ex:diessel:3}
\langinfo{Vietnamese}{Austro-Asiatic, Viet-Muong}{\citealt{Haiman1985}: 45}\\
\gll  [Không  có   màn],  không  chịu  nôi.\\
       not  be   net  not  bear  can\\
\glt   `If there’s no net, you can’t stand it.'
\z

\ea\label{ex:diessel:4}
\langinfo{Mapudungun}{Araucanian}{\citealt{Smeets2008}: 184}\\
\gll   [Aku-wye-fu-l-m-i],    pe-pa-ya-fwi-y-m-i.\\
       arrive-\textsc{plpf}-\textsc{ipd}-\textsc{cond}-2-\textsc{sg}  see-hither-\textsc{irr}-\textsc{obj}-\textsc{ind}-2-\textsc{sg}\\
\glt   `If you had arrived (by then), you would have seen him.'
\z

\ea\label{ex:diessel:5}
\langinfo{Wambaya}{West Barkly}{\citealt{Nordlinger1998}: 219}\\
\gll   [Yabu  ng-uda  gijilulu]            jiyawu  ng-uda.\\
       have    \textsc{1sg}.\textsc{a-nact.pst}  money.\textsc{iv(acc)}  give  1\textsc{sg.a-nact.pst}\\
\glt   `If I’d had the money I would have given (it to her).'
\z

While some of these languages have conditional\is{conditional clause} markers (e.g. \ili{Vietnamese} \textit{nêu} ‘if’), conditional\is{conditional clause} relations are commonly expressed by unmarked\is{zero marking} sentences that have the same structure as main clauses: they include \isi{finite} verb forms, occur with the same arguments and \isi{adjunct}s as independent sentences, and do not include an (obligatory) subordinate marker\is{subordinator}. Note, however, that while these constructions look like independent sentences, they are \isi{intonation}ally bound to the ensuing clause and sometimes constrained with regard to verb inflection. The \isi{conditional clause} in \ili{Mapudungun}, for instance, takes a \isi{mood} suffix that is optional in main clauses but obligatory in conditionals\is{conditional clause}. Moreover, in some languages these constructions occur with a \isi{topic} or \isi{focus} marker that one might analyze as a \isi{subordinator}, such as the \isi{focus} clitic at the end of the protasis in example \REF{ex:diessel:6} from \ili{Mangarayi}.

\ea\label{ex:diessel:6}
\langinfo{Mangarayi}{Isolate}{\citealt{Merlan1982}: 22}\\
\gll   [ña-yaŋ-gu=\textbf{bayi}]   wawg   wa-ñan-mi  biwin-gana.\\
       \textsc{2sg}-go-\textsc{di-foc}    follow   \textsc{irr-1sg>2sg-aux}  behind-\textsc{abl}\\
\glt   `If you go, I will follow (after) you.'
\z

In addition to \isi{conditional clause}s, preposed \isi{temporal clause}s are sometimes based on juxtaposed\is{juxtaposition} sentences (e.g. in \ili{Lao}, \ili{Vietnamese}, \ili{Taba}, \ili{Tetun}, \ili{Gooniyandi}); but preposed cause\is{causal clause} and \isi{purpose clause}s are usually based on other types of constructions. Adpositional\is{adposition} phrases, for instance, are often closely related to (preposed) cause\is{causal clause} and \isi{purpose clause}s. Consider, for instance, the following examples from \ili{Turkish} \REF{ex:diessel:7} and \ili{Amele} \REF{ex:diessel:8}, in which cause\is{causal clause} and \isi{purpose clause}s are marked by \isi{benefactive} \isi{postposition}s. 

\ea\label{ex:diessel:7}
\langinfo{Turkish}{Turkic}{\citealt{Kornfilt1997}: 74}\\
\gll   Hasan   [kitab-ı  san-a  ver-diğ-im  \textbf{için}]  çok  kız-dı.\\
       Hasan  book-\textsc{acc}  you-\textsc{dat}  give-\textsc{f.nmlz-1sg}  for  very  angry-\textsc{pst}  \\
\glt   `Hasan got very angry because I gave the book to you.'
\z

\ea\label{ex:diessel:8}
\langinfo{Amele}{Nuclear Trans New Guinea, Madang}{\citealt{Roberts1987}: 58}\\
\gll   [Ija    sab    faj-ec   \textbf{nu}]  h-ug-a.\\
       \textsc{1sg}   food   buy-\textsc{inf/nmlz}  for  come-\textsc{1sg-pst}\\
\glt   `I came to buy food.'
\z

\noindent Note that the \isi{adverbial clause}s in both examples are expressed by \isi{nominalization}s. While \isi{adposition}s and \isi{case} affixes are also found with \isi{finite} clauses, they are especially frequent with nominalized\is{nominalization} clauses, suggesting that \isi{nominalization} provides a link between \isi{adposition}al phrases and fully developed (subordinate) clauses (cf. \citealt{Deutscher2009,Heine2009}).

Adverbial\is{adverbial clause} clauses that are morphologically related to \isi{adposition}al phrases are widely used to express semantic relations of cause\is{causal clause} and purpose\is{purpose clause}. In addition, certain types of \isi{temporal clause}s denoting a prior or posterior event are often strikingly similar to (temporal) \isi{adposition}al phrases (e.g. \ili{English} \textit{after-,} \textit{since-} and \textit{before-}clauses) (\citealt{Blake1999,Hetterle2015}); but \isi{conditional clause}s and temporal\is{temporal clause} \textit{when}-clauses are only rarely marked by \isi{adposition}s. 

Apart from juxtaposed\is{juxtaposition} sentences and \isi{adposition}al phrases, \isi{relative clause}s provide a very frequent source for (preposed) \isi{adverbial clause}s. The development is well-known from \ili{English}. As \citet{HopperTraugott2003} have shown, temporal\is{temporal clause} \textit{while}-clauses have evolved from a relative or appositive construction\is{appositive clause} that modified a generic head noun meaning ‘time’ \REF{ex:diessel:9}.

\ea\label{ex:diessel:9}
\langinfo{Old English}{\ili{Indo-European}, \ili{Germanic}}{\citealt{HopperTraugott2003}: 90}\\
\gll   \& wicode    Þær   Þa   \textbf{hwile}  [Þe   man  Þa   burg  worthe  \& getimbrode].\\
       and  lived   there   that.\textsc{dat}   time.\textsc{dat}  that   one  that   fortress  worked.on  and built\\
\glt `… and camped there at the time that/while the fortress was worked on and built.'
\z

Similar types of \isi{adverbial clause}s occur in many other languages of the current sample (e.g. in \ili{Mayogo} \REF{ex:diessel:10} and \ili{Toqabaqita} \REF{ex:diessel:11}). Sometimes the \isi{subordinator} is based on a generic noun, and sometimes it is based on a relative marker (as for instance many of the adverbial \isi{subordinator}s in \ili{Tamasheq}; cf. \citealt{Heath2005}: 660). 

\ea\label{ex:diessel:10}
\langinfo{Mayogo}{\ili{Niger-Congo}, Ubangi}{\citealt{Sawka2001}: 153}\\
\gll   [\textbf{Nedhɨnga}  u   a-zʉ   ‘he],   ndɨlɨ-e   a-sɨ   kuto.\\
        \textbf{while/time} \textsc{3pl}   \textsc{pst-}eat   thing   child-\textsc{ref}   \textsc{pst-}sleep   down\\
\glt   `While they ate something, this child slept on (the) floor.'
\z

\ea\label{ex:diessel:11}
\langinfo{Toqabaqita}{\ili{Austronesian}, \ili{Oceanic}}{\citealt{Lichtenberk2008}: 1173}\\
\gll   [\textbf{Si}   \textbf{manga}  \textbf{na}   kero  fula  mai],  keko  qono  qa-daroqa\\
        \textsc{prtt}   time  \textsc{rel}   3\textsc{du.non.fut}  arrive  \textsc{vent}  \textsc{3du.seq}  sit  \textsc{sben-3du.pers}\\
\glt   `When (lit. `the time that’) they arrived, they sat (down) …'
\z

The development is especially frequent with temporal\is{temporal clause} \textit{when}- and \textit{while}-clauses, but there are also other semantic types of \isi{adverbial clause}s that are based on \isi{relative clause}s in my data. In \ili{German}, for instance, cause\is{causal clause} and condition\is{conditional clause} clauses are marked by adverbial \isi{subordinator}s (i.e. \textit{weil} and \textit{falls}) that are based on nominal heads of relative\is{relative clause} or \isi{appositive clause}s meaning ‘time (span)’ and ‘case’. Moreover, at least 25 languages of the current sample have \isi{conditional clause}s based on temporal\is{temporal clause} \textit{when/while}-clauses (which at least in some cases are ultimately based on \isi{relative clause}s). Note that this development does not only involve postnominal relatives but also prenominal and internally headed \isi{relative clause}s, as illustrated by the following examples from \ili{Amele} \REF{ex:diessel:12}, \ili{Korean} \REF{ex:diessel:13} and \ili{Jamsay} \REF{ex:diessel:14}.\footnote{According to \citet{Epps2009_RC}, \ili{Hup} has \isi{adverbial clause}s that are based on headless \isi{relative clause}s.} 

\ea\label{ex:diessel:12}
\langinfo{Amele}{Nuclear Trans New Guinea, Madang}{\citealt{Roberts1987}: 57}\\
\gll   [Ija  cabi   meul   ceh-ig-en   \textbf{sain}   \textbf{eu}   \textbf{na}]   ma   ca  ceta  ca\\
        1\textsc{sg}   garden   new   plant-1\textsc{sg-fut}   \textbf{time}   \textbf{that}    \textbf{at}   taro   add  yam  add  \\
\gll   mun    ca    manin    ca  ceh-ig-en.\\
       banana   add bean    add  plant-\textsc{1sg-fut}\\
\glt `When I plant my new garden, I will plant taro, yam, banana and beans.'
\z

\ea\label{ex:diessel:13}
\langinfo{Korean}{Isolate}{\citealt{Sohn1994}: 70}\\
\gll   Na-nun  [pi-ka  w-ass-ul   \textbf{ttay-(ey)}]    ttena-ss-ta.\\
       I-\textsc{tc}  rain-\textsc{nom}  come-\textsc{pst-prosp}  time-at     leave-\textsc{pst-decl}\\
\glt   `I left when it had rained.' 
\z

\ea\label{ex:diessel:14}
\langinfo{Jamsay}{Dogon}{\citealt{Heath2008}: 559}\\
\gll   [Wárú  \textbf{dògùrù}   ù  gô:-Ø] …\\
       farming  time   2\textsc{sg.sbj}  go:out.\textsc{pfv}-\textsc{ptcp.non.human} \\
\glt   `At the time when you (first) went out to do the farming, …'
\z

Finally, preposed \isi{adverbial clause}s are also often influenced by \isi{complement clause}s. In Middle \ili{English}, for instance, adverbial \isi{subordinator}s were frequently accompanied by the \isi{complementizer} \textit{that} \textit{(e.g.} \textit{after} \textit{that,} \textit{since} \textit{that,} \textit{gif} \textit{that}), which is still commonly used in \isi{result clause}s (cf. \textit{so} \textit{that}). Likewise, in Chalcatongo Mixtec\il{Mixtec (Chalcatongo)}, most \isi{adverbial clause}s are marked by the \isi{complementizer} \textit{xa=}, which also appears in complement and \isi{relative clause}s (\citealt{Macaulay1996}: 156--168). Moreover, there is a well-known path that leads from \isi{quotative} constructions, which in many languages are similar to \isi{complement clause}s, to \isi{adverbial clause}s. In particular, purpose\is{purpose clause} and cause\is{causal clause} clauses are sometimes derived from \isi{quotative}s (cf. \citealt{Güldemann2008}). 

Quotative\is{quotative} constructions consist of a “quote index”, including a “quotative mark\-er”, and a “quote clause” of direct speech that often shows little evidence for \isi{embedding} (cf. \citealt{Güldemann2008}). In many cases, the \isi{quotative} marker is a general verb of saying (e.g. ‘say’, ‘speak’), but it can also be a marker of similarity (e.g. ‘like’) or a \isi{manner} deictic\is{deixis} (e.g. ‘so’). Although quote clauses\is{quotative} are often not embedded\is{embedding} in the associated clause, the \isi{quotative} verb takes the quote clause\is{quotative} as some kind of semantic argument, which typically occurs in the same position as a direct \isi{object}.\footnote{\citet{Munro1982} and \citet{Güldemann2008} point out that quote clauses\is{quotative} do not generally occur in the same position as direct \isi{object}s, which is one reason why these researchers argue that quote clauses\is{quotative} are not (always) \isi{complement}s. However, while quote clauses\is{quotative} are often less tightly integrated into a clause (or VP) than direct \isi{object}s, they are related to \isi{object} \isi{complement clause}s by family resemblance and since \isi{object} \isi{complement clause}s pattern with \isi{object} NPs, there is also a tendency for quote clauses\is{quotative} to occur in the same position as direct \isi{object}s (see \citealt{Schmidtke-BodeDiessel2017} for some discussion of the relationship between quote clauses\is{quotative}, \isi{object} \isi{complement clause}s and nominal \isi{object}s from a cross-linguistic perspective).} When this happens in OV languages, the consequence is that quote clauses\is{quotative} precede the \isi{quotative} verb. If these constructions are extended into the domain of adverbial subordination\is{subordinate clause}, the \isi{adverbial clause} is preposed to the main clause (or main verb) and marked by a clause-final \isi{subordinator} that is ultimately based on the \isi{quotative} verb, as in the following examples from \ili{Aguaruna} \REF{ex:diessel:15} and \ili{Lezgian} \REF{ex:diessel:16}.

\ea\label{ex:diessel:15}
\langinfo{Aguaruna}{Jivaroan}{\citealt{Overall2009}: 175}\\
\gll   Nuwa-na  [yumi  ʃikika-ta  \textbf{tu-sã}]  awɨma-wa.\\
       woman-\textsc{acc}  water  draw.\textsc{asp-imp}  say-\textsc{sub.3.ss}  send.\textsc{asp-non.a/s>a/s}\\
\glt “When (they) sent a woman to draw water, ….' (lit. ‘saying “draw some water, …”’) 
\z

\ea\label{ex:diessel:16}
\langinfo{Lezgian}{Nakh-Daghestanian}{\citealt{Haspelmath1993}: 390}\\
\gll   Bazar.di-n  juğ  ada-z  [tars-ar  awa-č   \textbf{luhuz}]  tak’an  \^{x}a-nwa-j.\\
       Sunday-\textsc{gen}  day  he-\textsc{dat}  lesson-\textsc{pl}  be.in-\textsc{neg}   saying  hateful  become-\textsc{prf-pst}\\
\glt   `He hated Sunday because there were no lessons.'
\z

\begin{table}
\begin{tabularx}{\textwidth}{lQ}
\lsptoprule
Condition & juxtaposed sentences (\citealt{Haiman1985})\newline 	  temporal \textit{when/while-}clauses (\citealt{Traugott1985})\\
\tablevspace
Time      & adpositional phrases / nominalizations (\citealt{Genetti1991})\newline 	  pre- and postnominal relative clauses (\citealt{Givón1991})\\
\tablevspace
Cause    &  adpositional phrases / nominalizations (\citealt{Genetti1991})\newline 	  \isi{quotative} / complement constructions\is{complement clause} (\citealt{Ebert1991})\\
\tablevspace
Purpose &  adpositional phrases / nominalizations (\citealt{Schmidtke-Bode2009})\newline  \isi{quotative} / complement constructions\is{complement clause} (\citealt{Güldemann2008})\\
\lspbottomrule
\end{tabularx}

\caption{Frequent source constructions of preposed adverbial clauses}
\label{tab:diessel:4}
\end{table}

\tabref{tab:diessel:4} provides an overview of the various sources for preposed \isi{adverbial clause}s considered in this section. Let me emphasize that this table simplifies in several ways. First, as pointed out above, the development of \isi{adverbial clause}s is usually influenced by multiple constructions so that there are often several source constructions (though one of them is often dominant). Second, there are frequent diachronic connections between the various semantic types of \isi{adverbial clause}s that are not indicated in \tabref{tab:diessel:4} except for the development of temporal\is{temporal clause} \textit{when/while}-clauses into \isi{conditional clause}s, which is particularly frequent. Third, there is reason to assume that postposed \isi{adverbial clause}s can influence the structure of preposed \isi{adverbial clause}s through analogical extension\is{analogy} (cf. \citealt{Traugott1985}). Fourth, in addition to the eight source constructions shown in \tabref{tab:diessel:4}, there are other (less frequent) source constructions of preposed \isi{adverbial clause}s that have been disregarded. And finally, there is evidence that constructional change typically proceeds in a local fashion that is driven by language users’ experience with particular lexical expressions (e.g. \citealt{Givón1991}), but this has been ignored in the above discussion. In order to account for all of these factors, one would need a different theoretical approach — perhaps some kind of network model, in which \isi{adverbial clause}s are linked to various other types of constructions that simultaneously affect their use and their development (see \citealt{Diessel2015} for some discussion of such a model). However, in what follows we concentrate on the idealized developments that are summarized in \tabref{tab:diessel:4}.

\section{Discussion: Functional adaptation and/or persistence}
\label{sec:diessel:4}

To recapitulate, we have seen that the occurrence of preposed \isi{adverbial clause}s correlates with the position of other grammatical categories and the semantic relationship between main and \isi{adverbial clause} (\sectref{sec:diessel:2}), and we have seen that condition\is{conditional clause}, time\is{temporal clause}, cause\is{causal clause} and \isi{purpose clause}s develop from, or under the influence of, a wide range of constructions (\sectref{sec:diessel:3}). Concluding the paper, let us ask what leads to the development and cross-linguistic distribution of preposed \isi{adverbial clause}s.

Many linguistic typologists assume that language universals are motivated by semantic and pragmatic\is{pragmatics} factors that influence the diachronic developments of linguistic structure. On this view, cross-linguistic regularities are functional \isi{adaptation}s to communication and \isi{processing} (e.g. \citealt{FoleyVanValin1984,Dik1989,Hawkins2004}). However, as particularly \citetv{chapters/cristofaro} and \citetv{chapters/collins} argue, there is an alternative approach that stresses the importance of diachronic inheritance, or \isi{persistence}, for the rise of language universals. In this approach, cross-linguistic tendencies, or statistical universals, are the by-product of diachronic processes that are \textsc{not} immediately motivated by functional-adaptive\is{adaptation} factors. In the remainder of this paper, I argue that the cross-linguistic tendencies in the linear organization of \isi{adverbial clause}s are the result of both functional aspects of language use and \isi{persistence} effects of \isi{grammaticalization}. 

\textsc{preposed} \textsc{adverbial} \textsc{clauses} \textsc{in} \textsc{head-final} \textsc{languages}. Given that clause combining in discourse has a strong backwards orientation, one might wonder why \isi{adverbial clause}s are not generally postposed. However, there are several reasons why languages prepose \isi{adverbial clause}s. To begin with, above we have seen that the positional patterns of \isi{adverbial clause}s vary with their meaning, but in some rigid OV languages, they are consistently preposed to the main clause, suggesting that the order of main and \isi{adverbial clause}s is part of the traditional VO/OV typology (cf. \citealt{Diessel2001}).

There are numerous proposals in the literature to explain correlations between the order of verb and \isi{object} and that of other grammatical categories. Especially prominent is Hawkins’ \isi{processing} approach, in which word order correlations\is{word-order correlation} are explained by general principles of syntactic \isi{processing} that are assumed to influence both language use and language change (cf. \citealt{Hawkins1994, Hawkins2004}). Specifically, Hawkins proposed that head-final or OV languages tend to prepose dependent categories, including \isi{subordinate clause}s, because syntactic structures with consistent dependent-head orders are easier to process, and thus more strongly preferred, than structures with mixed or inconsistent dependent-head orders (see \citealt{Dryer1992} for a similar explanation).

\largerpage
The \isi{processing} approach provides a straightforward explanation for the dominant use of preposed \isi{adverbial clause}s in OV languages, but as \citet{Krifka1985} and others have noted, word order correlations\is{word-order correlation} can also be explained by \isi{analogy} or similarity. There is abundant evidence from \isi{psycholinguistic} research that speakers tend to arrange semantically or formally similar expressions in parallel positions (see \citealt{PickeringFerreira2008} for a review of \isi{psycholinguistic} research on the influence of structural \isi{priming} on linear order). Like \isi{object}s, \isi{adverbial clause}s are dependent categories, but other than that, \isi{adverbial clause}s do not seem to have much in common with \isi{object} NPs, making it rather unlikely that \isi{analogy} and similarity account for this correlation. However, if we broaden the perspective and include other types of constructions into the analysis, there is reason to assume that the correlation between \isi{adverbial clause}s and \isi{object} NPs is due to analogical\is{analogy} pressure that affects a whole network of constructions.


To begin with, \isi{adverbial clause}s are often similar to \isi{adposition}al phrases functioning as \isi{adjunct}s, and since the latter are usually similar to \isi{object} NPs, \isi{adverbial clause}s are also related to direct \isi{object}s (via \isi{adjunct}s). As \citet{Dryer1992} showed, there is a very strong tendency to place nominal \isi{object}s and (certain semantic types of) \isi{adjunct}s on the same side of the verb and since \isi{adverbial clause}s pattern like \isi{adjunct} phrases, they also pattern with \isi{object} NPs. 

Moreover, in many languages, \isi{adverbial clause}s are expressed by the same or very similar types of constructions as \isi{complement clause}s. Since \isi{complement clause}s are also related to \isi{object} NPs, we may hypothesize that the ordering correlation\is{word-order correlation} between complex sentences including \isi{adverbial clause}s and verb phrases including nominal \isi{object}s is (also) mediated by constructions including \isi{complement clause}s, as the latter share properties with both of them.

Thus, while \isi{adverbial clause}s do not have much in common with \isi{object} NPs, they are similar to \isi{adposition}al phrases and \isi{complement clause}s, which in turn are similar to nominal \isi{object}s, suggesting that OV (or head-final) languages prepose \isi{adverbial clause}s in \isi{analogy} to (preposed) \isi{adposition}al phrases and \isi{complement clause}s \REF{ex:diessel:17}.

\ea\label{ex:diessel:17}~\\
\begin{tikzpicture}
\node at (0,0) [name=diamond,shape=diamond,aspect=.5,draw,minimum width=7cm] {};
\node at (diamond.east)  [fill=white] {AC–V};
\node at (diamond.south) [fill=white]  {PP–V};
\node at (diamond.north) [fill=white]  {CC–V};
\node at (diamond.west)  [fill=white] {NP–V};
\end{tikzpicture}
\z

\textsc{preposed} \textsc{adverbial} \textsc{clauses} \textsc{in} \textsc{head-initial} \textsc{languages}. Analogy\is{analogy} is one factor that can motivate the occurrence of preposed \isi{adverbial clause}s in head-final languages, but since the occurrence of preposed \isi{adverbial clause}s is not restricted to head-final languages, \isi{analogy} alone is not sufficient to explain why \isi{adverbial clause}s are commonly preposed. As we have seen, certain semantic types of \isi{adverbial clause}s, notably \isi{conditional clause}s and temporal\is{temporal clause} \textit{when}-clauses, precede the main clause in both head-initial and head-final languages. In order to explain these patterns, we have to consider the semantic and discourse-pragmatic properties of \isi{adverbial clause}s.

As \citet{Chafe1984}, \citet{Givón1984} and many others have pointed out, preposed \isi{adverbial clause}s serve particular discourse-organizing functions. They provide a thematic ground or orientation for subsequent information, as evidenced by the fact that preposed \isi{adverbial clause}s are often marked as \isi{topic}s \citep{Haiman1978}. In addition, there are particular conceptual motivations to prepose certain semantic types of \isi{adverbial clause}s. Conditional clauses\is{conditional clause}, for instance, exhibit a strong tendency to precede the main clause, as conditionals\is{conditional clause} are used to create a particular conceptual framework for the semantic interpretation of associated clauses \citep{Diessel2005}, and some \isi{temporal clause}s precede the main clause for reasons of \isi{iconicity} \citep{Diessel2008}. 

Considering the semantic and discourse-pragmatic functions of preposed \isi{adverbial clause}s, we may hypothesize that these functions do not only influence speakers’ use of a particular clause order (where there is synchronic choice) but also the development of preposed \isi{adverbial clause}s in language change or language \isi{evolution}. In particular, the initial stages of the development seem to be motivated by semantic and discourse-pragmatic factors. For instance, as we have seen, \isi{adverbial clause}s are often based on \isi{relative clause}s and \isi{adposition}al phrases, which in VO languages usually follow the main verb (if we disregard center-embedded\is{embedding} RCs\is{relative clause}), but may be fronted in order to provide an orientation, or \isi{topic}, for the unfolding sentence. When the fronted constructions are routinely used for discourse-organizing functions, they may develop into preposed \isi{adverbial clause}s with the same or similar functions. 

Assuming that preposed \isi{adverbial clause}s inherit their discourse functions from fronted \isi{relative clause}s, \isi{adposition}al phrases and similar constructions, one might argue that while discourse considerations motivate the use of the various source constructions, they do not immediately motivate the extension of these constructions to \isi{adverbial clause}s, which 
\label{p:xxx:automatization}
seems to be a consequence of \isi{automatization}, semantic \isi{bleaching} and formal reduction\is{phonetic reduction} rather than of discourse \isi{processing}. However, since \isi{grammaticalization} is a gradual process with no sharp division between source and target, I would contend that the influence of discourse is not restricted to the initial uses of the source constructions but affects the entire course of the development. After all, \isi{automatization}, semantic \isi{bleaching} and formal reduction\is{phonetic reduction} are driven by \isi{frequency} of language use, which in turn is driven by the need to use fronted \isi{relative clause}s, \isi{adposition}al phrases or (incipient) \isi{adverbial clause}s for particular discourse purposes.

Thus, while one cannot say that preposed \isi{adverbial clause}s have evolved to fill a functional gap within the linguistic system, it is still reasonable to conceive of them as functional \isi{adaptation}s to particular discourse environments, as preposed \isi{adverbial clause}s develop under the continuing influence of discourse considerations.

\textsc{initial} \textsc{and} \textsc{final} \textsc{subordinators}. Let us finally turn to the correlation between the position of \isi{adverbial clause}s and that of the \isi{subordinator}. Recall that while postposed \isi{adverbial clause}s are commonly introduced by a clause-initial \isi{conjunction}, preposed \isi{adverbial clause}s often occur with a final marker. In particular, in languages in which \isi{adverbial clause}s are generally preposed to the main clause, the \isi{subordinator} typically occurs at the end of the \isi{adverbial clause}. There are two general explanations for the position of adverbial \isi{subordinator}s in pre- and postposed \isi{subordinate clause}s: one refers to \isi{processing}, the other to \isi{grammaticalization}.


In Hawkins’ (\citeyear{Hawkins1994,Hawkins2004}) \isi{processing} approach, the positional patterns of adverbial \isi{subordinator}s are explained by two general principles. To simplify, one principle predicts that the \isi{subordinator} occurs at the boundary to the main clause because linear structures of this type have a short “recognition domain” that is easy to process\is{processing} and thus more highly preferred than structures with a long recognition domain. And the second principle predicts that there is a general tendency to place the \isi{subordinator} at the beginning of the \isi{subordinate clause} (regardless of clause order), because initial \isi{subordinator}s prevent the parser from misinterpreting \isi{subordinate clause}s as main clauses (see also \citealt{Diessel2005}: 455--459).

Hawkins’ theory provides a good fit to the data, but lacks a diachronic dimension. As it stands, it is completely unclear how the word orders that are explained by syntactic \isi{processing} in this approach have evolved in language history. \citetv{chapters/haspelmath} argues that functional explanations do not need diachronic evidence if they correctly predict the typological data; but I disagree with this view because functional explanations can turn out to be spurious when we consider how particular phenomena have evolved.

In fact, there is evidence that the above described correlation between the position of the \isi{subordinator} and the position of the \isi{subordinate clause} is just a by-product of \isi{grammaticalization} processes that are not immediately influenced by syntactic \isi{processing}. That \isi{grammaticalization} can have an impact on the linear organization of syntactic constituents has been observed in previous research (\citealt{Li1974_Chin}). In fact, a number of studies have argued that (some) word order correlations\is{word-order correlation} are due to \isi{persistence} effects in \isi{grammaticalization} (e.g. \citealt{Givón1975}, \citealt{Aristar1991,Bybee2010,Collins2012}; see also \citealtv{chapters/collins} and \citealtv{chapters/dryer}). 

\largerpage

For instance, according to \citet[111]{Bybee2010}, the correlation between the order of verb and object and that of verb and \isi{auxiliary} does not need a particular functional explanation, as auxiliaries are usually derived from the main verb of a complement construction that includes an infinitive, or some other type of verb, as verbal complement (e.g. `want' \textsc{infinitive}). If the verb precedes the verbal complement of a complex VP in the diachronic source, the \isi{auxiliary} precedes the main verb in the target construction; but if the verb follows the verbal complement in the diachronic source, the \isi{auxiliary} is postposed to the main verb in the target construction. As a consequence of these developments, the order of \isi{auxiliary} and verb correlates with that of verb and object \REF{ex:diessel:18}.

\ea\label{ex:diessel:18}
\begin{tabular}[t]{@{}llcll@{}}
{[\textsc{verb}} & {\textsc{[verb]\textsubscript{obj}]\textsubscript{vp}}} & \hspace{2cm} & {\textsc{[[verb]\textsubscript{obj}}} & {\textsc{verb]\textsubscript{vp}}} \\
\multicolumn{1}{c}{↓}  & ~~~↓ & & ~~~~~↓ & ~~~↓\\
{\textsc{[aux}}  & {\textsc{verb]\textsubscript{vp}}}                      &  & {\textsc{[verb}}  & {\textsc{aux]\textsubscript{vp}}}\\
\end{tabular}
\z

It is conceivable that the correlation between the position of adverbial \isi{subordinator}s and that of \isi{adverbial clause}s is also due to \isi{persistence} effects of \isi{grammaticalization}. For instance, above we have seen that \isi{purpose clause}s in \ili{Amele} and cause\is{causal clause}/\isi{purpose clause}s in \ili{Turkish} are marked by a clause-final \isi{subordinator} that also serves as a \isi{benefactive} \isi{adposition} in \isi{postposition}al phrases. Since \isi{postposition}al phrases usually precede all other constituents in \ili{Amele} and \ili{Turkish} (and most other head-final languages), it is a plausible hypothesis that the occurrence of final \isi{subordinator}s in these constructions is related to the fact that they are based on \isi{postposition}s (of preposed \isi{adposition}al constructions). 

\ea\label{ex:diessel:19}
\begin{tabular}[t]{@{}l@{ }ll@{}}
{[[ \textsc{np} ]}  &   {\textsc{p]\textsubscript{pp}}} & {[ \textsc{… v …]\textsubscript{s}}} \\
\multicolumn{1}{c}{↓}  & ↓ & \hspaceThis{[ \textsc{… }}↓\\
{[\textsc{[… s …}} & {\textsc{sub]\textsubscript{ac}}} &  {[ \textsc{… v …]\textsubscript{mc}}]}\\
\end{tabular}
\z

In other cases, final \isi{subordinator}s are based on \isi{quotative} verbs, as for instance, in some temporal\is{temporal clause} and \isi{causal clause}s of \ili{Aguaruna} and \ili{Lezgian} (\REF{ex:diessel:15}-\REF{ex:diessel:16}). Here again, the final position of the \isi{subordinator} is likely to be a consequence of \isi{grammaticalization}. Since \isi{quotative} clauses precede the quote verb in \ili{Aguaruna} and \ili{Lezgian} (and many other head-final languages), the final position of the \isi{subordinator} is readily explained by the fact that it evolved from a \isi{quotative} verb that followed the quote clause\is{quotative} in the source construction. 

\ea\label{ex:diessel:20}
\begin{tabular}[t]{@{}l@{ }ll@{}}
{[\textsc{[quote]}} &   {\textsc{v]}}  & {[ \textsc{… v …]\textsubscript{simple S}}}\\
\multicolumn{1}{c}{↓}  & ↓ & \hspaceThis{[ \textsc{… }}↓\\
{[[\textsc{… s …}} & {\textsc{sub]\textsubscript{ac}}} &  {[ \textsc{… v …]\textsubscript{mc}]\textsubscript{complex S}}}\\
\end{tabular}
\z

Crucially, while Hawkins’ \isi{processing} approach can also account for the main trends in the data, it cannot explain the \isi{exception}al cases. For instance, while postposed and flexible \isi{adverbial clause}s are usually marked by initial \isi{subordinator}s (as predicted by Hawkins), there are 26 postposed (and flexible) \isi{adverbial clause} constructions in the data in which the \isi{subordinator} comes at the end of the \isi{adverbial clause}, as in example \REF{ex:diessel:21} from \ili{Yagua}.

\ea\label{ex:diessel:21}
\langinfo{Yagua}{Peba-Yaguan}{\citealt{PaynePayne1990}: 340}\\
\gll   Deerá-miy  sąąniy-yąą  [sa-tįįysįa  \textbf{túunu}].\\
       child-\textsc{coll}  shout-\textsc{distrib}  3\textsc{sg}-play  while\\
\glt   `The children are shouting while they play.'
\z

\noindent While the existence of these structures flies in the face of Hawkins’ \isi{processing} account, it has a straightforward diachronic explanation. As \citet[340]{PaynePayne1990} point out, the subordinate \isi{conjunction} comes at the end of the \isi{adverbial clause} in \REF{ex:diessel:21} because \textit{túunu} ‘while’ has evolved from a \isi{postposition} meaning ‘side’, and since \isi{postposition}al phrases follow the verb in \ili{Yagua}, the resulting \isi{adverbial clause} includes a clause-final marker.\label{p:diessel:exception}

Considering these examples, we may hypothesize that \isi{grammaticalization} accounts for the occurrence of final \isi{subordinator}s in preposed \isi{adverbial clause}s. However, since adverbial \isi{subordinator}s are derived from a wide range of sources, it is unclear at this point if the \isi{grammaticalization} account is sufficient to explain the cross-linguistic data. Moreover, even if it turns out that the position of the \isi{subordinator} is primarily determined by \isi{grammaticalization}, this does not necessarily exclude the possibility that \isi{processing} also affects the position of the \isi{subordinator} as an independent factor. More research is needed to determine the role of \isi{grammaticalization} (and \isi{processing}) on the development of word order correlations\is{word-order correlation}, but I suspect that the cross-linguistic distribution of initial and final \isi{subordinator}s is primarily caused by \isi{grammaticalization} rather than by Hawkins’ principles of syntactic \isi{processing}.

\section{Conclusion}

To summarize the main points of this paper, we have seen that the position of \isi{adverbial clause}s correlates with the meaning of adverbial relations and the position of other grammatical categories that are similar to \isi{adverbial clause}s. Since preposed \isi{adverbial clause}s include a forward orientation that deviates from the dominant backwards orientation of clause combining in discourse, there is no obvious (diachronic) connection between preposed \isi{adverbial clause}s and independent sentences. Only conditional\is{conditional clause} and some \isi{temporal clause}s that precede the main clause are (often) based on juxtaposed\is{juxtaposition} sentences that are oriented towards the subsequent clause. All other semantic types of preposed \isi{adverbial clause}s develop from, or under the influence of, other (source) constructions: \isi{adposition}al phrases and \isi{nominalization}s, pre- and postnominal \isi{relative clause}s, internally headed relatives, and \isi{quotative} constructions.

The positional patterns of \isi{adverbial clause}s 
\label{p:diessel:preposedadverbialclauses}
can be explained by functional and \isi{cognitive} processes that influence both speakers’ choice of a particular clause order in language use and the diachronic developments of pre- and postposed \isi{adverbial clause}s from certain source constructions. Some of these processes affect the whole class of \isi{adverbial clause}s (e.g. the discourse-organizing function that motivates the occurrence of preposed \isi{adverbial clause}s), others are only relevant for certain semantic types of adverbial relations (e.g. \isi{iconicity} of sequence). Crucially, while the positional patterns of \isi{adverbial clause}s are motivated by functional and \isi{cognitive} aspects of language use, the position of the adverbial \isi{subordinator} may just be a by-product of \isi{grammaticalization}. Like the positional patterns of auxiliaries and other grammatical markers that evolved through \isi{grammaticalization}, the positional patterns of adverbial \isi{subordinator}s seem to be determined by the position of their diachronic sources. Since the various source constructions tend to occur in reverse orders in VO and OV languages, it is not improbable that the position of adverbial \isi{subordinator}s correlates with that of other grammatical categories in head-initial and head-final languages because of \isi{persistence} effects in \isi{grammaticalization}. However, more research is needed to investigate the \isi{cognitive} and diachronic mechanisms behind these correlations.

\section*{Abbreviations}

The paper abides by the Leipzig Glossing Rules. Additional or deviant abbreviations include:

\begin{multicols}{2}
\begin{tabbing}
\textsc{prtt}\hspace{5mm}\=non-actual (irrealis) \isi{mood}\hspace{5mm}\kill
\textsc{ac}  \>\isi{adverbial clause}                 \\
\textsc{asp} \> \isi{aspect}                          \\
\textsc{coll}   \>  collective                  \\
\textsc{di} \>  desiderative-intentional (\isi{mood}) \\
\textsc{distrib}\>  \isi{distributive}               \\
\textsc{ipd} \> impeditive                      \\
\textsc{mc}  \> main clause                      \\
\textsc{mid} \> middle \isi{voice}                    \\
\textsc{nact}\>  non-actual (irrealis) \isi{mood}     \\
\textsc{part}\>  particle                       \\
\textsc{pers}\>  personal                       \\
\textsc{plpf}\>  pluperfect                     \\
\textsc{prosp}\> prospective                    \\
\textsc{prtt}\>  \isi{partitive}                      \\
\textsc{ref} \> referential                     \\
\textsc{s}  \>    sentence/clause              \\
\textsc{sben}\>  self-benefactive               \\
\textsc{seq} \> sequential                      \\
\textsc{ss} \>  same \isi{subject}                   \\
\textsc{tc} \>  topic-contrast                  \\
\textsc{vent}\>  ventive                        
\end{tabbing}
\end{multicols}

\largerpage
\section*{Appendix: Language sample}
 
\textsc{Africa}: \ili{Fongbe}, \ili{Hausa}, \ili{Jamsay}, \ili{Kana}, \ili{Khwe}, \ili{Konso}, \ili{Koyra Chiini}, \ili{Krongo}, \il{Lango}Lan\-go, \ili{Mayogo}, \ili{Mbay}, \ili{Nkore Kiga}, \ili{Noon}, \ili{Supyire}, \ili{Tamasheq}. 
\\
\textsc{North and Central America}: \ili{Choctaw}, (Barbareño) Chumash\il{Chumash (Barbareño)}, \ili{Kiowa}, \ili{Lakota}, (Chalcatongo) Mixtec\il{Mixtec (Chalcatongo)}, \ili{Musqueam}, \ili{Ojibwe}, \ili{Purépecha}, \ili{Rama}, \ili{Slave}, \ili{Tepehua}, (Jamul) Tiipay\il{Tiipay (Jamul)}, \ili{Tümpisa Shoshone}, \ili{Tzutujil}, \ili{Wappo}, West Greenlandic\il{Greenlandic (West)}. 
\\
\textsc{South America:} \ili{Aguaruna}, \ili{Awa Pit}, \ili{Barasano}, \ili{Cavineña}, \ili{Epena Pedee}, \ili{Hup}, \ili{Jarawara}, \ili{Kwazá}, \ili{Mapudungun}, \ili{Matsés}, \ili{Mekens}, \ili{Mosetén}, \ili{Ndyuka}, (Huallaga) Quechua\il{Quechua (Huallaga)}, \ili{Tariana}, \ili{Trumai}, \ili{Urarina}, \ili{Warao}, \ili{Wariˈ}, \ili{Yagua}, \ili{Yuracaré}. 
\\
\textsc{Eurasia:} \ili{Abkhaz}, \ili{Ainu}, (Gulf) Arabic\il{Arabic (Gulf)}, \ili{Basque}, \ili{Evenki}, \ili{French}, \ili{Georgian}, \ili{German}, \ili{Hungarian}, \ili{Japanese}, \ili{Korean}, \ili{Lezgian}, \ili{Malayalam}, \ili{Marathi}, \ili{Persian}, \ili{Santali}, \ili{Serbo-Croatian}, \ili{Turkish}, (Kolyma) Yukaghir\il{Yukaghir (Kolyma)}. 
\\
\textsc{South-East Asia and Oceania:} \ili{Burmese}, \ili{Hmong Njua}, \ili{Begak Ida’an}, (Karo) Batak\il{Batak (Karo)}, \ili{Lao}, Mandarin Chinese\il{Chinese (Mandarin)}, Dolakha Newar\il{Newar (Dolakha)}, \ili{Qiang}, \ili{Semelai}, \ili{Taba}, \ili{Tetun}, \ili{Toqabaqita}, \ili{Tukang Besi}, \ili{Vietnamese}, \ili{Yakan}. 
\\
\textsc{Australia and New Guinea:} \ili{Gooniyandi}, \ili{Imonda}, \ili{Kayardild}, \ili{Kewa}, \ili{Korafe}, \ili{Lavukaleve}, \ili{Mali}, \ili{Mangarayi}, \il{Menya}Men\-ya, \ili{Motuna}, \ili{Martuthunira}, \ili{Ungarinjin}, \il{Wambaya}Wam\-ba\-ya, \ili{Yimas}
 
 
\largerpage  
\sloppy
\printbibliography[heading=subbibliography,notkeyword=this] 
\end{document}

\documentclass[output=paper]{langsci/langscibook} 
\author{Jeremy Collins\affiliation{Radboud University Nijmegen}}
\title{Some language universals are historical accidents}
\abstract{In this short paper, I elaborate on previous work by \citet{Givón1971} and \citet{Aristar1991} to argue that a substantial part of the well-known word order correlations is best explained by grammaticalisation processes. Functional-adaptive accounts in terms of processing or learning constraints are currently weakly substantiated, and they suffer from the fact that they do not adequately control for language-internal inheritance patterns. More generally, historical relatedness between different types of phrases constitutes an important confound in typological research, one that needs to be taken seriously before word order correlations are motivated by anything other than the diachronic patterns that link the word order pairs in question.}
% %% hyphenation points for line breaks
%% Normally, automatic hyphenation in LaTeX is very good
%% If a word is mis-hyphenated, add it to this file
%%
%% add information to TeX file before \begin{document} with:
%% %% hyphenation points for line breaks
%% Normally, automatic hyphenation in LaTeX is very good
%% If a word is mis-hyphenated, add it to this file
%%
%% add information to TeX file before \begin{document} with:
%% %% hyphenation points for line breaks
%% Normally, automatic hyphenation in LaTeX is very good
%% If a word is mis-hyphenated, add it to this file
%%
%% add information to TeX file before \begin{document} with:
%% \include{localhyphenation}
\hyphenation{
affri-ca-te
affri-ca-tes
com-ple-ments
Haw-kins
broad-er
over-view
par-tic-u-lar
spe-cif-ic
adap-tive
lin-guis-ti-cal-ly
Hix-kar-yana
sour-ces
mark-er
ground-ed
evol-ved
Born-kessel-Schlesew-sky
}
\hyphenation{
affri-ca-te
affri-ca-tes
com-ple-ments
Haw-kins
broad-er
over-view
par-tic-u-lar
spe-cif-ic
adap-tive
lin-guis-ti-cal-ly
Hix-kar-yana
sour-ces
mark-er
ground-ed
evol-ved
Born-kessel-Schlesew-sky
}
\hyphenation{
affri-ca-te
affri-ca-tes
com-ple-ments
Haw-kins
broad-er
over-view
par-tic-u-lar
spe-cif-ic
adap-tive
lin-guis-ti-cal-ly
Hix-kar-yana
sour-ces
mark-er
ground-ed
evol-ved
Born-kessel-Schlesew-sky
}
\begin{document}
\maketitle 
 
\label{p:collins:historicalaccidents} 
\section{Introduction}
 
% There are surprisingly few properties that all languages share. Almost every attempt at articulating a genuine language universal tends to have at least one \isi{exception}, as documented in \citet{EvansLevinson2009}. However, there are non-trivial properties that are found in, if not literally all languages, enough of them and across multiple language families and independent areas of the world, that they demand an explanation.  
 
There are surprisingly few properties that all languages share. Almost every attempt at articulating a genuine language universal tends to have at least one \isi{exception}, as documented in \citet{EvansLevinson2009}. However, there are non-trivial properties that are found in if not literally all languages, enough of them and across multiple language families and independent areas of the world, that they demand an explanation.  
 

An example is the fact that languages have predictable word orders. If a language has the verb before the object, it tends to have prepositions rather than postpositions, as in \ili{English}; if the verb is after the object, it is a good bet that the language will have postpositions rather than prepositions \citep{Greenberg1963}. The ordering of different elements such as a possessed noun and its possessor, or a noun and elaborate modifiers (complex adjective phrases, relative clauses), are to some extent free to vary among languages, but again tend to fall into correlating types (\citealt{Dryer1992,Dryer2011}). Why should knowing the \isi{word order} of one category in a language help predict the orderings of other categories? One prominent view holds that these patterns reflect an innate harmonic ordering principle of \isi{Universal Grammar}, which is ultimately argued to solve the logical problem of language \isi{acquisition} (\citealt{Pinker1994,Baker2001,Roberts2007}). This would amount to what \citetv{Haspelmath2019tv} calls a “representational constraint” on the shape of grammars. Another possible explanation is that \isi{word-order correlations} have evolved in the service of efficient language \isi{processing} (e.g. \citealt{Hawkins1994,KirbyHurford1997}), i.e. for functional-adaptive reasons. We find this view in the functional-typological literature (e.g. \citealt{Dryer1992,EvansLevinson2009}) as well as in computer simulations in the literature on language \isi{evolution} (\citealt{VanEverbroeck1999}). 

However, I would argue that many of these patterns are not evidence of our psychological preferences, but are accidental consequences of language history. More specifically, they are accidental in the sense that they arise as a by-product of grammaticalisation processes. These processes do not seem to have \isi{word order} correlations as a goal, nor is there good evidence for a “pull force” in that direction. Accordingly, grammaticalisation is an \textit{alternative} to functional motivations here, and an understanding of this historical dimension is thus crucial to explaining \isi{word order} correlations. In this short paper, I first elaborate this claim (\sectref{sec:collins:2}) based on an earlier publication \citep{Collins2012}, before I outline its consequences for typological theory and practice (\sectref{sec:collins:3}). In doing so, I am extending a line of argumentation by \citet{Givón1971} and \citet{Aristar1991}, but I relate the discussion specifically to the concerns of the present volume, and to Haspelmath’s position paper in particular.

\section{Word order correlations as a result of grammaticalisation}\label{sec:collins:2}

% Grammaticalisation is the process by which new grammatical categories can be formed from other (often lexical) categories. For example, Mandarin \ili{Chinese} has a class of words which might be called prepositions from a cross-linguistic point of view but which clearly have their historical roots in verbs. An example is 從 \textit{cóng}, which in modern Mandarin is a \isi{preposition} meaning ‘from’ but which in Classical \ili{Chinese} was a verb meaning ‘to follow’. It has lost its ability to be used as a full verb, requiring another verb such as ‘come’ in the sentence, just as \ili{English} requires a verb in the sentence \textit{I} \textit{come} \textit{from} \textit{London}. Other \ili{Chinese} prepositions such as 跟 \textit{gēn} ‘with’ also have a verbal origin, and many preposition-like words such as 給 \textit{gěi} ‘for’ and 在 \textit{zài} ‘in/at’ even retain verbal meanings (‘to give’ and ‘to be present’) and verbal syntax (such as being able to be used as the sole verb in the sentence and to take \isi{aspect} marking). These patterns of inheritance directly explain why the two types of constituents (i.e. PP and VP) have the same \isi{word order}: Prepositions and verbs were once the same category, and they simply have not changed their word orders since then. Since the verb precedes its NP object in classical and modern \ili{Chinese}, its prepositional offspring in modern \ili{Chinese} also precedes its NP complement. Interestingly, \ili{Chinese} also has postpositions, such as \textit{lǐ} ‘in’, and these, too, are simply continuations of their lexical sources (cf. also \citealtv{Dryer2019tv}). Thus \textit{lǐ} is etymologically ‘interior’ or ‘village’, hence \textit{fángzi} \textit{lǐ} ‘in the house’ might be glossed more literally as ‘the house’s inside’. Again, the order of the younger construction as noun (\textit{fángzi})–\isi{postposition} (\textit{lǐ}) reflects the order of the older construction with \isi{genitive} (\textit{fángzi})–noun (\textit{lǐ}). Very similar remarks apply to \ili{Niger-Congo} languages like \ili{Dagaare} in Ghana, which also shows typologically mixed adpositional phrases \citep{Bodomo1997}. 
Grammaticalisation is the process by which new grammatical categories can be formed from other (often lexical) categories. For example, Mandarin \ili{Chinese} has a class of words which might be called prepositions from a cross-linguistic point of view but which clearly have their historical roots in verbs. An example is \zh{從} \textit{cóng}, which in modern Mandarin is a \isi{preposition} meaning ‘from’ but which in Classical \ili{Chinese} was a verb meaning ‘to follow’. It has lost its ability to be used as a full verb, requiring another verb such as ‘come’ in the sentence, just as \ili{English} requires a verb in the sentence \textit{I} \textit{come} \textit{from} \textit{London}. Other \ili{Chinese} prepositions such as \zh{跟} \textit{gēn} ‘with’ also have a verbal origin, and many preposition-like words such as \zh{給} \textit{gěi} ‘for’ and \zh{在} \textit{zài} ‘in/at’ even retain verbal meanings (‘give’ and ‘to be present’) and verbal syntax (such as being able to be used as the sole verb in the sentence and to take \isi{aspect} marking). These patterns of inheritance directly explain why the two types of constituents (i.e. PP and VP) have the same \isi{word order}: Prepositions and verbs were once the same category, and they simply have not changed their word orders since then. Since the verb precedes its NP object in classical and modern \ili{Chinese}, its prepositional offspring in modern \ili{Chinese} also precedes its NP complement. Interestingly, \ili{Chinese} also has postpositions, such as \textit{li} ‘in’, and these, too, are simply continuations of their lexical sources (cf. also \citealtv{Dryer2019tv}). Thus \textit{li} is etymologically ‘interior’ or ‘village’, hence \textit{fangzi} \textit{li} ‘in the house’ might be glossed more literally as ‘the house’s inside’. Again, the ordering of the younger construction as noun (\textit{fangzi})–\isi{postposition} (\textit{li}) reflects the order of the older construction with \isi{genitive} (\textit{fangzi})–noun (\textit{li}). Very similar remarks apply to \ili{Niger-Congo} languages like \ili{Dagaare} in Ghana, which also shows typologically mixed adpositional phrases \citep{Bodomo1997}. 

More generally, the pattern of adpositions inheriting the ordering of the noun or verb they derive from is replicated in different language families: We find it in many \ili{Oceanic} languages (\citealt{LynchEtAl2002}: 51), where adpositions are transparently nouns and reflect whatever ordering of \isi{genitive}–noun the language has (hence it can be either prepositional, as in \ili{Hawaiian}, or postpositional, as in \ili{Motu}); we also see it in \ili{Indo-European} languages (e.g. \ili{English} \textit{across} < 13 ct. Anglo-\ili{French} \textit{an cros} ‘on cross’ (\citealt{BordetJamet2010}: 16)), in \ili{Japanese} (e.g. \textit{kara} ‘from’ < ‘way’, \textit{si} restrictive particle < ‘do’ (\citealt{Frellesvig2010}: 132–135)), in \ili{Australian} languages in which adpositions are morphologically still nouns \citep{Dixon2002}, in \ili{Tibetan} and \ili{Burmese} (\citealt{DeLancey1997}), and so on. \citet[62]{HeineKuteva2007} even remark that “we are not aware of any language that has not undergone such a process”.

Grammaticalisation can also often explain the ordering of verb and object correlating with \isi{genitive} and noun ordering \citep{Dryer2011}. Certain types of verb phrase derive historically from noun phrases made up of a nominalised verb and its \isi{patient} argument in a \isi{possessive construction}. An example is \ili{Ewe}:

\ea

\langinfo{Ewe}{Atlantic-Congo, Gbe}{\citealt{Claudi1994}: 220}\\
\gll Me-le       é-kpɔ     dzí.\\
     \textsc{1sg}.-be.at  3\textsc{sg.poss/obj-}see   surface/on\\
\glt ‘I am seeing him.’ (lit. ‘I am on his seeing.’).
\z

\ili{Ewe} is normally SVO but employs the \isi{genitive}–noun ordering here (‘his seeing’), creating a construction which is SOV. Nominalisations of this kind are used cross-linguistically for expressing \isi{aspect} (such as the continuous \isi{aspect} in \ili{Ewe}), for subordinate clauses (expressing ‘I was surprised that he saw me’ as ‘I was surprised at his seeing of me’ in \ili{Javanese}, cf. \citealt{Ogloblin2005}: 618) and for \isi{voice} marking (in \ili{Austronesian} languages, cf. \citealt{Himmelmann2005}: 174). These verb phrases can become the most frequently used and unmarked verb phrases in the languages, thus the basic verb–object order of a language can evolve from a \isi{genitive}–noun construction, even if the nominal origins of the verb form are no longer transparent. 

This development of (main-clause) verb phrases from nominalised verbs with a possessor object is again attested in very different language families, although it is more complicated to reconstruct. A typical example is the \isi{evolution} of VOS ordering in Proto-\ili{Austronesian}, which has been inherited by over a thousand \ili{Austronesian} languages or evolved further into SVO or VSO \citep[7]{Adelaar2005}. It is now generally accepted that verb phrases in \ili{Austronesian} languages evolved from nominalising verbs, with a sentence such as ‘The children are looking for the house’ deriving from a Proto-\ili{Austronesian} construction of the type ‘The children are the searchers of the house’. \citet{StarostaEtAl1982} as well as \citet{Kaufman2009} present several pieces of evidence in favour of this diachronic hypothesis: For example, the \isi{voice} markers on verbs derive from nominalising morphemes, cognates of which still exist in \ili{Tagalog} and other languages, such as the \isi{locative} \isi{voice} marker \textit{an} which is also used for deriving place names (\textit{aklat-an} ‘library’ < \textit{aklat} ‘book’). Moreover, the \isi{direct object} of the verb is \isi{marked} with the \isi{genitive} marker \textit{ng} or put into the \isi{genitive} \isi{case} if a \isi{pronoun}. Both nominalisation and the use of \isi{equational} sentences of the form AB ‘A is B’ are extremely common in conservative \ili{Austronesian} languages and presumably were in Proto-\ili{Austronesian}, allowing this frequently used construction to become a standard form of predication. Thus the verb–object ordering in \ili{Austronesian} languages derives simply from the noun–\isi{genitive} ordering of Proto-\ili{Austronesian}, which is still retained in these languages. At a stroke this \isi{word order} correlation is accounted for in roughly a sixth of the world’s languages.

As \citet[167]{Sasse2009} notes in a comment on \citet{Kaufman2009}, the situation in \ili{Austronesian} is “not as ‘exotic’ as it seemed to be at first sight, especially not for a Semiticist or an Afroasiaticist”. He notes that the \ili{Cushitic} languages also replaced their finite verb forms with participles and are used with \isi{dative} marking on the \isi{agent}, in effect saying ‘I have heard it’ as ‘To me was hearing’ \citep[174]{Sasse2009}; and that the \isi{dative} pronouns eventually grammaticalised further to finite verbal morphology. This change also took place in the \ili{Iranian} and \ili{Indo-Aryan} languages, stretching over a large linguistic area. 

Sasse also notes independent developments of agents \isi{marked} with \isi{genitive} \isi{case} in \ili{Mayan} and \ili{Inuit} languages, and \citet{Gildea1997} made a similar \isi{reconstruction} for the \ili{Cariban} language family, of which the famous OVS language \ili{Hixkaryana} is an example: It has \isi{genitive} marking on the object, effectively expressing ‘the enemy will destroy the city’ as ‘it will be the city’s destruction by the enemy’ \citep[153]{Gildea1997}, explaining among other things why the \isi{subject} is placed last, and why it has \isi{ergative} marking. One can add to this list many languages in Asia, as described in \citet{YapEtAl2011}, such as \ili{Tibeto-Burman} languages that often use nominalised forms in main clauses (e.g. ‘goat-killing exists’ for ‘he is killing a goat’, cf. \citealt{DeLancey2011}: 349), and even \ili{Japanese}, in which argument markers such as \textit{ga} were originally \isi{genitive} markers \citep[461]{Shinzato2011}. Examples of \ili{Niger-Congo} languages such as \ili{Ewe} were given earlier and are discussed by \citet{Claudi1994}, while Heine describes how many \ili{Nilo-Saharan} and \ili{Chadic} languages render desiderative sentences in the following way:

\ea\label{ex:key:} 
\langinfo{Angas}{\ili{Afro-Asiatic}, Chadic}{\citealt{Heine2009}: 31}\\
\gll Musa   rot   dyip   kə-shwe.\\
     Musa   want   harvest   \textsc{poss-}corn\\
\glt ‘Musa wants to harvest corn.’ (lit. ‘Musa wants the harvesting of the corn.’)
\z

The historical data thus show that these processes of grammatical change are not limited to individual languages or families but can instead be found much more widely, and independently of one another. They lead us to predict, then, that ultimately all correlations between the ordering of elements in verb phrases (V–NP), adpositional phrases (P–NP) and possessive noun phrases (GEN–NP) are due to direct historical connections between pairs of phrases (cf. also \citealt{Croft2003}: 77–78 for more discussion of such pairs). In the next section, I consider the implications of this assumption for both explanation and methodology in linguistic typology.

\section{Consequences for typology}\label{sec:collins:3}

As historical evidence for the grammaticalisation account is accumulating, one may ask whether this makes alternative, functional-adaptive explanations invalid. Recall from above that on non-nativist approaches, \isi{word order} correlations are often argued to make sentences easier or more efficient to parse in real time, as compared to sentences with mixed head–dependent ordering patterns (e.g. \citealt{Hawkins2004}). Is it possible that these factors play a role alongside grammaticalisation, such that, for example, \isi{processing} demands filter out certain difficult-to-process constructions, as \citet{KirbyHurford1997} suggest (cf. also \citealt{Christiansen2000})? Put somewhat differently, could it not be the \isi{case} that grammaticalisation happens to produce orderings that are easy (or easier) to parse?

There is currently not much evidence to substantiate this view. From a theoretical perspective, there is no indication that the processes involved in grammaticalisation are instigated by considerations of efficient \isi{parsing} or learning. They happen through pragmatic \isi{inference} in specific communicative contexts (\citealt{HopperTraugott2003}: Ch. 4), through widespread metaphorical mappings (cf. \citealt{Deutscher2005}: Ch. 4) and by means of \isi{chunking} of repeated sequences \citep{Bybee2002}. Through these mechanisms, a new construction begins to emerge that gradually emancipates from its original lexical source. Since it is gradual, this process often creates a chain of intermediate cases, such as denominal adpositions in \ili{Tibetan}, some of which still require \isi{genitive} marking (e.g. \textit{mdun} ‘front’) while others have shed this marking (e.g. \textit{nang} ‘inside’; cf. \citealt{DeLancey1997}: 58–59). In other words, grammaticalisation has its origin in common non-linguistic processes (cf. also \citealt{Bybee2010}: 6–8) and has predictable consequences, such as the gradual and sometimes only partial elimination of the morphology associated with the source. Importantly, a hallmark of grammaticalisation is syntagmatic “freezing” (\citealt{Croft2000}: 159; cf. also \citealt{Lehmann2015}: 168), so that the order of the elements in the new construction mirrors the order of elements in the source. The result is a “correlation” between the syntagmatic structure of the old and the new construction, but one that effectively rests on inertia rather than overarching \isi{processing} principles that work towards a correlation.

From a methodological perspective, \isi{processing} and learning accounts are an example of a broader trend of the “ad hoc search for functions that match the universals to be explained”, as \citet[13]{Kirby1999} puts it. Attempts in the evolutionary literature to simulate \isi{processing} or learning with computers in order to derive Greenberg’s \isi{word order} universals (e.g. \citealt{VanEverbroeck1999,KirbyChristiansen2003}), have a particularly “just-so” flavour: All that computer simulations can do is show that \isi{processing} or learning preferences of individuals can cause these correlations to emerge over time, all other historical factors being equal, not that they are actually responsible. What we would thus need is independent historical evidence that \isi{processing} concerns do, in fact, guide historical change. There are some attempts to show this, for example, in earlier \ili{English} (e.g. \citealt{Fischer1992,ClarkEtAl2008}), when the language appeared to converge on the \isi{word order} correlations after a period of freer \isi{word order}. This could indeed be evidence for \isi{word order} correlations emerging at least in part out of \isi{processing} considerations; but there are other possibilities in this \isi{case} which need to be investigated further, such as it being related to the rise of analytic verb forms and periphrastic \textit{do}, to the loss of inflections or as a result of \isi{contact} from \ili{French} (cf. also  \citealt{FischervanderWurff2006}: 187–188 for some of the controversies). The historical role of \isi{processing} is unclear even in this \isi{case}, and there is no conclusive cross-linguistic evidence for it either.

One possibility for establishing such causal relations cross-linguistically would be to look for cases of correlated \isi{evolution}, i.e. situations in which a change in one \isi{word order} can be shown to be followed by a change in another \isi{word order} in the history of a language, or in its descendants. For example, if a language has verb–object order and prepositions but then changes to having object–verb order and postpositions, then this suggests that the two word orders are functionally linked (if this event takes place after any grammaticalisation linking these verbs and postpositions). The only solid statistical test of this so far has been a widely discussed study by \citet{DunnEtAl2011}. Dunn and colleagues examined the ways in which four language families have developed (\ili{Bantu}, \ili{Austronesian}, \ili{Indo-European} and \ili{Uto-Aztecan}) and tested models of \isi{word order} change using a Bayesian phylogenetic method for analysing correlated \isi{evolution}. They found that some word orders do indeed change together: For example, the order of verb and object seems to change simultaneously with the order of adposition and noun in \ili{Indo-European}. A model in which these two word orders are dependent is preferred over a model in which they are independent with a \isi{Bayes factor} of above 5, a conventional threshold for significance. This seems to vindicate the idea that adpositions and verb-object order are functionally linked in \ili{Indo-European}, and the pattern also holds up in \ili{Austronesian}. It does not show up in the smaller and younger families \ili{Uto-Aztecan} and \ili{Bantu}, although that may be because of the low statistical power of this test when applied to small language families (cf. \citealt{CroftEtAl2011}). But a more important drawback is that there is no control for language \isi{contact}. What could be happening is that some \ili{Indo-European} languages in India have different word orders because of the languages that they are near, such as \ili{Dravidian} languages, which also have object-verb order and postpositions.  A similar point could be made about the \ili{Austronesian} languages that undergo \isi{word order} change, which are found in a single group of Western \ili{Oceanic} languages on the coast of New Guinea, which is otherwise dominated by languages with object-verb order and postpositions.

In the context of the present discussion, an important result of \citegen{DunnEtAl2011} paper is that word orders are very stable, staying the same over tens of thousands of years of evolutionary time (i.e. the total amount of time over multiple branches of the families). In this light, it is also instructive to note that some typologically “mixed” or non-correlating languages show the same inert behaviour: Despite the fact that grammaticalisation has produced a mixture of prepositions and postpositions (e.g. in \ili{Chinese} or \ili{Dagaare}), the resulting systems have also survived for many generations, or even thousands of years, without showing any inclination to change. This, too, is a problem for processing-based theories, which sometimes explicitly predict that such inconsistencies should die out (e.g. \citealt{KirbyHurford1997}).

In the absence of convincing evidence for functional-adaptive motivations, I suggest that we accept that different types of syntactic constituents share their ordering patterns because they are historically related to each other, i.e. because they are linked by common ancestry. This also has important methodological consequences for typology. The kind of historical relatedness we observe here \label{pg:collins:dependency}qualifies as a subtle, language-internal variant of Galton’s problem (cf. \citealt{Cysouw2011} for an introduction), and it is thus actually a \textit{confound} in typological samples. Just as other, more widely known, types of historical relatedness, such as a genealogical or areal interaction between two data points in a sample, need to be controlled for before one can test for a typological correlation, so does the language-internal historical relatedness between the grammatical patterns that make up that correlation. Put differently, languages in which possessor arguments are known to have developed from former object arguments and have simply adopted their order from this source, do not constitute an independent data point in support of the alleged \isi{word order} correlation. For typological practice, this entails that we need large databases
\label{pg:collins:refforhaspelmath}of attested grammaticalisation pathways, and that we need to examine more carefully the actual markers and their (likely) etymologies before we set out to test a functional-adaptive hypothesis. In principle, it would then be possible to inspect whether certain grammaticalisation pathways tend to be taken only in certain types of languages; for example, do postpositions only develop from nouns in a \isi{genitive} construction (‘table's head’ > ‘table on’) if the language also places the verb after the object? It is easy enough to find exceptions to that, such as \ili{Dagaare} (Atlantic-Congo), which has taken this route to postpositions despite being a VO language \citep{Bodomo1997}. But in a large database, we might still find interesting structural constraints, as well as geographical patterns, that could potentially speak for or against functional-adaptive motivations in addition to grammaticalisation. 

For now, the major point is that the historical non-independence of data points can create correlations that are not causal. Such spurious correlations are well-known from non-linguistic research (cf., e.g., the spurious correlation between chocolate consumption and Nobel Prize winners; cf. also \citealt{RobertsWinters2013} for further discussion), and my claim in this paper is that this is a serious methodological pitfall in the domain of \isi{word order} correlations. Given the \isi{naturalness} of grammaticalisation, and the above observation that word orders tend to be preserved and long retained after grammaticalisation, invoking functional-adaptive motivations to explain the correlations in \isi{question} is not only redundant, but actually wrong-headed. It is as if one wanted to claim that there was a deeper ecological reason why chimpanzees and humans share 98.8\% or so of their DNA, rather than just the primary historical reason, which is that they have a common ancestor.

Having said this, it should be pointed out that I am neither arguing against functional-adaptive explanations in \isi{general}, nor am I denying the relevance of \isi{processing} to understanding \isi{word order} patterns as such, including some combinations of \isi{word order} that tend to be preferred over others. For example, the fact that VO languages strongly tend to have postnominal relative clauses is plausibly related to \isi{processing} constraints \citep{Hawkins2004}. Similarly, correlations between \isi{numeral}–noun and adjective–noun ordering do not have a clear explanation in terms of grammaticalisation, but they do seem to be functionally linked and hence show interesting dependencies in experiments in \isi{artificial language} learning (e.g. \citealt{CulbertsonEtAl2012}; cf. also \citealtv{Dryer2019tv}). But with more and more diachronic evidence coming to light, historical links between many grammatical categories (VPs, auxiliaries, genitives, adpositions) can no longer be dismissed as marginal and as “lack[ing] generality” \citep[131]{Hawkins1983}. Our default assumption, then, should be that the core \isi{word order} correlations are first and foremost an accidental by-product of grammaticalisation. 

\citetv{Haspelmath2019tv} actually acknowledges this type of explanation, at least for the ordering patterns of adpositional phrases, and labels it a “\isi{mutational constraint}” – a situation in which historical sources and grammaticalisation pathways directly determine the synchronic outcomes and hence make functional-adaptive explanations superfluous. On the other hand, he rejects “common pathways” as too weak to have explanatory power in typology. But how common is “common”, and when do we begin to speak of a \isi{mutational constraint}? It is perfectly possible that common pathways (such as those documented in \citealt{HeineKuteva2002}; 2007), while not exhausting the possible sources and routes, are still frequent enough to produce a principled synchronic result. Therefore, I disagree with Haspelmath (p. \pageref{Haspelmathchapterpageref}) that we need not be able to understand the diachronic patterns behind a universal tendency if there is a good functional-adaptive motivation available for it. In the \isi{case} of \isi{word order} correlations, and possibly other domains of grammar, it is the other way around: We first need to understand the diachronic links between different types of phrases and then control for them when we attempt to establish whether there are universal correlations beyond historical dependencies at all. It may turn out that the real \isi{question} is why it should ever be the \isi{case} that the order of grammaticalised categories, such as adpositions, genitives or auxiliaries does \textit{not} correlate with that of their source constructions.

\section{Conclusion}

Word order correlations are often invoked as evidence for universals of language \isi{acquisition} or language \isi{processing}. In this paper, I have argued that, before we can do so, it is important to understand the historical background of these patterns, which standard interpretations do not take into account. Given the \isi{naturalness} and the non-teleological nature of grammaticalisation processes, it should be our default assumption that the order of grammaticalised categories retains the order of their respective source constructions. From this perspective, \isi{word order} correlations are far from mysterious and, in many cases, do not require functional-adaptive motivations (such as specific \isi{processing} principles) or innate constraints (such as a head-ordering parameter). Instead, the correlations arise during the creation of new constructions by extending old constructions. The grammaticalisation processes involved are well-understood and ubiquitous (cf. \citealt{Bybee2015}). And although we will never be able to have a full picture of the possible routes that lead to adpositions, auxiliaries, genitives, etc., the ones we know of seem common enough to produce the correlations in \isi{question}. At the very least, they constitute language-internal dependencies, in Galton’s spirit, that need to be controlled for in any typological investigation of \isi{word order} correlations, in addition to areal dependencies that hold across languages. If they are not, one runs the risk of erroneously inferring causation from correlation, as the \isi{word order} correlations would appear so strong that they require a deeper explanation, when in fact they are largely dependencies built into the sample.

\section*{Acknowledgements}

I would like to thank the editors of the present volume, and Karsten Schmidtke-Bode in particular, for detailed discussion and extensive editorial help in compiling this paper, which is based largely on an earlier publication \citep{Collins2012} and a more recent blog post \citep{Collins2016}.

\sloppy
\printbibliography[heading=subbibliography,notkeyword=this] 


\end{document}
\documentclass[output=paper]{langsci/langscibook} 
\author{Ilja A. Seržant\affiliation{Leipzig University}}
\title{Weak universal forces: The discriminatory function of case in differential object marking systems}
\shorttitlerunninghead{Weak universal forces: The discriminatory function of case in DOM systems}
\abstract{Standard typological methods are designed to test hypotheses on strong universals that broadly override all other competing universal and language-specific forces. In this paper I argue that there are also weak universal forces. Weak universal forces systematically operate in the course of development but then interact with, or are even subsequently overridden by, other processes such as analogical extension, persistence effects from the source function, etc. This, in turn, means that there can be statistically significant evidence for violations at the synchronic level and, accordingly, only a weak positive statistical signal. But crucially, the absence of statistical prima-facie evidence for such forces does not amount to evidence for their absence. The assumption that there are also weak universal forces that affect language evolution goes in line with the view that human cognition in general and language acquisition in particular are constrained by probabilistic biases of different range, including weak ones (cf. \citealt{ThompsonEtAl2016}). By way of example, the present paper claims that the discriminatory function of case in differential object marking (DOM) systems is a weak universal: It keeps appearing in historically, synchronically and typologically very divergent constellations but is often overridden by other processes in further developments and is, therefore, not significant at the synchronic level in a large sample.}
\begin{document}
\maketitle 
 

\section{Introduction}

In this paper, I adopt a dynamic approach to universals \citep{Greenberg1978_Diachr} and, accordingly, the following definition of a universal:

\ea\label{ex:serzant:1}
A dynamic definition of universals\\
principled preferences that affect how languages change over time \citep[401]{Bickel2011_Modelling}.
\z

I conceive of these preferences as statistical tendencies (cf. \citealt{Bickel2011_Modelling}) rather than “inviolable constraints” on language in \citet{Kiparsky2008}. This definition singles out those universals that are not predetermined by the historical origin of the structures in question, thus resembling Haspelmath’s “functional-adaptive constraints” on language (\citealtv{Haspelmath2019tv}). Universal forces of this kind produce structures that occur with “overwhelmingly greater than chance frequency” or “well more than chance frequency” (\citealt{Greenberg1963}: 45, 47, \textit{passim}), and they thus allow for exceptions. The number of such exceptions, in turn, is indicative of the \textit{strength} of a universal force. 

Strong universal forces reveal themselves as universal on both of the methodological approaches used in typology: on the \textit{static} and on the \textit{dynamic} approach (see \citealt{Greenberg1969} for these notions). The former crucially relies on the relative frequency in the synchronic distribution across languages, while the latter is based on the relative frequency of the relevant changes across languages from a proto-stage (\textsc{stage} 0) into the synchronic stage (\textsc{stage} 1).\footnote{Note that the static approach, too, assumes that the synchronic distributions are the result of diachronic changes that have led to them (cf. \citealtv{Haspelmath2019tv}). It is, therefore, only methodologically but not ideologically synchronic.} A typical characteristic of strong universals is that the dynamic and the static evidence for these universals converge. For example, the force that \textsc{all languages must have vowels} \citep[19]{Comrie1989} finds solid evidence for universality on the static approach, in the sense that one would hardly find a spoken language violating this universal, i.e. a language without any vowels. The dynamic approach will equally show that, despite various language-specific processes such as vowel reduction strategies and even vowel loss, these never succeed to such an extent as to yield a language without any vowel, because no other universal or language-specific force may override this universal force in any type of language change.  

Another strong universal – albeit somewhat weaker than the former – concerns inflection: \textsc{if there is any inflection in nouns, there is also some inflection in pronouns} (\citealt{Moravcsik1993,PlankEtAl2002f}f.). A still weaker universal – a number of exceptions can be found in the literature (cf. \citealt{Handschuh2014}) – concerns case marking: \textsc{in a language with case, the zero-marked case tends to be the one that marks the subject of intransitive verbs} \citep[95]{Greenberg1963}. 

Thus, there is gradience in the strengths of universals and, accordingly, in the number of exceptions found at \textsc{stage 1} with each universal. By entertaining the idea of gradience a bit further, one may also think of a force that systematically operates in the development of a particular category across languages, i.e. in the transition between \textsc{stage} 0 and \textsc{stage} 1, and is, therefore, a universal according to the definition in \REF{ex:serzant:1} above. However, this universal is not strong enough to override competing internal and/or universal forces to remain visible at \textsc{stage} 1. A universal of this kind is referred to as \textit{weak} \textit{universal} \textit{force}\textsc{:} 

\ea\label{ex:serzant:2}
Definition of a weak universal force\\
A weak universal is a force that systematically exercises an impact in the historical development from \textsc{stage 0} into \textsc{stage 1} in a particular (grammatical) domain; this impact is found across geographic areas and genealogical affiliations in the diachrony with significant frequency, but may be marginal and heavily restricted or not be visible at all in the synchronic layer (\textsc{stage 1}).
\z

The synchronic effects of a weak universal force often reside in marginal subdomains or are overridden altogether by some other, stronger processes (cf. \citealt{Bickel2014}: 117). This, in turn, means that there will be a significant number of violations and only a weak positive statistical signal (if at all). As a result, the standard methodologies that rely on the relative frequency in the \textit{prima-facie} data will provide disproof of universality. 

To give an example, \citet{Hammarström2015} argues on the basis of 5,230 languages that there is a universal trend for SVO word order across languages (cf. \citealt{Gell-MannRuhlen2011,MauritsGriffiths2014}), henceforth, the \textsc{SVO universal}. Having said this, he claims that “the universal is not the only, nor the most important factor” constraining the synchronic distribution; the most important factor responsible for the current distribution is the order of the immediate ancestor, i.e. inheritance. The following figures illustrate this point: SOV is much more widespread than SVO across language families, with 65.1\% SOV vs. 16.2\% SVO,\footnote{SOV (43.3\%) is attested only slightly more frequently than SVO (40.2\%) if the genealogical bias is not controlled for (cf. \citealt{Dryer2013_SOV}). This effect is just due to a few large families with SVO \citep{Hammarström2015}.} but a change from SOV to SVO and from VSO to SVO is significantly more probable than the respective reverse changes (\citealt{Croft2003}: 234; \citealt{MauritsGriffiths2014}). \citet{Hammarström2015} shows that the pressure to retain the inherited word order accounts for 78\% of the sample, while the universal SVO accounts for only 14\% of the static evidence. The SVO \textsc{universal} is thus a weak universal in the sense that it cannot so easily force a language to change into SVO against the pressure of inheritance.

In what follows, I argue that the \textit{discriminatory} \textit{function} of flagging is a weak universal despite apparent counterevidence. I illustrate this with qualitative data and arguments about how different motivations may lead to a result that is easily misinterpreted if taken at face value. In order to do so, I first introduce the \textit{(global)} \textit{discriminatory} \textit{function} and the related phenomenon of \textit{local} \textit{disambiguation} (\sectref{sec:serzant:2}). \sectref{sec:serzant:3} exemplifies various differential object marking (DOM) systems and how the discriminatory function interacts with other, stronger forces in each of them. Finally, \sectref{sec:serzant:4} provides a discussion of the phenomenon of weak universals and conclusions.

\section{The (global) discriminatory function}\label{sec:serzant:2}

Since a transitive clause has two arguments (A and P), it must be ensured that the hearer will be able to discern which of the arguments should be interpreted as A and P, respectively. Moreover, other potential misinterpretations, such as one NP modifying the other NP – if both are adjacent to each other – or both NPs being coordinated (without a conjunction), should be excluded. There are many ways in which the discriminatory function may be implemented in a particular language or even in a particular sentence, with flagging being one of them: 

\ea\label{ex:serzant:3}
Definition of the global discriminatory function of P flagging (economy subsumed)\\
In a transitive clause, the A and the P argument must be sufficiently disambiguated, e.g. by word order, agreement, voice, world knowledge, and it is only if they are not that there is dedicated P flagging.\\
\z

A number of researchers have argued that there is only little or no evidence for (A or P) flagging systems being driven by the discriminatory function as defined in \REF{ex:serzant:3} cross-linguistically (\textit{inter} \textit{alia}, \citealt{Aissen2003,Malchukov2008}; various papers in \citealt{deHoopdeSwart2009}). Levshina (in prep.) shows on the basis of the large-scale AUTOTYP database that there is no statistically significant effect of the discriminatory function observable for flagging because there are only very few languages in which flagging is primarily driven by the discriminatory function. Sometimes even in these languages, the discriminatory function does not serve the purpose of discrimination between A and P alone: a function inherited from the source construction and often some ongoing conventionalization of the most frequent discrimination patterns override the discriminatory function to various extents. 

Having said this, it has been repeatedly suggested that flagging might also serve the discriminatory function, especially if A and P have similarly ranked input (cf., \textit{inter} \textit{alia}, \citealt{Comrie1978,Comrie1989}; \citealt{Dixon1994,Silverstein1976,Kibrik1997}). \citet[117]{Bossong1985} even assumed that the emergence of DOM is primarily due to the discriminatory function. In the following section, I follow this line of thinking and provide qualitative evidence for the claim that the discriminatory function does operate across genealogically and areally diverse DOM systems and is therefore a universal according the definition given in \REF{ex:serzant:1}. However, it is not a typical universal in that its impact is mostly weakened by other competing processes to which it is subordinate, the effect being that there is only marginal evidence for it at the synchronic \textsc{stage} 1.

\section{Evidence from DOM systems}\label{sec:serzant:3}

Consider the DOM system of the rural variety of Donno Sɔ, as described in \citet{Culy1995}. The DOM suffix \nobreakdash-\textit{ñ} marks human and often animal-denoting pronouns and nouns if the latter are definite:

\ea\label{ex:serzant:4}
\langinfo{Donno Sɔ }{Dogon: Mali}{\citealt{Culy1995}: 48}\\
\gll   Anta-ñ     ibɛra     yaw     aa      bem.\\
     Anta-\textsc{dom}  market.\textsc{loc}  yesterday  see.\textsc{ptcp} \textsc{aux.1sg}\\
\glt ‘I saw Anta at the market yesterday.’
\z

\ea\label{ex:serzant:5}
\langinfo{Donno Sɔ }{Dogon: Mali}{\citealt{Culy1995}: 48}\\
\gll  Jalɔmbe   izɔmbe-ñ     keraa     biyaa.\\
     donkey.\textsc{def.pl} dog.\textsc{def.pl-dom}   bite.\textsc{ptcp}   \textsc{aux.3pl}\\
\glt ‘The donkeys bit the dogs.’
\z

In contrast, neither indefinite animates nor inanimate definites are marked. We observe that at least two referential scales are simultaneously operating here:\footnote{In this paper I do not make any assumptions about the nature of referential scale effects: whether they stem from generalizing the most frequent patterns conditioned by the discriminatory function (cf. \citealt{Aissen2003}) or from the source (i.e. from topics, cf. \citealt{DalrympleNikolaeva2011}), or are language-specific \citep{BickelEtAl2015}, or whether they represent an independent phenomenon \textit{sui} \textit{generis}, is irrelevant here.}

\ea\label{ex:serzant:6}
Animacy scale\\
human > animate > inanimate
\z

\ea\label{ex:serzant:7}
Definiteness scale\\
definite > specific > indefinite
\z

Superficially, the discriminatory function does not seem to apply in this language since scale effects from \REF{ex:serzant:6} and \REF{ex:serzant:7} predominate: all animate and definite NPs are marked regardless of whether they really need to be globally disambiguated or not. However, the animate indefinite NPs that are not (yet) affected by the scale effects do show the operation of the discriminatory function. With these NP types, the DOM marker may be employed to discriminate between A and P in a particular utterance (cf. the Disambiguation Principle in \citealt{Culy1995}: 52). For example, when both the object and the subject NP are indefinite and animate and there are no other clues how to discriminate between A and P, the DOM marker may be employed “against” the force of marking definite animates only:

\ea\label{ex:serzant:8}
\langinfo{Donno Sɔ }{Dogon: Mali}{\citealt{Culy1995}: 53}\\
\gll Wɛzɛwɛzɛginɛ   yaana     po-ñ     don   wo mɔ ni tɛmbɛ.\\
     crazy.person  woman    large\textsc{{}-dom}   place  \textsc{3sg} \textsc{ps}  at found\\
\glt ‘A crazy person found a large woman at his/her place.’
\z

In this example, both indefinite NPs ‘a crazy person’ and ‘a large woman’ may potentially be interpreted as A \citep[53]{Culy1995}. Therefore, the DOM marker -\textit{ñ} is used here to unequivocally mark the syntactic role of ‘a large woman’. The discriminatory function is the weakest among other forces here \citep[53]{Culy1995} because it applies in a way exceptionally by constraining only one slot on the referential scales in \REF{ex:serzant:6} and \REF{ex:serzant:7}: the indefinite animate P. In accordance with \citet[51]{Culy1995}, one can thus posit the following forces and their relative weight (from the strongest to the weakest): 

\ea\label{ex:serzant:9}
The relative weight of the main forces on DOM in Donno Sɔ (and Malayalam, see below)\\
animacy scale + definiteness scale > discriminatory function\\
\z

Another important observation can be made here. Notice that the slot on the referential scales in \REF{ex:serzant:6} and \REF{ex:serzant:7} that is open for the application of the discriminatory function is immediately next to the slots that require rigid marking. I interpret this in the following way. In their historical developments, many DOM systems extend the DOM markers gradually from left to right on referential scales such as \REF{ex:serzant:6} and \REF{ex:serzant:7} (cf. \citealt{DalrympleNikolaeva2011}). For example, many languages start with a DOM system that applies only to animate nouns but then gradually extend the DOM marker onto inanimate nouns as well. Note that very often the difference in meanings between the two neighbouring slots on a referential scale is quite substantial and is certainly not graspable in terms of semantic extension. For example, the expansion of the DOM marker \textit{{}-rā(y)} from mostly animates in Middle Persian (\citealt{Key2008}: 244; cf. also \citealt{Paul2008}: 152–153) to the inclusion of inanimates in Modern Persian is not semantically straightforward, since the two are rather antonymic in meaning. I suggest that it is precisely the discriminatory function that is responsible for the expansion of the DOM marker into the next slot on the scale because the discriminatory function is not dependent on the lexical meaning of the noun in the same way as, for example, the animacy scale. The discriminatory function then applies to the next slot until that slot also becomes conventionalized, and so on. 

A constellation very similar to Donno Sɔ is found in Malayalam (Dravidian). The Accusative marker \textit{{}-(y)e} is regularly used with animate specific object referents but is normally ungrammatical with inanimate referents: 

\ea\label{ex:serzant:10}
\langinfo{Malayalam }{Dravidian: India}{\citealt{AsherKumari1997}: 204}\\
\gll Tiiyyǝ       kuʈil   naʃippacu.\\
     fire.\textsc{nom}  hut.\textsc{nom}   destroy.\textsc{pst}\\
\glt ‘Fire destroyed the hut.’
\z

However, in one special case, it may be used on inanimate referents as well, i.e. precisely when there is no other way to (globally) discriminate P from A (\citealt{AsherKumari1997}: 204, cf. \citealt{Stiebels2002}: 16; \citealt{Subbārāo2012}: 174\-–176):

\ea\label{ex:serzant:11}
\langinfo{Malayalam }{Dravidian: India}{\citealt{AsherKumari1997}: 204}\\
\gll \textup{a.} Kappal   tiramaalakaɭ{}-e   bheediccu.\\
     ship.\textsc{nom}   wave.\textsc{pl-acc(=dom)}  split.\textsc{pst}\\
\glt ‘The ship broke through the waves.’
\z

\ea
\gll \textup{b.} Tiramaalakaɭ kappal-ine     bheediccu.\\
     wave.\textsc{pl}     ship-\textsc{acc(=dom)}   split.\textsc{pst}\\
\glt ‘The ship broke through the waves.’
\z

As in Donno Sɔ above, the discriminatory function becomes visible only in those slots on the referential scales that are not (yet) affected by the scale effects. While in Donno Sɔ the indefinite animate slot became available for the discriminatory function, it is the inanimate slot (both definite and indefinite) in Malayalam. The relative weight of the discriminatory function of the DOM marker in Malayalam is lower than the effect of the referential scales, cf. \REF{ex:serzant:9} again. 

Crucially, if one were to superficially evaluate whether or not the discriminatory function operates in Donno Sɔ or Malayalam, one would have to conclude that it does not, because of the rigid marking of animates (definite animates in Donno Sɔ) and the rigid zero with inanimates. Thus, from the perspective of the discriminatory function, utterances like \REF{ex:serzant:4} redundantly mark their objects; conversely, examples such as \REF{ex:serzant:10} are economical but equally violate the discriminatory function since same-rank A and P are not disambiguated. I summarize:

\ea\label{ex:serzant:12}
The relative weight of the main forces in DOM in Donno Sɔ and Malayalam\\
animacy scale + definiteness scale > economy > discriminatory function\\
\z

Catalan is another example of this pattern. Here, the DOM marker \textit{a} is obligatory only for strong (non-clitic) personal, relative and reciprocal pronouns in the non-colloquial register (cf. \citealt{Escandell-Vidal2009}). Thus, the DOM marker of Catalan is primarily conditioned by the parts-of-speech scale: Pronouns are marked while other NPs are unmarked: 

\ea\label{ex:serzant:13}
Parts-of-speech scale\\
(independent) pronouns > nouns\\
\z

However, the DOM marker may exceptionally appear also with definite animate NPs in the contexts of subject–object ambiguity (\citealt{WheelerEtAl1999}: 243):

\ea\label{ex:serzant:14}
\langinfo{Catalan }{Romance: Spain}{\citealt{WheelerEtAl1999}: 243}\\
\gll T’estima     com a     la   seva     mare.\\
     \textsc{2sg.obj}=love.\textsc{prs.3sg} like \textsc{dom}   \textsc{def.f}   \textsc{3sg.f.poss}   mother\\
\glt ‘She loves you like (she loves) her mother.’
\z

Again, the discriminatory function is subordinate to the parts-of-speech scale \REF{ex:serzant:13}. It may only exceptionally violate the cut-off point between pronouns and nouns on this scale that is otherwise rigid in this language. Additionally, the animacy scale \REF{ex:serzant:6} and definiteness scale \REF{ex:serzant:7} apply in that they determine the NP type for which the discriminatory function may operate: the discriminatory function can only operate on definite animates but not on inanimates or indefinites in this language. I summarize:

\ea\label{ex:serzant:15}
The relative weight of the main forces in DOM in Catalan\\
parts-of-speech scale > animacy + definiteness scale > discriminatory function
\z

The situation in Spanish is somewhat different but largely analogical. Animate and specific NPs must be marked while inanimate and/or non-specific NPs must remain unmarked. However, the DOM marker \textit{a} is obligatory in certain contexts of disambiguation, even with inanimate NPs:

\ea\label{ex:serzant:16}
\langinfo{Spanish }{Romance: Spain}{\citealt{vonHeusingerKaiser2007}: 89}\\
\gll En esta receta, la   leche   puede   sustituir \textbf{a}=l     huevo.\\
     in  \textsc{dem} recipe \textsc{dem}   milk   can   replace \textsc{dom=det}   egg\\
\glt ‘In this recipe egg can replace the milk.’
\z

We observe the same constellation here: the discriminatory function is subordinate to the effects of referential scales. 

Another example is the DOM marker \textit{{}-ǎn} in Hup (Nadahup). It is obligatory with definite animates (including pronouns) as well as with the plural collective marker =\textit{d’ǝh} (\citealt{Epps2008}: 170–177). At the same time, the DOM marker \textit{{}-ǎn} may be used with indefinite animates to discriminate the P argument from A \citep[95]{Epps2009_DOM}. Consider the following example, in which the A argument is left out because it is non-referential: 

\ea\label{ex:serzant:17}
\langinfo{Hup }{Nadahup: Brazil/Columbia}{\citealt{Epps2009_DOM}: 95}\\
\gll Húp-ǎn   tǝ´w-ǝ´y,   húp-ǎn    dóh-óy.\\
     person-\textsc{dom}  scold-\textsc{dyn}  person-\textsc{dom}  curse-\textsc{dyn}\\
\glt ‘(Some people) scold people, cast curses on people.’
\z

The P argument is not referential either, let alone definite. Since it is indefinite, it should not be marked. However, in order to discriminate the P argument from a possible misinterpretation as A, the object marker is used here \citep[95]{Epps2009_DOM}. Again, the discriminatory function is weak because it is subordinate to the referential-scale effects which primarily determine the slots in which the discriminatory function may apply (e.g. on inanimates or indefinites or non-referential NPs, etc.). The relative weight of these is the same as in Catalan in \REF{ex:serzant:15} above. 

The subordinate discriminatory function is found in other Nahadup languages as well. For example, the object marker \textit{{}-\~\i:yɁ} in Dǎw accompanies topical objects but it may also be used for the discriminatory function (\citealt{MartinsMartins1999}: 263–264). 

Similarly, the Papuan language Awtuw obligatorily marks all pronominal and proper-name direct objects regardless of whether there is a need for discrimination or not:

\ea\label{ex:serzant:18}
\langinfo{Awtuw }{Sepik: Papua New Guinea}{\citealt{Feldman1986}: 109}\\
\gll *Wan   rey   du-k-puy-ey.\\
     \textsc{1sg}   \textsc{3m.sg}  \textsc{fa-ipfv}{}-hit-\textsc{ipfv}\\
\glt [Intended meanings] ‘I’m hitting him.’ / ‘He’s hitting me.’
\z

In addition, overt definiteness – marked either by a demonstrative or a possessor NP – has the tendency to attract object marking regardless of the context (\citealt{Feldman1986}: 109\-–110). By contrast, the marking of common nouns is optional. In case of ambiguity it becomes obligatory, or else the NPs will be interpreted as conjoined \citep[110]{Feldman1986}:

\ea\label{ex:serzant:19}
\langinfo{Awtuw }{Sepik: Papua New Guinea}{\citealt{Feldman1986}: 109}\\
\ea
\gll Piyren-re  yaw  di-k-æl-iy.\\
     dog-\textsc{dom}  pig  \textsc{fa-ipfv}{}-bite-\textsc{ipfv}\\
\glt ‘The pig is biting the dog.’
\z

\ex
\gll Piyren  yaw  di-k-æl-iy.\\
     dog  pig  \textsc{fa-ipfv}{}-bite-\textsc{ipfv}\\
\glt ‘The dog and the pig bite.’ / *‘The pig is biting the dog.’ / *‘The dog is biting the pig.’
\z

The situation in Awtuw is slightly different from the one found in the languages above: the slot affected by the discriminatory function (common nouns) already allows for the overt marking; the discriminatory function turns the marking in a particular utterance from optional into obligatory for this particular interpretation.

The prepositional DOM marker \textit{bă} of Chinese primarily occurs before animate, definite or, rarely, indefinite specific preverbal object NPs while postverbal objects are never marked with it (\citealt{LiThompson1981,Bisang1992}: 158–159; \citealt{YangvanBergen2007}):

\ea \label{ex:serzant:20}
\langinfo{Chinese }{Sinitic: China}{\citealt{LiThompson1981}: 464}\\
\gll Tā   bă   fàntīng   shōushi   gānjing le.\\
     \textsc{3sg}  \textsc{dom}  dining.room   tidy.up   clean   \textsc{pfv}\\
\glt ‘S/He tidied up the dining room.’
\z

The discriminatory function as defined in \REF{ex:serzant:3} above is not relevant in \REF{ex:serzant:20}. In addition to the general SVO and S \textit{bă} OV word orders, Chinese also allows for OSV with topical objects and prominent subjects, cf. \REF{ex:serzant:21}:

\ea\label{ex:serzant:21}
\langinfo{Chinese }{Sinitic: China}{\citealt{Bisang1992}: 158}\\
\ea
\gll Láng   Mary   chī-le.\\
     wolf  Mary  eat-\textsc{pfv}\\
\glt ‘Mary ate the wolf.’

\ex
\gll Láng bă   Mary   chī-le.\\
     wolf \textsc{dom}   Mary  eat-\textsc{pfv}\\
\glt ‘The wolf ate Mary.’
\z
\z

To force the interpretation of \REF{ex:serzant:21} with SOV, the \textit{bă} marker has to be used in order to disambiguate the referentially more prominent NP (\textit{Mary}) as P (cf. \citealt{Bisang1992}: 158). Again as in the examples above, the DOM system of Chinese is primarily driven by the cut-off points on referential scales (definiteness, animacy) and some other strong rules pertaining to affectedness, aspectuality and the “disposability” of the object referent (cf. \citealt{LiThompson1981}). Some of these functions are most probably inherited from the source, such as the requirement on disposability or the preverbal position, which may be explained as the retention of the properties of the source construction.\footnote{The \textit{bă} marker stems from the lexical verb ‘to hold’ in a serial verb construction (\citealt{Sun1996}: 61–62).}  The discriminatory function is thus again limited to a particular constellation of \REF{ex:serzant:21} in which the source function, referential scale effects and other forces allow it to operate.

The discriminatory function in Mam (Mayan) is carried out by the obligatory cross-referencing of both A and P on the verb; no flagging is involved. By contrast, the Antipassive form of the verb does not allow for cross-referencing the P argument, which is regularly marked by the preposition / relational noun \textit{{}-iɁj} ‘about’ or \textit{{}-ee} (dative, beneficiary) \citep[212]{England1983}:

\ea 
\langinfo{Mam }{Mayan: Guatemala}{\citealt{England1983}: 213}\\
\gll ma   ø-tzyuu-n   Cheep   *(t-iɁj)   xiinaq\\
     \textsc{rec}   3\textsc{a}{}-grab-\textsc{antip}   Jose   *(3\textsc{sg}{}-\textsc{rn})   man\\
\glt ‘Jose grabbed the man.’
\z

However, “if there is no confusion as to which noun phrase is the agent and which is the patient” the relational noun may be omitted in order to code the meaning of an unintentional act \citep[212]{England1983}:

\ea 
\langinfo{Mam }{Mayan: Guatemala}{\citealt{England1983}: 212—213}\\
\ea
\gll  Ma  ø-tzyuu-n   Cheep   t-iɁj    ch'it. \\
     \textsc{rec} 3\textsc{a}{}-grab-\textsc{antip}   Jose   3\textsc{sg}{}-\textsc{rn} bird\\
\glt ‘Jose grabbed the bird.’

\ex
\gll Ma  ø-tzyuu-n   Cheep   ch'it. \\
     \textsc{rec} 3\textsc{a}{}-grab-\textsc{antip}   Jose   bird\\
\glt ‘Jose unintentionally grabbed the bird.’
\z
\z

The discriminatory function thus delimits the range of the input with which unintentional acts can be expressed (in the Antipassive). In other words, the discriminatory function of flagging is found in a very small subdomain of the language, i.e. in the unintentional use of the Antipassive.

A somewhat different constellation is found in Tamasheq (Berber). The marker \textit{na} (\textit{ná,} \textit{nà} depending on the dialect and tone sandhi) occurs only in SOV word order – never in SVO or VSO – and only if there is no verb inflection (Perfective Indicative), i.e. when no disambiguation via indexing is possible (\citealt{Heath2007}: 92, 94).\footnote{It is referred to as a “bidirectional case marker” in \citet{Heath2007} as well as in the descriptions of some Mande languages, cf. \citet{Diagana1995}, \citet{Nikitina2018}. Bidirectional case markers cannot be straightforwardly related to either A or P marking since they occur only when both are present and do not show any phonetic or syntactic fusion effects. Note that bidirectional case markers are treated under the heading of differential argument marking, cf. \citet{Nikitina2018}.} Moreover, both arguments must be expressed overtly. For example, the marker cannot be used in the imperative with the subject dropped (\citealt{Heath2007}: 92–93). These requirements suggest that the marker is conditioned by the discriminatory function:

\ea 
\langinfo{Tamasheq }{Afro-Asiatic, Berber: North Africa; \citealt{Heath2007}: 91}{glosses adapted}\\
\gll Hàr-òó    nà  háns-òò  kárú.\\
     man-\textsc{det.sg}  \textsc{dom}  dog-\textsc{det.sg}  hit\\
\glt ‘The man hit the dog.’
\z

Without \textit{nà}, both NPs may be misinterpreted as either a compound or as a possessor phrase ‘the man’s dog’ \citep[91]{Heath2007}.

Moreover, some Mande languages such as Soninke, Bambara, Wan or Songhay languages of the area also have similar markers that primarily fulfil the discriminatory function of unambiguous identification of the subject and the object in a clause (\citealt{Heath2007,CreisselsDiagne2013,Nikitina2018}). While Tamasheq, similarly to many Central Mande languages, has generalized the marker, extending it onto all SOV utterances, Wan (South-eastern Mande) employs the marker \textit{laa} predominantly only in those input configurations which are in need of disambiguation given SOV: The marker is used with nominal A and pronominal P (62\%) but not with pronominal A and nominal P (0\%) \citep[202]{Nikitina2018}. In contrast to the languages discussed above, in these languages the discriminatory function is somewhat stronger, as it applies across the board under SOV. Analogically, the DOM marker is optional in the most frequent SOV word order in Korean but becomes almost obligatorily when the object is preposed (OSV) (\citealt{AhnCho2007}).

At least two Loloish languages (Tibeto-Burman) also attest a strong discriminatory function that is not subordinate to some other force. The direct-object markers \textit{tʰ}\textit{aʔ} in Lahu and \textit{tʰ}\textit{ie} in Lolo are only used if the context does not help to discriminate between A and P. That is, these markers code direct objects only where the inherent semantics of the participants (such as animacy) and the semantics of the event fail to do so:

\ea\label{ex:serzant:25}
Yongren Lolo (Tibeto-Burman, Loloish: China; adapted from \citealt{Gerner2008}: 299)\footnote{I simplified the transliteration and slightly adjusted the glossing of all examples from \citet{Gerner2008}.}\\
\gll ƞo   ɕεmo   tʰie   ʈʂɔ   ʑi.\\
     1\textsc{sg}  snake  \textsc{dom}  follow  go\\
\glt ‘I will follow the snake’
\z

\ea\label{ex:serzant:26}
\langinfo{Yongren Lolo }{Tibeto-Burman, Loloish: China}{adapted from \citealt{Gerner2008}: 300}\\
\gll Sɨka   tʰie   χekʰɯ   ti   na.\\
     tree  \textsc{dom}  house  smash  broken\\
\glt ‘The house smashed the tree.’
\z

The absence of the Accusative marker would not be ungrammatical but would create ambiguity as to who is following whom in \REF{ex:serzant:25} or what is smashing what in \REF{ex:serzant:26} (\citealt{Matisoff1973}: 156; \citealt{Gerner2008}). However, along with the synchronically primary function of discriminating P from A (and also R from A), this marker also has the diachronically primary function of coding contrastive focus (\citealt{Gerner2008}: 298–289). For example, \REF{ex:serzant:27a} cannot be used with the DOM marker \textit{tʰ}\textit{ie} because of the lack of a focal contrast. By contrast, \REF{ex:serzant:27b} is acceptable with it if the numeral is interpreted as bearing contrastive focus \citep[299]{Gerner2008}:

\ea\label{ex:serzant:27}
\langinfo{Yongren Lolo }{Tibeto-Burman, Loloish: China}{adapted from \citealt{Gerner2008}: 299}\\
\ea\label{ex:serzant:27a}
\gll Bɔlu   mɔlu     tsɨ   ɔ.\\
     Bolu  trousers  wash  \textsc{prf}\\
\glt ‘Bolu washed trousers.’

\ex\label{ex:serzant:27b}
\gll  Bɔlu   mɔlu     sɔ   khǝ   tʰie   tsɨ   ɔ.\\
     Bolu  trousers  \textsc{num.}3  \textsc{clf}  \textsc{dom}   wash   \textsc{prf}\\
\glt ‘Bolu washed THREE pairs of trousers [not just TWO].’
\z
\z

Importantly, \REF{ex:serzant:27b} may at first glance be interpreted as counterevidence to the discriminatory function because A and P are sufficiently disambiguated by the lexical meanings anyway. Hence, the marking is not due to the discriminatory function. I claim that this is not a piece of counterevidence for the hypothesis of a weak discriminatory function. It may only count as counterevidence for the strong hypothesis of the discriminatory function being the only force constraining DOM (which is counter-intuitive anyway). The source function of marking contrastive focus overrides the discriminatory function here. A situation where various new and inherited functions cluster on one marker is typical of many grammatical categories (cf. \citealt{Hopper1991}: 22). For example, if an indefinite article does not mark plural indefinite NPs but only singular ones, this cannot be taken as counterevidence for its being an indefinite article. A more plausible account is that the restriction to the singular is just the impact of the source meaning. 

Another similar DOM system is the one of Khwe. In this language, proper names must obligatorily be marked with \textit{à/-à}; additionally, this marker encodes contrast and/or focus on the NP (\citealt{KilianHatz2006}: 82–83). At the same time, the marker may also be used in contexts in which the distinction between subject and object would have been impeded, for example, when both arguments are animate and topical (\citealt{KilianHatz2006}: 82–83):

\ea 
\langinfo{Khwe }{Khoe: Southern Africa}{adapted from \citealt{KilianHatz2006}: 83}\\
\ea
\gll  Tcá   tí   à   kx’ṓā´.\\
     \textsc{2sg.m}   \textsc{1sg}   \textsc{dom}  wait\\
\glt ‘You have to wait for me!’

\ex
\gll Yàá!   Cáò   à   tí   kyá-rá-hã!\\
     yes   \textsc{2du.f}   \textsc{dom}  \textsc{1sg}  love-\textsc{act-pst}\\
\glt ‘That’s it! I love you two (women)!’
\z
\z

Further examples may be added. For example, the DOM marker \textit{{}-m} in Imonda (Papuan) is used obligatorily with some verbs such as \textit{eg} ‘to follow’ or \textit{hetha} ‘to hit’ as well as with others to denote something like resultativity (“directionality”) of the action \citep[163]{Seiler1985}. However, in addition to that, the marker may also serve to disambiguate Ps from As when both have similar-rank input \citep[165]{Seiler1985}. Furthermore, DOM in Guaraní is primarily conditioned by animacy, definiteness and topicality but it may also marginally fulfil the discriminatory function (\citealt{Shain2009}: 89–92). In Telkepe (Semitic, Aramaic), the new object marker \textit{ta} may be employed in those situations where agreement alone does not provide for disambiguation while it is otherwise heavily constrained by its meaning of marking topics \citep[354]{Coghill2014}. Finally, \citet{KurumadaJaeger2015} show for Japanese that, in addition to animacy, disambiguation also triggers the DOM marker \textit{{}-o} (see also \citealt{FedzechkinaEtAl2012}).

The discriminatory function may help to explain the world-wide distribution of DOM, namely, why there are more animacy-driven DOM systems than those driven by definiteness and/or specificity across languages. Thus, in a large-scale typological study by \citet[295]{Sinnemäki2014}, roughly 39\% of DOM systems are conditioned by animacy, while DOM systems conditioned by definiteness/specificity are areally biased towards the Old World and occur less frequently (34\% of his sample). I claim that the reason for this is that animate referents are much more strongly associated with the A role than definite/specific referents. Hence, there is a more urgent need with animate than with definite referents for the discriminatory function to apply. A number of corpus studies from various languages show that only animacy shows reversed association tendencies with A and P such that As tend to be animate while Ps tend to be inanimate; by contrast, both As and Ps – with minor differences – tend to be definite and/or specific (\citealt{Dahl2000,Hofling2003,Everett2009,FauconnierVerstraete2014}). 

Finally, there is neurolinguistics evidence for the discriminatory function, suggesting that A and P are not treated symmetrically by the processor. Instead, \citet{Bornkessel-SchlesewskySchlesewsky2015} claim that the effects they observe cannot be explained by simply arising from the degree of semantic associations for the A or P role. Rather, both arguments are interpreted relatively to each other (\citealt{Bornkessel-SchlesewskySchlesewsky2015}: 336). Analogically, \citet{KurumadaJaeger2015} found in their psycholinguistic study on DOM in Japanese that just the properties of the arguments are insufficient to explain the results of their experiments and that the case-marking is affected by the plausibility of role assignment given both arguments and the verb (2015: 161; cf. also \citealt{AhnCho2007,FedzechkinaEtAl2012}).

Above I have argued that the global discriminatory function as defined in \REF{ex:serzant:3} is found to operate in many diverse languages. Moreover, I have found that it is most frequently the weakest force alongside other forces, such as referential scale effects (based on animacy, definiteness or parts of speech) or the source meaning (focus, topic, etc.). All these forces constrain the DOM systems at the same time. The weakness of the discriminatory function is not correlated, I claim, with scarce attestation across languages. On the contrary, I suspect that its impact could be found across most of the DOM systems if one took a closer look at the historical developments and if the synchronic descriptions were more detailed. 

The context-dependent, global discriminatory function in \REF{ex:serzant:3} is relatively costly because it requires whole-utterance planning and online decision making on the part of the speaker. It is costly for the hearer as well since ambiguous NPs (e.g. German \textit{die} \textit{Frau} ‘\textsc{det.nom=acc} woman’) – if placed clause-initially – can only be interpreted by the hearer once enough context has been provided, and not incrementally (\citealt{Bornkessel-SchlesewskySchlesewsky2014}: 107). It is perhaps for this reason that the global discriminatory function often develops into what may be called a \textit{local} \textit{discriminatory} \textit{function} (cf. \citealt{Aissen2003,ZeevatJäger2002,Jäger2004,Malchukov2008}: 208, 213). By virtue of the \textit{local} \textit{discriminatory} \textit{function}, the NP is disambiguated as A or P immediately and regardless of whether the whole utterance might make disambiguation redundant. The local discriminatory function is more efficient because it allows for more reliable incremental processing of the utterance. The degree of efficiency and processability, in turn, correlates with the strength of a force (\citealt{Hawkins2014_CompMot}: 60, 69). This is why the global discriminatory function (cf. \REF{ex:serzant:3}) is a weak force and its effects tend to be generalized over diachronically, for example, by conventionalizing the flagging on those NP types that tend to be disambiguated most frequently or, alternatively, by conventionalizing the marker in those constructions that require disambiguation most frequently (such as SOV in Tamasheq). 

A number of languages have undergone this change towards local disambiguation. I illustrate this with the development of DOM in Russian. I base my argumentation on the philologically profound evidence from Krys’ko (1994; 1997). 

Old Russian inherited from Proto-Slavic the emergent DOM system that evolved in the following way. The direct object was marked by the Accusative case in affirmative clauses and by the Genitive case in clauses with predicate negation. Already during the Proto-Slavic period, the Genitive started penetrating into affirmative transitive clauses \citep{Klenin1983}. The reason is that, under predicate negation, the Genitive no longer carried any functional load but became just a purely syntactically conditioned rule.\footnote{Originally it had an emphatic function similarly to double negation in, for example, French, cf. \citet{Kuryłowicz1971}.} The Genitive was thus just another way of marking direct objects (when the predicate was negated) alongside Accusative. At the same time, due to the overall loss of all word final consonants, the old Accusative and Nominative markers became phonetically indistinguishable in the singular in most of the Proto-Slavic declensions and, subsequently, turned into zero (\tabref{tab:serzant:1}):

\begin{table}
\begin{tabularx}{\textwidth}{lp{2.3cm}p{2.3cm}Q}
\lsptoprule
\bfseries ~ & \bfseries Proto-Slavic Nominative & \bfseries Proto-Slavic Accusative & {\bfseries Resulting form}
\bfseries Accusative = Nominative\\
\midrule
\textit{u}{}-declension & \textit{*-us} & \textit{*-um} & > \textit{*-u} > \textit{-{ъ}} > \textit{ø}\\
\textit{i}{}-declension & \textit{*-is} & \textit{*-im} & > \textit{*-i} > \textit{-{ь}} > \textit{ø}\\
\textit{o}{}-declension & \textit{*-os} > \textit{*-us} & \textit{*-om} > \textit{*-um} & > \textit{*-u} > \textit{-{ъ}} > \textit{ø}\\
\textit{jo}{}-declension & \textit{*jos} > \textit{-jus} & \textit{*jom} > \textit{-jum} & > \textit{*-ju} > \textit{*}\textit{{}-j}\textit{{ъ} }> \textit{-\textsuperscript{j}}\textit{{ь}} > \textit{\textsuperscript{j}}\textit{ø}\\
\lspbottomrule
\end{tabularx}

\caption{Phonetically driven conflation of the old Accusative with the old Nominative in most of the declensions (cf. \citealt{Arumaa1985}: 130)}
\label{tab:serzant:1}
\end{table}

The new DOM marker – i.e. the Genitive case – replaced the old (zero) Accusative only on animate nouns and some pronouns. Importantly, only those animate nouns and pronouns were affected which belonged to the declension classes that did not differentiate between the Nominative and the Accusative anymore (cf. \tabref{tab:serzant:1}). Thus, the expansion of the new DOM marker (Genitive) was crucially conditioned by the local discriminatory function alongside the animacy scale (Krys’ko 1995). 

The evidence for this is abundant: (i) The Genitive did not replace the old Accusative in the \textit{a-}declension because, in this declension, the old Accusative (\nobreakdash-\textit{ǫ} > \textit{-u)} had not become indistinguishable from the Nominative (-\textit{a}) due to nasalization of the former. (ii) The first NP types affected were proper names while personal pronouns generally remained unaffected to begin with, which is atypical of DOM systems that tend to expand along the referential scales.\footnote{There are no unambiguous Genitive forms of pronouns in the position of a direct object in Early Slavic (\citealt{Meillet1897}: 84, 97; \citealt{Vondrák1898}: 327; Krys’ko 1994: 128). Following Meillet, Kuryłowicz (1962: 251) concludes that chronologically, the Accusative-from-Genitive with personal pronouns must be later than with animate masculine nouns.} The reason for this is that personal pronouns had not undergone the phonetic conflation of the Nominative (cf. \textit{az{ъ}} ‘1\textsc{sg.nom}’) and the Accusative (cf. \textit{mȩ} ‘1\textsc{sg.acc}’) and hence were not in need of disambiguation. (iii) The plural of the \textit{o}{}-declension – in contrast to the singular – did retain the phonetic distinction between the old Accusative (\textit{{}-y}) and the old Nominative (-\textit{i}) and thus the old Accusative was not replaced by the new DOM marker here. Only later, between the 14th and 16th c., were both the old Accusative plural and the old Nominative plural conflated into \textit{{}-y}. Precisely from this period onwards, the new DOM marker (Genitive plural) started to be used instead of the Accusative in the plural (Krys’ko 1994: 144). (iv) The third person pronoun \textit{j-} did not have a Nominative form in Early Slavic (various demonstratives were used instead here). Hence, there was no need for disambiguation; Although the form \textit{ji} itself would have been morphologically ambiguous between the Nominative and Accusative, it was reserved for the Accusative only. This pronoun acquired the new DOM marker much later than the relative pronoun \textit{ji-že} (both are etymologically related). Since the relative pronoun \textit{ji-že} did have both the Nominative and the homophonous Accusative forms, it acquired DOM very early. (v) Finally, as \citet{Krys’ko1993} shows, the conflation of the old Nominative with the old Accusative took place much later in the Old Novgorodian dialect, because the latter retained the dedicated Nominative form \textit{{}-e} in the \textit{o-}declension, as opposed to the old Accusative (\textit{{}-}\textit{{ъ}} > ø). The erstwhile retention of the dedicated Nominative affix guaranteed the distinction between A and P and hence no DOM was needed until the Nominative affix disappeared in this dialect, too. 

In all instances in which either the Accusative or the Nominative was not zero or the Nominative did not exist at all, the new DOM marker was introduced much later or not all. It was precisely the Nominative-Accusative syncretism, i.e. the indistinguishability of A and P, that triggered the introduction of the new DOM marker. This relative chronology of the expansion of the Genitive to different NP types suggests that the discriminatory function was the crucial trigger conditioning it (first in \citealt{Dobrovsky1834}: 39; \citealt{Krys’ko1994}: 156; \citealt{Tomson1908,Tomson1909}). Although there is no direct evidence for the global discriminatory function as in \REF{ex:serzant:3}, the consistent application of local disambiguation in different nominal and pronominal classes might suggest that there was a development from global to local disambiguation by means of conventionalization.

The domain of the discriminatory function was determined by a language-specific phonological process, namely, the loss of word-final consonants: Only those declensions were affected which had undergone the phonetic conflation of the old Nominative and Accusative. I conclude that the following forces were crucial in the development of Russian DOM (alongside some others such as analogical levelling):

\ea\label{ex:serzant:29}
The relative weight of the main forces in the development of DOM in Russian\\
complete loss of word-final consonants > discriminatory function > animacy scale\\
\z

It is clear that the complete loss of word-final consonants was a stronger force in Proto-Slavic than the discriminatory function because otherwise the latter would have blocked the former. Crucially, the resulting synchronic picture – if looked at superficially – clearly violates the animacy scale and the global discriminatory function as in \REF{ex:serzant:3}. While some declensions distinguish between animate and inanimate nouns by means of the new DOM marker, other declensions do not have this distinction and mark animate and inanimate Ps indistinguishably. 

\section{Discussion and conclusions}\label{sec:serzant:4}

In this paper I have taken a dynamic perspective on the development of DOM systems. I have provided qualitative evidence from a number of areally and genealogically unrelated languages for the claim that the discriminatory function of case keeps appearing in the diachrony of DOM systems in various subdomains and/or leaves behind traces in the form of local disambiguation. Importantly, the discriminatory function is not dependent on the respective historical source of the DOM marker and its particular developmental path. It is only the range of its application in a particular DOM system that is indeed very much constrained by the source meaning of the marker and/or by scale effects. Even scale effects themselves are sometimes just a strong residual of the source meaning of the DOM marker. For example, DOM markers of many languages (Persian, Romance, Kanuri, etc.) stem from topic markers (cf. \citealt{Iemmolo2010,DalrympleNikolaeva2011}; see also \citealtv{Cristofaro2019tv}). In other instances, the scales are epiphenomenal, as they represent conventionalizations of the most frequent patterns originally conditioned by the discriminatory function (e.g. in Russian).

Thus, the discriminatory function is frequently subordinate to other, stronger pressures, foremost the source meaning of the relevant marker. In addition, pressures like paradigmatic levelling (cf. \citealt{Jäger2007}: 102) or analogical extension play a role in individual systems. Even those DOM systems which are primarily conditioned by the discriminatory function synchronically (such as the one of Yongren Lolo) never have the discriminatory function as the only constraint. I conclude that – even though recurrent from language to language in the transition – the discriminatory function is not strong enough to resist competition with other forces. 

But what conditions the power of the discriminatory function in a particular DOM system? The degree to which the discriminatory function is found to operate synchronically in a particular DOM system or subsystem sometimes correlates positively with how recent the DOM (sub)system is in the language. Thus, the evidence for the discriminatory function is most clearly found in those DOM (sub)systems that emerged relatively recently. For example, the use of the marker \textit{laa} in Wan (South-eastern Mande) to discriminate between A and P is a very recent phenomenon, while its original function was one of marking the focus and the focused agent in a perfect-passive-like construction \citep{Nikitina2018}. In Wan, it is the whole system of differential marking that is recent \citep{Nikitina2018}. In Spanish, only the subsystem of definite inanimate NPs, as in \REF{ex:serzant:16}, has recently been affected by DOM (inanimates are not affected by DOM in Old Spanish). It is this slot where the discriminatory function is found to operate occasionally. But differential object marking as such is quite an old phenomenon in this language.

By contrast, in older DOM systems, the effects of the discriminatory function tend to conventionalize to replace context-dependent rules that are much costlier in processing. The DOM marker is generalized in those contexts that were most frequently in need of global disambiguation. The generalization may proceed (i) along particular NP types or (ii) along particular constructions / word orders. For example, (i) Catalan generalized the DOM marker with personal pronouns regardless of whether there was a need for disambiguation or not in a particular utterance. By contrast, (ii) many Mande languages, Songhay and Tamasheq (Berber) generalized the marker in the APV (SOV) word order with no auxiliaries intervening between A and P in constructions requiring both overt A and P. These were precisely those contexts in which the distinction between A and P was particularly blurred. By contrast, the imperative does fulfil the discriminatory function, albeit in a different way: The sole NP that is expressed overtly is the P argument, while A is dropped. Hence, there was no need for a distinction between A and P by means of flagging here.

There are other types of bivalent constructions, such as equative constructions or comparative constructions, which are also sometimes constrained by the discriminatory function in order for the hearer to coherently process them. Unfortunately, they have never been considered in the general discussion on the discriminatory function of flagging, probably because the conventionalization processes involved here do not proceed along the same scales as the prototypical transitive constructions. However, this effect is certainly just due to different semantic expectations, e.g., as to the standard of comparison and the comparee in the comparative construction, than in a prototypical transitive construction. Furthermore, the discriminatory function of flagging is found to apply in ditransitive constructions of some languages in order to distinguish between A and R, which have similar semantic entailments and thus often do not provide for sufficient cues for the correct interpretation themselves. 

In more general terms, I have argued for the existence of weak universals – a type of universal force that applies across different languages and language families but which is not strong enough to prevail into the synchronic \textsc{stage} 1. I claim that the (global) discriminatory function of flagging is a weak universal. This claim is supported by neurolinguistics and psycholinguistic evidence which suggests that both arguments are interpreted relatively to each other and cannot be reduced to the degree of semantic association of each argument with the role it bears (\citealt{Bornkessel-SchlesewskySchlesewsky2015}: 336; \citealt{AhnCho2007,FedzechkinaEtAl2012,KurumadaJaeger2015}).

Its weakness is possibly motivated by a higher processing load (cf. \citealt{Hawkins2014_CompMot}: 60, 69) as compared to local disambiguation: it requires pre-planning of the whole clause by the speaker and hinders incremental processing by the hearer. By contrast, local disambiguation is straightforward and may be processed incrementally without “having to wait” until sufficient context is provided (cf. \citealt{Bornkessel-SchlesewskySchlesewsky2014}: 107). This is why patterns produced by the (global) discriminatory function often become conventionalized (cf. \citealt{Aissen2003,ZeevatJäger2002,Jäger2004,Malchukov2008}: 208, 213).

The concept of \textsc{strength of universals}, in particular, of weak universals, is relatively new to linguistics (though see \citealt{Bickel2013} for some discussion). However, it ties in with the insight that human cognition in general and language acquisition in particular are better characterized by probabilistic biases or constraints ranging from weak to strong (cf. \citealt{ThompsonEtAl2016}). Moreover, it seems that very strong (absolute) universals have a different motivation than weak universals. The former may indeed reflect some innate properties of human beings, as suggested by nativists (cf. \citealt{Chomsky1965}), though not necessarily domain-specific properties. For example, the universal that all languages must have vowels \citep[19]{Comrie1989} is a very strong, probably, absolute universal. It seems likely that it is caused by innate properties of the human articulatory (and auditory?) apparatus. By contrast, weak universals are rather motivated by cultural evolution, for example, by the strive towards efficient communication between individuals (\citealtv{Haspelmath2019tv}). 

Weak universals constitute a number of challenges for typological research. While strong universals override all potentially competing pressures and can thus be detected by relatively simple techniques, weak universals enter into competition with both other functional motivations as well as language-specific factors, not least the source meaning of the relevant marker (cf. \citealt{Cristofaro2012}, 2017; \citealt{Hammarström2015}). The only way of modelling this adequately is a fine-grained competing-motivations account (cf. \citealt{Haiman1983}; \citealt{DuBois1985,Croft2003}: 59; \citealt{Bickel2014}: 115; \citealt{Hawkins2014_CompMot}: 60, 69; \textit{pace} \citealtv{Cristofaro2019tv}).\footnote{In contrast to optimality-theoretic approaches that also primarily assume competition among universal constraints (cf. \citealt{Aissen2003}), an adequate approach to weak universals has to take language-specific forces into account as well.} For the same reason, weak universals also pose a methodological problem for typological testing for universality, even on the dynamic approach that relies on the transition from \textsc{stage} 0 into \textsc{stage} 1. Dynamic methods based on transitional probabilities do take into account one of the competing motivations, namely, the impact of inheritance (transitional probabilities are measured given the original state, i.e. \textsc{state 0}). However, many other factors that may influence the probability of change towards a particular pattern are glossed over on this approach as well. Finally, weak universals raise an important question about the nature of evidence in typology. Traditionally, typologists have been interested in defining what qualifies as positive evidence. Statistically significant signals that are due to the common genealogical or areal relationships of the languages of the sample have been ruled out as not offering positive evidence for universality. Other types of signals that may not count as positive evidence, such as same-source constructions, have also been identified (cf. \citealt{Cristofaro2017}; \citealtv{Collins2019tv}). At the same time, a definition of what really counts as negative evidence, i.e. the proof of absence, is missing. As was argued in this paper, a random distribution in the sample given coarse data mining methods without taking the dynamic and historical evidence into account, might not be sufficient. This is problematic because, intuitively, it seems probable that strong universals are only the tip of the iceberg, not being numerous in number, while many more universals are rather weak universals of the type investigated here. 

\section*{Acknowledgements} 

My first thanks goes to the first editor of the volume Karsten Schmidtke-Bode, who extensively commented on and discussed with me the earlier versions of the paper. I also thank Eitan Grossman, Martin Haspelmath and Natalia Levshina. The support of the European Research Council (ERC Advanced Grant 670985, Grammatical Universals) is gratefully acknowledged.

\section*{Abbreviations} 

All examples abide by the Leipzig Glossing Rules. Additional abbreviations are: 

\begin{tabularx}{.45\textwidth}{lQ}
\textsc{fa}  factitive \\
\textsc{ps}  person \\
\textsc{dom}  differential object marker 
\end{tabularx}
\begin{tabularx}{.45\textwidth}{lQ}
\textsc{dyn}   dynamic \\
\textsc{rec}  recent past \\
\textsc{rn}   relational noun.
\end{tabularx}

\sloppy
\printbibliography[heading=subbibliography,notkeyword=this] 
\end{document}
\documentclass[output=paper]{langsci/langscibook} 
\author{Matthew S. Dryer\affiliation{University at Buffalo}}
\title{Grammaticalization accounts of word order correlations}
\abstract{This paper examines the role that grammaticalization plays in explaining word order correlations. It presents some data that only grammaticalization accounts for, but also argues that there are correlations that grammaticalization does not account for. The conclusion is that accounts entirely in terms of grammaticalization or accounts that make no reference to grammaticalization are both inadequate.}
\begin{document}
\maketitle 

\section{Introduction}

There is extensive literature both on identifying word order correlations (\citealt{Greenberg1963,Hawkins1983,Dryer1992}) and on possible explanations for these correlations. Proposed explanations can be grouped loosely into three types. First, it is proposed that some correlations exist because of some sort of similarity or shared property of the pairs that correlate. An example of this is the hypothesis that the order of object and verb correlates with the order of adposition and noun phrase because both involve a pair of head and dependent. A second type of explanation is in terms of sentence processing (\citealt{Kuno1974,Dryer1992,Dryer2009_Corr,Hawkins1994_Perf,Hawkins2004_Eff,Hawkins2014_VarEff}), under which the types that do not conform to the correlations are less frequent because structures containing the two inconsistent types are more difficult to parse. This would be what \citetv{Haspelmath2019tv} calls a functional-adaptive type of explanation. A third line of explanation is in terms of grammaticalization (\citealt{Givón1979,HeineReh1984}: 241–244; \citealt{Bybee1988,Aristar1991,DeLancey1994,Collins2012}, \citealtv{Collins2019tv}). For example it is hypothesized that the reason (or a reason) why the order of adposition and noun phrase correlates with the order of verb and object is that one grammaticalization source for adpositions is verbs and the order of verb and object remains the same when the verb grammaticalizes as an adposition. This line of explanation is thus crucially based on the diachronic sources of adpositions and hence a type of source-based explanation (\citealtv{Cristofaro2019tv}).

Despite these competing hypotheses for explaining word order correlations, there is surprisingly little attempt by proponents of an explanation in terms of grammaticalization to argue against other approaches or by proponents of other approaches to argue against grammaticalization. In fact, proponents of other approaches rarely even mention the possible role of grammaticalization. The goal of this paper is to argue that both explanations in terms of grammaticalization and explanations in terms of shared features or processing are needed in explaining word order correlations. I will focus on the pros and cons of grammaticalization explanations, largely ignoring the difference between accounts in terms of shared features and accounts in terms of sentence processing.\footnote{In \citet{Dryer1992}, I argue for a processing account over an account in terms of heads and dependents for the various correlations between pairs of elements and the order of verb and object.}

In \sectref{sec:dryer:2}, I discuss explanations for correlations involving order of adposition and noun phrase and discuss evidence that only a grammaticalization approach can account for. Namely I examine SVO  \& GenN languages that have both prepositions and postpositions and show that not only does an approach involving grammaticalization predict languages with both prepositions and postpositions but it also correctly predicts the semantics associated with each type of adposition. 
In \sectref{sec:dryer:3}, I give reasons why grammaticalization cannot account for all word order correlations, concluding that grammaticalization and other factors conspire to account for some correlations. 
And in \sectref{sec:dryer:4}, I discuss data involving word order properties of definiteness markers where grammaticalization seems to make the wrong prediction.

\section{A grammaticalization account of the correlations with the order of adposition and noun phrase} \label{sec:dryer:2}

In this section, I present evidence for a grammaticalization account for correlations involving the order of adposition and noun phrase that only grammaticalization can account for. In \sectref{sec:dryer:2.1}, I discuss evidence of adpositions grammaticalizing from verbs. In \sectref{sec:dryer:2.2}, I discuss evidence of a second grammaticalization source for adpositions, namely head nouns in genitive constructions. The next section, \sectref{sec:dryer:2.3}, is, in my view, the most important section of this paper. In that section, I discuss SVO languages which employ GenN word order. Grammaticalization theory predicts that in such languages, if adpositions arise from both grammaticalization sources discussed in \sectref{sec:dryer:2.1} and \sectref{sec:dryer:2.2}, the language will have both prepositions and postpositions, the former arising from verbs, the latter from head nouns in genitive constructions, with particular semantics associated with each. I present evidence from a number of languages showing that this prediction is borne out.

\subsection{Adpositions that grammaticalize from verbs}\label{sec:dryer:2.1}

Let me turn now to one of the best-known word order correlations, between the order of object and verb and the order of adposition and noun phrase, where VO languages tend to have prepositions while OV languages tend to have postpositions (\citealt{Greenberg1963,Dryer1992}). Evidence for this correlation is given in Tables \ref{tab:dryer:1} and \ref{tab:dryer:2}. The data for VO languages is given in \tabref{tab:dryer:1}. The numbers in \tabref{tab:dryer:1} denote numbers of genera containing languages of the given sort, divided into five large continental areas \citep{Dryer1989}. The more frequent type in each area is enclosed in square brackets.

\begin{table}
\begin{tabularx}{\textwidth}{Xlrrrrr} 
\lsptoprule
& \bfseries Africa & \bfseries Eurasia & \bfseries Oceania & \bfseries N.America & \bfseries S.America & \bfseries TOTAL\\
\midrule
VO \& Po & 10 & 6 & 3 & 5 & 15 & 39\\
VO \& Pr & [37] & [25] & [48] & [27] & 15 & 152\\
\lspbottomrule
\end{tabularx}
\caption{\label{tab:dryer:1}Order of adposition and noun phrase in VO languages}
\end{table}



\tabref{tab:dryer:1} shows that prepositions outnumber postpositions in VO languages by a wide margin in four of the five areas, with the fifth area (South America) having an equal number of genera containing languages with prepositions and those containing languages with postpositions. Overall, VO \& Pr outnumbers VO \& Po by 152 to 39 genera.

\begin{table}
\begin{tabularx}{\textwidth}{Xlrrrrr} 
\lsptoprule
& \bfseries Africa & \bfseries Eurasia & \bfseries Oceania & \bfseries N.America & \bfseries S.America & \bfseries TOTAL\\
\midrule
OV \& Po & [25] & [45] & [82] & [28] & [41] & 221\\
OV \& Pr & 3 & 2 & 5 & 0 & 1 & 11\\
\lspbottomrule
\end{tabularx}
\caption{\label{tab:dryer:2}Order of adposition and noun phrase in OV languages}
\end{table}

The corresponding data for OV languages is given in \tabref{tab:dryer:2}.
\tabref{tab:dryer:2} shows a stronger preference for postpositions in OV languages than the preference for prepositions in VO languages shown in \tabref{tab:dryer:1}, in that \tabref{tab:dryer:2} shows only 11 genera containing OV \& Pr languages while \tabref{tab:dryer:1} shows 39 genera containing VO \& Po languages. I discuss an explanation for this difference in terms of grammaticalization in §3 below.

An explanation of this correlation in terms of grammaticalization appeals to the fact that verbs are a common source for adpositions, so that when a verb is grammaticalized as an adposition, the order with the verb followed by object is retained as preposition followed by noun phrase, while the order with the object followed by verb is retained as noun phrase plus postposition.

Two examples of this process of grammaticalization in English are the prepositions \textit{including} and \textit{concerning}, as in \REF{ex:dryer:1}.

\ea\label{ex:dryer:1}
{English}\\
\ea  Four men, including John, arrived.\\
\ex  I will talk to you later concerning your thesis.\\
\z
\z

Both of these prepositions retain the present participle form ending in -\textit{ing}, coming from the verbs \textit{include} and \textit{concern}. 

Grammaticalization of adpositions from verbs is common in many other languages and widely described in the literature. The examples in \REF{ex:dryer:2} to \REF{ex:dryer:7} illustrate apparent examples of grammaticalization of particular semantic types of adpositions from verbs with particular meanings.

\paragraph*{give → for}

\ea\label{ex:dryer:2}
\langinfo{Efik}{Niger-Congo, Delta Cross: Nigeria}{\citealt{Givón2001}: 163}\\
\gll   nam  utom  emi  \textbf{ni}  mi.\\
       do  work  this  \textbf{give}  me\\
\glt   ‘Do this work \textbf{for} me.’
\z

\paragraph*{give → to (marking addressee)}

\ea\label{ex:dryer:3}
\langinfo{Yoruba}{Niger-Congo, Defoid: Nigeria}{\citealt{Givón2001}: 163}\\
\gll   mo  so̧ \textbf{fún}  o̧.\\
       I  said \textbf{give}  you\\
\glt   ‘I said \textbf{to} you.’
\z

\paragraph*{go → to (marking goal of motion)}

\ea\label{ex:dryer:4}
\langinfo{Nupe}{Niger-Congo, Ebira-Nupoid: Nigeria}{\citealt{Givón1979}: 221}\\
\gll   ū  bīcī  \textbf{lō}  dzūká.\\
       he  ran  \textbf{go}  market\\
\glt   ‘He ran \textbf{to} the market.’
\z

\paragraph*{follow → with (comitative)}

\ea\label{ex:dryer:5}
\langinfo{Mandarin Chinese}{Sino-Tibetan, Sinitic: China}{\citealt{LiThompson1981}: 423}\\
\gll   tā  bu  \textbf{gēn}  {wǒ}  jiǎng-hua.\\
       3\textsc{sg}  \textsc{neg}  \textbf{follow}  \textsc{1sg}  speak-speech\\
\glt   ‘He doesn't talk \textbf{with} me.’
\z

\paragraph*{take → object case marker}

\ea\label{ex:dryer:6}
\langinfo{Yatye}{Niger-Congo, Idomoid: Nigeria}{\citealt{Givón2001}: 163}\\
\gll     ìywi  \textbf{awá}  \textbf{utsì}  ikù.\\
         boy  \textbf{took} \textbf{door}  shut\\
\glt     ‘The boy shut \textbf{the} \textbf{door}.’
\z

\paragraph*{be at → at}

\ea\label{ex:dryer:7}
\langinfo{Mandarin Chinese}{Sino-Tibetan, Sinitic: China}{Yu Li, p.c.}\\
\gll     tā  \textbf{zài}  {guō-li}  chǎo  fàn.\\
         3\textsc{sg}  \textbf{be.at}  pot-in  fry  rice\\
\glt     ‘He is frying rice \textbf{in} the pot.’
\z

The grammaticalization of adpositions from verbs provides a possible basis for an explanation of the correlation between the order of verb and object and the order of adposition and noun phrase.

\subsection{Adpositions that grammaticalize from head nouns in nominal possessive constructions}\label{sec:dryer:2.2}

Another common grammaticalization source for adpositions (and probably the more common source) is head nouns in genitive constructions. English has a number of prepositions that have arisen from head nouns in genitive constructions, including those in \REF{ex:dryer:8}.
\newpage
\ea\label{ex:dryer:8}
{English}\\
\ea in the side of NP → inside NP\\
\ex by the side of NP → beside NP\\
\ex by the cause of NP → because of NP\\
\z
\z

Because these adpositions arose from head nouns in a genitive construction with NGen order, they ended up as prepositions rather than postpositions. The opposite situation arose in Amharic, where the examples in \REF{ex:dryer:9} illustrate two postpositions arising from head nouns in a GenN construction.

\ea\label{ex:dryer:9}
\langinfo{Amharic}{Afro-Asiatic, Semitic: Ethiopia}{\citealt{Givón1971}: 399} \\
\ea
NP + bottom → NP + under\\

\gll   kä-bet    \textbf{tač}    allä.\\
       at-house  \textbf{bottom}  is\\
\glt   ‘He is \textbf{under} the house.’

\ex  NP + reason → NP + because of\\

\gll   bä-ɨssu  \textbf{mɨknɨyat}  näw.\\
       at-he    \textbf{reason}  is\\
\glt   ‘It is \textbf{because} \textbf{of} him.’
\z
\z

This type of grammaticalization of adpositions from head nouns in genitive constructions would explain the correlation between the order of noun and genitive and the order of adposition and noun phrase. The data in Tables \ref{tab:dryer:3} and \ref{tab:dryer:4} provides evidence for this correlation. The data in \tabref{tab:dryer:3} shows GenN languages being overwhelmingly postpositional, while the data in \tabref{tab:dryer:4} shows NGen languages being overwhelmingly prepositional.

\begin{table}
\begin{tabularx}{\textwidth}{Xlrrrrr}
\lsptoprule
& \bfseries Africa & \bfseries Euras & \bfseries Oceania & \bfseries N.Amer & \bfseries S.Amer & \bfseries TOTAL\\
\midrule 
GenN \& Po & [31] & [50] & [73] & [33] & [54] & 241\\
GenN \& Pr & 2 & 7 & 13 & 3 & 4 & 29\\
\lspbottomrule
\end{tabularx}
\caption{\label{tab:dryer:3}Order of adposition and noun phrase in GenN languages}
\end{table}


\begin{table}
\begin{tabularx}{\textwidth}{Xlrrrrr}
\lsptoprule
& \bfseries Africa & \bfseries Euras & \bfseries Oceania & \bfseries N.Amer & \bfseries S.Amer & \bfseries TOTAL\\
\midrule 
NGen \& Po & 7 & 0 & 6 & 0 & 1 & 14\\
NGen \& Pr & [34] & [19] & [29] & [19] & [10] & 111\\
\lspbottomrule
\end{tabularx}
\caption{\label{tab:dryer:4}Order of adposition and noun phrase in NGen languages}
\end{table}


\subsection{An interesting prediction of grammaticalization accounts for adpositions}\label{sec:dryer:2.3}


Sections \sectref{sec:dryer:2.1} and \sectref{sec:dryer:2.2} illustrate two grammaticalization sources for adpositions, one from verbs, the other from head nouns in genitive constructions. In languages which are VO and NGen, both sources will lead to prepositions rather than postpositions. Conversely, in languages which are OV and GenN, both sources will lead to postpositions. But there are many languages which are VO but GenN. \citet{Dryer1997_Six,Dryer2013_Six2} shows that although OV languages tend to be GenN and verb-initial languages tend to be NGen, both orders of noun and genitive are common among SVO languages, as shown in \tabref{tab:dryer:5}.

\begin{table}
\begin{tabularx}{\textwidth}{Xlrrrrr} 
\lsptoprule
& \bfseries Africa & \bfseries Euras & \bfseries Oceania & \bfseries N.Amer & \bfseries S.Amer & \bfseries TOTAL\\
\midrule
SVO \& GenN & 11 & 11 & 13 & 2 & [11] & 48\\
SVO \& NGen & [41] & [16] & 13 & [5] & 3 & 78\\
\lspbottomrule
\end{tabularx}
\caption{\label{tab:dryer:5}Order of genitive and noun in SVO languages}
\end{table}


\tabref{tab:dryer:5} shows that NGen is more common overall than GenN among SVO languages by 78 genera to 48. However, the higher number of genera containing SVO \& NGen languages turns out to be entirely due to languages in Africa. Outside Africa, SVO \& NGen and SVO \& GenN are both found in exactly 37 genera. The general conclusion is that there is no evidence of any preference for NGen order over GenN order among SVO languages.

Because of the two grammaticalization sources for adpositions described in the two preceding sections, grammaticalization theory makes an interesting prediction about SVO \& GenN languages. Namely, if adpositions arise in any such languages from both grammaticalization sources, the language should have both prepositions and postpositions, those arising from verbs being prepositions and those arising from nouns being postpositions. The evidence in this section shows that this prediction is borne out.

In fact, grammaticalization theory makes more specific predictions about what meanings will be associated with prepositions and what meanings will be associated with postpositions in such languages. The lefthand column in \tabref{extab:dryer:10} summarizes typical meanings associated with adpositions that arise from verbs, while the righthand column summarizes typical meanings associated with adpositions that arise from head nouns in genitive constructions.

\begin{table}
\caption{Typical meanings associated with adpositions} 
\label{extab:dryer:10}
\begin{tabularx}{\textwidth}{lXlX}
\lsptoprule
\bfseries & \bfseries Typical meanings associated with  adpositions that come from verbs & \bfseries & \bfseries Typical meanings associated with adpositions that come from nouns\\
\midrule 
 & {benefactive (‘for’)}

{instrumental (‘with’)}

{comitative (‘with’)}

{similative (‘like’)}

{allative (‘to, toward’)}

{general locations (‘at’)}

adpositions marking direct objects &  & {specific locations like}

{‘under’, ‘behind’, ‘in front of’}

‘because of’\\
\lspbottomrule
\end{tabularx}
\end{table}

Note that it is typically adpositions denoting specific locations that arise from nouns. Adpositions denoting general locations (meaning something like ‘at’) often arise from verbs. Similarly adpositions associated with motion away from or towards a location also more often arise from verbs. Grammaticalization theory predicts that in an SVO \& GenN language with both prepositions and postpositions, the prepositions will tend to have meanings like those in the lefthand column in \tabref{extab:dryer:10}, while the postpositions will tend to have meanings like those in the righthand column. This section shows that these predictions are also borne out.

The first language illustrating how these predictions are borne out is Nǀuuki. The SVO order of Nǀuuki is illustrated in \REF{ex:dryer:11}, GenN order in \REF{ex:dryer:12}.

\ea\label{ex:dryer:11}
\langinfo{Nǀuuki}{Tuu: South Africa}{\citealt{CollinsNamaseb2011}: 10}\\
\gll ǂharuxu    ke  ãi  ʘoe.\\
       Haruxu  \textsc{decl}  eat  meat \\
\glt ‘Haruxu is eating meat.’  
\z

\ea\label{ex:dryer:12}
\langinfo{Nǀuuki}{Tuu: South Africa}{\citealt{CollinsNamaseb2011}: 37}\\
\gll siso  ŋǂona\\
       Siso  knife\\
\glt   ‘Siso’s knife’
\z

Nǀuuki has both prepositions and postpositions. Examples illustrating a preposition \textit{ŋǀa} are given in \REF{ex:dryer:13} and \REF{ex:dryer:14}, \REF{ex:dryer:13} illustrating an instrumental use, \REF{ex:dryer:14} a comitative use.

\ea\label{ex:dryer:13}
\langinfo{Nǀuuki}{Tuu: South Africa}{\citealt{CollinsNamaseb2011}: 25}\\
\gll n-a  si  ǀaa  ʘoe  \textbf{ŋǀa}  \textbf{ŋǂona}.\\
       1\textsc{sg-decl}  \textsc{irr}  cut  meat  \textbf{with}  \textbf{knife}  \\
\glt ‘I will cut the meat \textbf{with} \textbf{a} \textbf{knife}.’  
\z

\ea\label{ex:dryer:14}
\langinfo{Nǀuuki}{Tuu: South Africa}{\citealt{CollinsNamaseb2011}: 25}\\
\gll ǀaǀaʕe  ke  sĩĩsen  \textbf{ŋǀa} \textbf{ŋǀaŋgusi}.\\
       ǀaǀaʕe  \textsc{decl}  work  \textbf{with}  \textbf{Nǀaŋgusi}  \\
\glt ‘ǀaǀa\textsuperscript{ʕ}e works \textbf{with} \textbf{Nǀaŋgusi}.’  
\z

In contrast, example \REF{ex:dryer:15} illustrates a postposition \textit{xuu} ‘in front of’.

\ea\label{ex:dryer:15}
\langinfo{Nǀuuki}{Tuu: South Africa}{\citealt{CollinsNamaseb2011}: 80}\\
\gll ǀx’esi  ǀʔaa  s\~ui  \textbf{ǀoβa}  \textbf{xuu}\\
       necklace  go  sit.down  \textbf{child}  \textbf{front}  \\
\glt ‘The necklace fell \textbf{in} \textbf{front} \textbf{of} \textbf{the} \textbf{child}.’ 
\z

The prepositions and postpositions in Nǀuuki (\citealt{CollinsNamaseb2011}: 24–25) are listed in \tabref{extab:dryer:16}.


\begin{table}
\caption{Prepositions and postpositions in Nǀuuki }
\label{extab:dryer:16}
\begin{tabularx}{\textwidth}{lll lll} 
\lsptoprule
 & \multicolumn{2}{c}{\bfseries Prepositions} &  & \multicolumn{2}{c}{\bfseries Postpositions}\\
\midrule
 & ŋǀa & ‘instrumental, comitative’ &  & ǀǀãʔẽ & ‘in’\\
 & ǀǀa & ‘like’ &  & xuu & ‘in front of’\\
 & ŋ & ‘linker’ &  & tsʔĩi & ‘behind’\\
 &  &  &  & ǀqhaa & ‘next to’\\
\lspbottomrule
\end{tabularx}
\end{table}

Two of the three prepositions, \textit{ŋǀa} ‘instrumental, comitative’ and \textit{ǀǀa} ‘like’, conform to the semantic types of adpositions arising from verbs and the fact that they are prepositions rather than postpositions can be explained if they have arisen from verbs in a VO language. And all four of the postpositions represent specific locations, conforming to what we expect semantically of adpositions arising from head nouns in genitive constructions; the fact that they are postpositions rather than prepositions can be explained in that they have arisen from head nouns in a genitive construction in a GenN language.

A second example is provided by Logba. Like Nǀuuki, Logba is SVO, as illustrated in \REF{ex:dryer:17}, and GenN, as in \REF{ex:dryer:18}.

\ea\label{ex:dryer:17}
\langinfo{Logba}{Niger-Congo, Kwa: Ghana}{\citealt{Dorvlo2008}: 105}\\
\gll Setor  ó-kpe  i-gbeɖi=é.\\
       Setor  \textsc{sg}{}-peel  \textsc{nc}{}-cassava=\textsc{det}\\
\glt   ‘Setor peeled the cassava.’ 
\z

\ea\label{ex:dryer:18}
\langinfo{Logba}{Niger-Congo, Kwa: Ghana}{\citealt{Dorvlo2008}: 71}\\
\gll Kɔdzo    a-klɔ=a\\
       Kɔdzo    \textsc{nc}{}-goat=\textsc{det}\\
\glt   ‘Kɔdzo’s goat’ 
\z

Also like Nǀuuki, Logba has both prepositions and postpositions. The preposition \textit{kpɛ} with instrumental or comitative meaning is illustrated in \REF{ex:dryer:19}.

\ea\label{ex:dryer:19}
\langinfo{Logba}{Niger-Congo, Kwa: Ghana}{\citealt{Dorvlo2008}: 96}\\
\gll Udzi=é  ó-glɛ  uzugbo  \textbf{kpɛ}  \textbf{a-futa}.\\
       woman=\textsc{det}  \textsc{sg}{}-tie  head  \textbf{with}  \textbf{\textsc{nc}}\textbf{{}-cloth}  \\
\glt ‘The lady tied her head \textbf{with} \textbf{a} \textbf{cloth}.’  
\z

In contrast, an example illustrating a postposition \textit{etsi} ‘under’ is given in \REF{ex:dryer:20}.

\ea\label{ex:dryer:20}
\langinfo{Logba}{Niger-Congo, Kwa: Ghana}{\citealt{Dorvlo2008}: 98}\\
\gll i-datɔ=a  í-tsi  a-fúta=á  etsi.\\
       \textsc{nc}{}-spoon=\textsc{det}  \textsc{sg}{}-be.in  \textbf{\textsc{nc-}}\textbf{cloth}=\textbf{\textsc{det}}  \textbf{under}\\
\glt   ‘The spoon is \textbf{under} \textbf{the} \textbf{cloth}.’ 
\z

In \tabref{extab:dryer:21} is a list of the prepositions and postpositions of Logba (\citealt{Dorvlo2008}: 95, 98).


\begin{table}
\caption{Prepositions and postpositions of Logba }
\label{extab:dryer:21}
\begin{tabularx}{\textwidth}{lll lll}
\lsptoprule
 & \multicolumn{2}{c}{\bfseries Prepositions} &  & \multicolumn{2}{c}{\bfseries Postpositions}\\
 \midrule 
 & fɛ & ‘at’ &  & nu & ‘inside’\\
 & na & ‘on’ &  & etsi & ‘under’\\
 & kpɛ & ‘instrumental, comitative’ &  & tsú & ‘on’\\
 & gu & ‘about’ &  & ité & ‘in front of’\\
 & dzígu & ‘from’ &  & zugbó & ‘on’\\
 &  &  &  & yó & {‘surface contact’}\newline (e.g. on a wall)\\
 &  &  &  & anú & ‘at tip of, at edge of’\\
 &  &  &  & otsoe & ‘on the side of’\\
 &  &  &  & amá & ‘behind’\\
\lspbottomrule
\end{tabularx}
\end{table}


While one of the prepositions has a meaning more commonly associated with adpositions that arise from nouns (\textit{na} ‘on’), the other prepositions all have meanings that grammaticalization theory predicts for adpositions arising from verbs and all the postpositions have meanings involving specific locations, the types of meanings that grammaticalization predicts for adpositions that arise from nouns.

The third SVO \& GenN language with both prepositions and postpositions is Eastern Kayah Li, a Karenic language in the Sino-Tibetan family spoken in Myanmar and Thailand. The prepositions and postpositions of Eastern Kayah Li are listed in \tabref{extab:dryer:22} (\citealt{Solnit1997}: 209–214).

\begin{table}
\caption{Prepositions and postpositions of Eastern Kayah Li}
\label{extab:dryer:22}
\begin{tabularx}{\textwidth}{lll@{}lll}
\lsptoprule
  & \multicolumn{2}{c}{\bfseries Prepositions} &  & \multicolumn{2}{c}{\bfseries Postpositions}\\
\midrule
 & dɤ & ‘at’ &  & kū & ‘inside’\\
 & mú & ‘at’ &  & klɔ & ‘outside’\\
 & bɤ & ‘at’ &  & khu & ‘on, above’\\
 & bá & ‘as much as’ &  & kɛ {\textasciitilde} kɛdē & ‘down inside’\\
 & tí & ‘as big as’ &  & khʌ & ‘at apex of’\\
 & tɤ {\textasciitilde} thɤ & ‘as long as’ &  & lē & ‘under, downhill from’\\
 & phú {\textasciitilde} hú & ‘like’ &  & chá & ‘near’\\
 &  &  &  & ŋē {\textasciitilde} béseŋē & ‘in front of’\\
 &  &  &  & khjā {\textasciitilde} békhjā & ‘behind’\\
 &  &  &  & lo & ‘on non-horizontal surface’\\
 &  &  &  & klē & ‘in (an area)’\\
 &  &  &  & rɔklē & ‘beside’\\
 &  &  &  & ple {\textasciitilde} ple kū & ‘in narrow space between’\\
 &  &  &  & cɔkū & ‘in middle of, between’\\
 &  &  &  & thɯ & ‘on edge of’\\
 &  &  &  & təkjā & ‘in the direction of’\\
\lspbottomrule
\end{tabularx}
\end{table}


Apart from three prepositions with unusual meanings (‘as much as’, ‘as big as’, ‘as long as’), the rest of the prepositions and all of the postpositions have meanings conforming to the semantics typically associated with adpositions arising from verbs and adpositions arising from head nouns in genitive constructions respectively.

The fourth SVO \& GenN language exhibiting a similar pattern is Jabem, an Oceanic language in the Austronesian family spoken in Papua New Guinea. In \tabref{extab:dryer:23} is a list of the prepositions and postpositions of Jabem (\citealt{Dempwolff1939,BradshawCzobor2005}: 42–44; \citealt{Ross2002}: 291).

\begin{table}
\caption{Prepositions and postpositions of Jabem }
\label{extab:dryer:23}
\begin{tabularx}{\textwidth}{lll lll} 
\lsptoprule
 & \multicolumn{2}{c}{\bfseries Prepositions} &  & \multicolumn{2}{c}{\bfseries Postpositions}\\
\midrule
 & tamiŋ & ‘next to, onto’ &  & lêlôm & ‘inside’\\
 & baŋ & ‘close to’ &  & lôlôc & ‘on top of’\\
 & paŋ & ‘close to’ &  & làbu & ‘under’\\
 & ŋa & ‘instrumental’ &  & sawa & ‘between’\\
 & a\textsuperscript{ŋ}ga & ‘from’ &  & lùŋ & ‘in middle of’\\
 &  &  &  & nêm & ‘in front of’\\
 &  &  &  & mu & ‘behind’\\
 &  &  &  & gala & ‘near’\\
 &  &  &  & tali & ‘at edge of’\\
\lspbottomrule
\end{tabularx}
\end{table}

While all the postpositions again have meanings denoting specific locations, as we would expect of adpositions arising from head nouns in genitive constructions, three of the prepositions also have meanings of that sort (‘next to’, ‘close to’). In fact, Dempwolff specifically suggests that these prepositions arose from verbs (suggesting, for example, that \textit{tamiŋ} ‘next to’ comes from a verb meaning ‘to be close upon’).\footnote{I base this on \citegen{BradshawCzobor2005} English translation of \citet{Dempwolff1939}.}
\newpage
In \tabref{extab:dryer:24} to \ref{extab:dryer:29} are lists of prepositions and postpositions from six other SVO \& GenN languages that have both. All show patterns similar to those in the four languages described above in this section, with the prepositions having meanings associated with adpositions arising from verbs and the postpositions with meanings associated with adpositions arising from nouns.

\begin{table}
\caption{ǂHoã (Kxa: Botswana, \citealt{CollinsGruber2014}: 101–105)}
\label{extab:dryer:24} 

\begin{tabularx}{\textwidth}{lll lll} 
\lsptoprule
 & \multicolumn{2}{c}{\bfseries Prepositions} &  & \multicolumn{2}{c}{\bfseries Postpositions}\\
\midrule
 & kì & ‘linker’ &  & na & ‘in’\\
 & ke & ‘comitative’ &  & za & ‘by, beside’\\
 &  &  &  & ǀǀq'am & ‘above’\\
 &  &  &  & ǂkȁ & ‘below’\\
 &  &  &  & ǂ’hàã & ‘in front of’\\
 &  &  &  & kya“m & ‘near’\\
\lspbottomrule
\end{tabularx}
\end{table}

\begin{table}
\caption{Koromfe (Niger-Congo, Gur: Burkina Faso, Mali, \citealt{Rennison1997_Koromfe,Rennison2017})}
\label{extab:dryer:25} 
\fittable{
\begin{tabular}{lll lll} 
\lsptoprule
 & \multicolumn{2}{c}{\bfseries Prepositions} &  & \multicolumn{2}{c}{\bfseries Postpositions}\\
\midrule
 & la & ‘instrumental, comitative’ &  & nɛ & ‘benefactive, purpose, about’\\
 & hal & ‘until’ &  & kana & ‘like’\\
 &  &  &  & dɔba & ‘on top of’\\
 &  &  &  & hɛrəga & ‘beside, near’\\
 &  &  &  & hogo & ‘under’\\
 &  &  &  & jɪka nɛ & ‘in front of’\\
 &  &  &  & joro & ‘in, inside’\\
 &  &  &  & bɛllɛ & ‘behind’\\
 &  &  &  & tʊllɛ & ‘in the middle of, between’\\
\lspbottomrule
\end{tabular}
}
\end{table}

\begin{table}
\caption{Mandarin Chinese (Sino-Tibetan, Sinitic: China, \citealt{LiThompson1981})}
\label{extab:dryer:26} 
\begin{tabularx}{\textwidth}{lll lll} 
\lsptoprule
& \multicolumn{2}{c}{\bfseries Prepositions (or coverbs)} &  & \multicolumn{2}{c}{\bfseries Postpositions (or locative particles)}\\
\midrule
 & gēn & ‘with (comitative)’ &  & shàng & ‘on top of, above’\\
 & gěi & ‘for’ (benefactive) &  & xià & ‘below’\\
 & bǎ & object marker &  & lǐ & ‘in, inside’\\
 & duì & ‘toward’ &  & wài & ‘outside’\\
 & cóng & ‘from’ &  & qián & ‘in front of’\\
 & zài & ‘at’ &  & hòu & ‘behind’\\
 & tì & ‘instead of’ &  & páng & ‘beside’\\
 & bèi & ‘by’ &  & dōngbu & ‘east of’\\
 & àn & ‘according to’ &  & zhèr & ‘this side of’\\
 & dào & ‘to’ &  & qián & ‘in front of’\\
 &  &  &  & hòu & ‘behind’\\
 &  &  &  & páng & ‘beside’\\
 &  &  &  & zhōngjian & ‘in the centre of’\\
\lspbottomrule
\end{tabularx}
\end{table}

\begin{table}
\caption{Koyra Chiini (Songhay: Mali, \citealt{Heath1999}: 104–109)}
\label{extab:dryer:27}
\begin{tabularx}{\textwidth}{lll lll}
\lsptoprule
  & \multicolumn{2}{c}{\bfseries Prepositions} &  & \multicolumn{2}{c}{\bfseries Postpositions}\\
\midrule
 & nda & ‘comitative, instrumental’ &  & se & ‘dative’\\
 & bilaa & ‘without’ &  & ra & ‘locative’\\
 & hal & ‘until’ &  & ga & ‘beside, from’\\
 & jaa & ‘since’ &  & doo & ‘at the place of’\\
 & bara & ‘except’ &  & banda & ‘behind’\\
 & kala & ‘except’ &  & beene & ‘above’\\
 &  &  &  & čire & ‘under’\\
 &  &  &  & kuna & ‘in’\\
 &  &  &  & jere & ‘beside’\\
 &  &  &  & jine & ‘in front of’\\
 &  &  &  & maasu & ‘inside’\\
 &  &  &  & tenje & ‘facing’\\
\lspbottomrule
\end{tabularx}
\end{table}

\begin{table}
\caption{Taba (Austronesian, South Halmahera: Indonesia, \citealt{Bowden2001}: 109–111)}
\label{extab:dryer:28}
\begin{tabularx}{\textwidth}{lll lll}
\lsptoprule
& \multicolumn{2}{c}{\bfseries Prepositions} &  & \multicolumn{2}{c}{\bfseries Postposition}\\
\midrule
 & ada & ‘comitative, instrumental’ &  & li & ‘on, in, at’\footnote{The fact that the one postposition in Taba has general locative meaning does not fit the expectations for a postposition in a GenN language. But the fact that it is locative while the prepositions are not does fit loosely. It is possible that it originally had a narrower locative meaning that has become bleached.}\\
 & pake & ‘instrumental’ &  &  & \\
 & untuk & ‘benefactive’ &  &  & \\
 & lo & ‘like’ &  &  & \\
\lspbottomrule
\end{tabularx} 
\end{table}

\begin{table}
\caption{Dagbani (Niger-Congo, Gur: Ghana, \citealt{Olawsky1999})}
\label{extab:dryer:29}

\begin{tabularx}{\textwidth}{lll lll} 
\lsptoprule
  & \multicolumn{2}{c}{\bfseries Prepositions} &  & \multicolumn{2}{c}{\bfseries Postpositions}\\
\midrule
 & ni & ‘comitative, instrumental’ &  & nyaaŋa & ‘behind’\\
 & jɛndi & ‘about, concerning’ &  & zuɣu & ‘on top of’\\
 &  &  &  & gbinni & ‘under’\\
 &  &  &  & sani & ‘towards’\\
 &  &  &  & sunsuuni & ‘in the middle of’\\
 &  &  &  & ni & ‘in, at, to’\\
 &  &  &  & puuni & ‘inside’\\
 &  &  &  & polo & ‘in the direction of’\\
 &  &  &  & lɔŋni & ‘under’\\
\lspbottomrule
\end{tabularx}
\end{table}

\clearpage 
The languages illustrated in \tabref{extab:dryer:16} to \tabref{extab:dryer:29}
above are instances of SVO languages with GenN order and both prepositions and postpositions. Though less common, there are also languages of the opposite sort, OV languages with NGen order and both prepositions and postpositions, where the semantics associated with prepositions and postpositions respectively is the opposite of that found in SVO \& GenN languages. An example is Iraqw. Example \REF{ex:dryer:30} illustrates the preposition \textit{daandú} ‘behind’. That it has grammaticalized from the head noun in a genitive construction is clear from the fact that it occurs in construct state, the morphological form that head nouns take in genitive constructions.

\ea\label{ex:dryer:30}
\langinfo{Iraqw}{Afro-Asiatic, Cushitic: Tanzania}{\citealt{Mous1993}: 97}\\
\gll   looʾa  i  \textbf{daandú}  \textbf{hunkáy}.\\
       sun  3\textsc{sbj}  \textbf{behind.\textsc{constr}}  \textbf{cloud} \\
\glt ‘The sun is \textbf{behind} \textbf{the} \textbf{cloud}.’
\z

In contrast, example \REF{ex:dryer:31} illustrates a postpositional clitic =\textit{i} ‘directional’ that attaches to the last word in the noun phrase. In \REF{ex:dryer:31} it attaches to the noun \textit{doʾ} ‘house’, the possessor of \textit{afkú} ‘mouth’ (‘door’), but it is marking the entire noun phrase \textit{afkú} \textit{doʾ} ‘mouth (door) of the house’ as the goal of the motion denoted by the verb \textit{qaas} ‘put’.

\ea\label{ex:dryer:31}
\langinfo{  Iraqw}{Afro-Asiatic, Cushitic: Tanzania}{\citealt{Mous1993}: 252}\\
\gll famfeʾamo  u-n  \textbf{af-kú}  \textbf{doʾ=i}  qaas-áan.\\
       snake  \textsc{masc.obj-expec}  \textbf{mouth-\textsc{constr.masc}}  \textbf{house=\textsc{dir}}  put-\textsc{1pl}  \\
\glt ‘Let us put a snake \textbf{on} \textbf{the} \textbf{door} \textbf{of} \textbf{the} \textbf{house}.’
\z

In \tabref{extab:dryer:32} is a list of prepositions and postpositions in Iraqw (\citealt{Mous1993}: 95–107).


\begin{table}
\caption{Prepositions and postpositions in Iraqw}
\label{extab:dryer:32}
\begin{tabularx}{\textwidth}{lll lll}
\lsptoprule
 & \multicolumn{2}{c}{\bfseries Prepositions} &  & \multicolumn{2}{c}{\bfseries Postpositions}\\
\midrule
 & ar & ‘instrumental’ &  & =(a)r & ‘instrumental, comitative’\\
 & as & ‘because of’ &  & =sa & ‘because of’\\
 & ay & ‘to’ &  & =i & ‘to’\\
 & dír & ‘to' &  & =wa & ‘from’\\
 & amór & ‘at’ &  &  & \\
 & daandú & ‘on’ &  &  & \\
 & alá & ‘behind’ &  &  & \\
 & gurúu & ‘inside’ &  &  & \\
 & gamú & ‘under’ &  &  & \\
 & bihháa & ‘beside’ &  &  & \\
 & tlaʿá(ng) & ‘between’ &  &  & \\
 & tseeʿá & ‘outside’ &  &  & \\
 & afíqoomár & ‘until’ &  &  & \\
 & gawá & ‘on’ &  &  & \\
 & geerá & ‘before’ &  &  & \\
 & afá & ‘at the edge of’ &  &  & \\
 & bará & ‘in’ &  &  & \\
\lspbottomrule
\end{tabularx}
\end{table}

Setting aside momentarily the first three prepositions in \tabref{extab:dryer:32}, the semantics associated with the prepositions and postpositions in Iraqw is the reverse of what we found in \REF{ex:dryer:11} to \REF{extab:dryer:29} for SVO \& GenN languages. Namely, in \tabref{extab:dryer:32}, it is the prepositions which denote specific locations, while the postpositions have meanings that are generally associated with adpositions arising from verbs.

The first three prepositions in \tabref{extab:dryer:32} have the same meanings as the first three postpositions in the table. Their meanings are thus ones that we might have expected to be associated with postpositions in an OV language. These prepositions take the form of /a/ plus the corresponding postpositional clitics. \citet[102]{Mous1993} speculates that the /a/ in these forms may have originally been the copula \textit{a}. It is possible that these prepositions have arisen by analogy to other prepositions in the language.

A second instance of an OV \& NGen language with both prepositions and postpositions is Kanuri. Example \REF{ex:dryer:33} illustrates the locative-instrumental postpositional clitic =\textit{lan} attaching to a postnominal modifier \textit{Musa=be} ‘Musa’s’, marking \textit{fər} \textit{Musa=be} ‘Musa’s horse’ as an instrumental.\footnote{There are thus two postpositional clitics in the phonological word \textit{Musa=be=lan} in \tabref{extab:dryer:32}, the \textit{=be} marking Musa as possessor of \textit{fər} ‘horse’ and the =\textit{lan} marking  \textit{fər} \textit{Musa=be} ‘Musa’s horse’ as an instrumental.}

\ea\label{ex:dryer:33}
\langinfo{Kanuri}{Saharan: Nigeria, Niger}{\citealt{Hutchinson1976}: 5}\\
\gll [\textbf{fər}  \textbf{Musa=be}]\textbf{=lan}  kadio.\\
       [\textbf{horse}  \textbf{Musa=\textsc{gen}}]=\textbf{\textsc{ins}}  come.\textsc{pst}.3\textsc{sg}  \\
\glt ‘He came \textbf{on/by} \textbf{Musa’s} \textbf{horse}.’
\z

Kanuri also has prepositions, like \textit{suro} ‘inside’ in \REF{ex:dryer:34}.

\ea\label{ex:dryer:34}
\langinfo{Kanuri}{Saharan: Nigeria, Niger}{\citealt{Hutchinson1976}: 80}\\
\gll   \textbf{suro}  \textbf{fato=be=ro}  kargawo.\\
       \textbf{inside}  \textbf{house=\textsc{gen}}\textbf{=to}  enter.\textsc{pst}.3\textsc{sg} \\
\glt   ‘He went \textbf{into} \textbf{the} \textbf{house}.’
\z

Note that \textit{suro} retains its nominal nature in \REF{ex:dryer:34}, in that its complement \textit{fato} ‘house’ is marked as a possessor, with the genitive postpositional clitic \textit{=be}, and the entire phrase marked with the postpositional clitic \textit{=ro} ‘to’, so that \REF{ex:dryer:34} could be glossed as ‘He went to the inside of the house’. To what extent these locational nouns have grammaticalized as prepositions is not clear. Even if they have not grammaticalized much yet, they illustrate how an OV \& NGen language could acquire prepositions.

In \tabref{extab:dryer:35} is a list of prepositions and postpositions of Kanuri (\citealt{Hutchinson1981}: 257–263).


\begin{table}
\caption{Prepositions and postpositions of Kanuri }
\label{extab:dryer:35}
\fittable{
\begin{tabular}{lll@{}lll} 
\lsptoprule
 & \multicolumn{2}{c}{\bfseries Prepositions} &  & \multicolumn{2}{c}{\bfseries Postpositions}\\
\midrule
 & bótówò & ‘next to’ &  & =(là)n & ‘locative, instrumental’\\
 & cî & ‘at edge of’ &  & =rò & ‘benefactive, indirect object, to’\\
 & dàryé & ‘at the end of’ &  & =mbèn & ‘through, towards’\\
 & dáwù & ‘in middle of’ &  &  & \\
 & fúwù & ‘in front of’ &  &  & \\
 & fərtə & ‘at base of’ &  &  & \\
 & gəré & ‘next to’ &  &  & \\
 & kátè & ‘between’ &  &  & \\
 & kəlâ & ‘on top of’ &  &  & \\
 & ngáwò & ‘behind, after’ &  &  & \\
 & sədíà {\textasciitilde} cídíà & ‘under’ &  &  & \\
 & súró & ‘inside, during’ &  &  & \\
\lspbottomrule
\end{tabular}
}
\end{table}

The meanings associated with the prepositions in Kanuri are similar to those of the prepositions in Iraqw, but are also similar to the meanings of the postpositions in the various SVO \& GenN languages discussed above. Conversely, the meanings associated with the postpositions in Kanuri are similar to those of the postpositions in Iraqw and also similar to the meanings of the prepositions in the various SVO \& GenN languages discussed above

There is another instance of a language with both prepositions and postpositions that provides an interesting variation of the argument in this section, namely English. While English is predominantly a prepositional language, it has at least two postpositions, \textit{ago} and \textit{notwithstanding}, as in \REF{ex:dryer:36}.\footnote{\textit{Notwithstanding} also occurs as a preposition. The postpositional use is apparently the original use. I suspect that the use as a preposition arose due to its semantic similarity to another preposition \textit{despite}.}

\ea\label{ex:dryer:36}
\langinfo{}{}{English}\\
\ea  I saw him three weeks ago.\\
\ex  I went to the concert, the doctor’s advice notwithstanding.\\
\z
\z

What is unusual about these two postpositions in English is that although both are apparently grammaticalizations of verbs, they are ones where what is now the object of that postposition was originally the subject of the verb (rather than the object, the more common situation with grammaticalizations from verbs). According to the Merriam Webster online dictionary,\footnote{\url{https://www.merriam-webster.com/dictionary}} \textit{ago} comes from an obsolete verb meaning ‘pass’ so that \textit{three} \textit{weeks} \textit{ago}  derives from \textit{three} \textit{weeks} \textit{have} \textit{passed}, where \textit{three} \textit{weeks} was originally the subject of this verb. And \textit{notwithstanding} comes from \textit{not} plus a form of the verb meaning ‘withstand’ in the sense of ‘providing an obstacle for’; again, what is now the object of the postposition \textit{notwithstanding} was originally the subject of the verbal expression. The fact that these two words arose as postpositions rather than as prepositions reflects the fact that subjects normally preceded the verb, even in earlier varieties of English when word order was more flexible. Again, only a grammaticalization account explains these.

The evidence in this section involves data that only grammaticalization can explain. An explanation in terms of grammaticalization for the correlation between the order of verb and object and order of adposition and noun phrase as well as the correlation between the order of noun and genitive and order of adposition and noun phrase predicts that we should find both prepositions and postpositions in the same language where the former derive from verbs and the latter from head nouns in genitive constructions, as well as predicting the semantic differences between the two types of adposition. The evidence in this section shows how these predictions are borne out. There is no obvious way in which accounts in terms of processing or similarity could explain this data.

\section{What grammaticalization does not explain}\label{sec:dryer:3}

The preceding section provides evidence that grammaticalization explains, at least partly, the correlation between the order of verb and object and order of adposition and noun phrase as well as the correlation between the order of noun and genitive and order of adposition and noun phrase. In this section, I discuss the question whether grammaticalization fully explains word order correlations and argue that it does not. I first discuss word order correlations for which there does not seem to be any good explanation in terms of grammaticalization. \tabref{tab:dryer:6} provides a list of pairs of elements that are shown by \citet{Dryer1992} to correlate with the order of verb and object, where the verb patterner refers to elements that occur first in these pairs more often among VO languages than among OV languages (and where the object patterner refers to the other member of the pair).

\begin{table}
\begin{tabularx}{\textwidth}{XXl}
\lsptoprule
verb patterner &  object patterner  & example\\
\midrule 
verb &  adpositional phrase  	&\textit{slept} + \textit{on} \textit{the} \textit{floor}   \\
verb & manner adverb 		& \textit{ran} + \textit{slowly}                            \\
copula verb & predicate  	&\textit{is} + \textit{a} \textit{teacher}                  \\
‘want' & VP  			&\textit{wants} + \textit{to} \textit{see} \textit{Mary}    \\
noun & relative clause  	&\textit{movies} + \textit{that} \textit{we} \textit{saw}   \\
adjective & standard of comparison  &\textit{taller} + \textit{than} \textit{Bob}           \\
complementizer &  clause  	&\textit{that} + \textit{John} \textit{is} \textit{sick}    \\
question particle & sentence     & {}                      \\
adverbial subordinator & clause & \textit{because} + \textit{Bob} \textit{has} \textit{left}\\
\lspbottomrule
\end{tabularx}

\caption{\label{tab:dryer:6}Pairs of elements that correlate with the order of verb and object}
\end{table}

For none of these pairs of elements that correlate with the order of verb and object is there a convincing explanation in terms of grammaticalization. For example, the order of verb and adpositional phrase most likely correlates with the order of verb and object because of semantic similarities between these two pairs of elements or because of processing factors. It is hard to imagine an explanation in terms of grammaticalization for this correlation.

I devote the remainder of this section to discussing the correlation between the order of verb and object and the order of noun and genitive. While there have been attempts to explain this correlation in terms of grammaticalization, I claim here that such attempts fall short of providing a plausible explanation. A good summary of this approach is provided by \citetv{Collins2019tv}. However, most of the cases discussed by Collins are highly speculative, especially compared to the evidence for adpositions deriving from verbs or nouns. The arguments involve cases where the constructions now used for main clauses are claimed to have originated from nominalizations (where a construction like \textit{John’s} \textit{seeing} \textit{Peter} is claimed to have replaced an existing finite construction like \textit{John} \textit{saw} \textit{Peter}).\footnote{Some of Collins’ arguments are particularly unconvincing. He cites data from Angas showing nominalizations being used for complements of the verb meaning ‘want’. But this only shows that some languages express such complements using nominalizations; it provides no evidence of nominalizations coming to be used as main clauses. He also cites the large number of Austronesian languages as evidence for the frequency by which nominalizations become main clauses. But quite apart from the fact that Collins provides no evidence to support his claim that it is generally accepted that nominalizations came to be used as main clauses in Austronesian, the size of the family is not relevant; what is relevant is the number of instances of changes of this sort. A number of proposals that main clause constructions originated as nominalizations are based largely on the fact that the same case marker is used for both possessors and subjects (or transitive subjects). But
\label{p:dryer:manyotherways}
there are many ways by which this can arise without nominalizations coming to be used as main clauses.} Assuming that the word order in nominalizations reflects the order of noun and genitive (an assumption that is probably valid), the new construction will employ an order of verb and object that reflects the order of noun and genitive.\footnote{It will also determine the order of verb and subject, especially for intransitive verbs. There are issues arising here that are beyond the scope of this paper. And while I find the evidence that grammaticalization explains the correlation between the order of verb and object and the order of noun and genitive unconvincing, I must concede that it would account for the large number of SVO \& GenN languages. In other words, it would account for the fact that the order of noun and genitive is one of the few orders that correlates not only with the order of verb and object but also with the order of verb and subject \citep{Dryer2013_Six2}.}

While there probably have been some instances in which a nominalization construction came to be used as the primary construction for main clauses, there is little evidence of this in most families and the correlation between the order of verb and object and the order of noun and genitive seems far too strong to be explained purely in this way. Consider the data in \tabref{tab:dryer:7} on the relative frequency of the different orders of noun and genitive in OV languages.

\begin{table}
\begin{tabularx}{\textwidth}{Xlrrrrr}
\lsptoprule
& \bfseries Africa & \bfseries Euras & \bfseries Oceania & \bfseries N.Amer & \bfseries S.Amer & \bfseries TOTAL\\
\midrule
OV \& GenN & [26] & [46] & [87] & [34] & [54] & 247\\
OV \& NGen & 13 & 1 & 10 & 0 & 1 & 25\\
\lspbottomrule
\end{tabularx}
\caption{\label{tab:dryer:7}Order of noun and genitive in OV languages}
\end{table}

\tabref{tab:dryer:7} shows that GenN order outnumbers NGen by 247 to 25 genera, a ratio of almost 10-to-1. The evidence for nominalizations coming to be used as main clauses is far too meagre to account for such a strong correlation.

It should be noted that the order of noun and genitive correlates with the order of verb and object less strongly than the order of adposition and noun phrase correlates with either the order of verb and object or the order of noun and genitive: Tables \ref{tab:dryer:1} and \ref{tab:dryer:2} above show a particularly strong correlation between the order of verb and object and the order adposition and noun phrase; Tables \ref{tab:dryer:3} and \ref{tab:dryer:4} show an even stronger correlation between the order of noun and genitive and the order of adposition and noun phrase. But the large number of SVO \& GenN languages shows that the correlation between the order of verb and object and the order of noun and genitive is less strong.

One possible explanation for why the correlation between the order of verb and object and the order of noun and genitive is weaker is that all three of these correlations are due in part to factors other than grammaticalization (such as the processing explanations of \citealt{Dryer1992} and \citealt{Hawkins1994_Perf,Hawkins2004_Eff,Hawkins2014_VarEff}), but that grammaticalization augments the correlation between the order of verb and object and the order of adposition and noun phrase as well as the correlation between the order of noun and genitive and the order of adposition and noun phrase. In other words, it may be a mistake to try to choose between grammaticalization and other factors in explaining word order correlations; they may conspire to lead to these stronger correlations.

In fact, data presented by \citet{Dryer1992_Greenb,Dryer2013_Six2} suggests that the correlation between the order of verb and object and the order of adposition and noun phrase as well as the correlation between the order of noun and genitive and the order of adposition and noun phrase are stronger than most of the correlations in \tabref{tab:dryer:6} above. Since there do not appear to be promising explanations for those correlations in terms of grammaticalization, the fact that the two correlations involving adpositions are particularly strong suggests again that both grammaticalization and other factors play a role in explaining those correlations.

Note also that grammaticalization explains the fact mentioned above in \sectref{sec:dryer:2.1} that the preference for postpositions among OV languages is stronger than the preference for prepositions among VO languages. Namely, OV languages are overwhelmingly GenN so that both sources for adpositions lead to postpositions in OV languages. In contrast there are many SVO languages with GenN order. In such languages the adpositions derived from head nouns will be postpositions, so that (assuming some such languages lack adpositions derived from verbs) \largerpage we expect to find SVO languages with postpositions.

\section{Order of noun and definiteness marker}\label{sec:dryer:4}

In this section, I discuss a different type of problem for grammaticalization accounts of word order correlations. In the cases discussed in \sectref{sec:dryer:3}, grammaticalization simply fails to predict a word order correlation which can be shown to be real. In the case discussed in this section, grammaticalization makes a prediction that turns out not to hold, involving the order of definiteness marker and noun.

The most common grammaticalization source for markers of definiteness appears to be demonstratives. In fact my database contains 102 instances of languages that use demonstratives as markers of definiteness, compared to 274 languages with markers of definiteness that are distinct from demonstratives. Both the order of definiteness marker and noun and the order of demonstrative and noun exhibit weak correlations with the order of verb and object, but what is surprising from the perspective of grammaticalization is that they exhibit opposite correlations. Namely, definiteness markers \textit{precede} the noun more often in VO languages than in OV languages, while demonstratives \textit{follow} the noun more often in VO languages than in OV languages.

Consider first definiteness markers in VO languages. \tabref{tab:dryer:8} provides data on the order of definiteness marker and noun in VO languages. The last line in \tabref{tab:dryer:8} gives the proportion of the number on the first line as a proportion of the sum of the number on the first line and the number on the second line. For example, the .21 on the third line in \tabref{tab:dryer:8} under Africa represents 8 as a proportion of 39 (the sum of 8 and 31). I use these proportions in the discussion below.

\begin{table}
\begin{tabularx}{\textwidth}{llrrrrr} 
\lsptoprule
& \bfseries Africa & \bfseries Euras & \bfseries Oceania & \bfseries N.Amer & \bfseries S.Amer & \bfseries TOTAL\\
\midrule
VO \& DefN & 8 & [11] & [16] & [17] & [7] & 59\\
VO \& NDef & [31] & 3 & 13 & 8 & 0 & 55\\
Proportion DefN & .21 & .79 & .55 & .68 & 1.00 & $\bar{x}$=.64\\
\lspbottomrule
\end{tabularx}
\caption{\label{tab:dryer:8}Order of noun and definiteness marker in VO languages} 
\end{table}

\tabref{tab:dryer:8} shows the two orders of definiteness marker and noun to be about equally common among VO languages, with DefN order found in languages in 59 genera and NDef order found in languages in 55 genera. This is a case, however, where the total numbers of genera are somewhat misleading, since one area, Africa, exhibits a very different pattern from what we find in the other four areas. In Africa, genera containing VO languages in which the definiteness marker follows the noun outnumber genera containing VO languages in which the definiteness marker precedes the noun by 31 to 8. In the other four areas, in contrast, it is more common among VO languages for the definiteness marker to precede the noun; in fact, in three of the areas (Eurasia, North America, and South America), DefN order is more than twice as common as NDef order. The mean of the proportions over the five areas, namely .64, also reflects a preference for DefN order among VO languages. Another way to see this is that if we exclude Africa, DefN outnumbers NDef among VO languages by 51 to 24.\footnote{The higher preference for NDef order among VO languages in Africa reflects a general difference between Africa and the rest of the world in that postnominal modifiers are more common in Africa than elsewhere \citep{Dryer2010}. \tabref{tab:dryer:7} above shows a similar difference between Africa and the rest of the world: while GenN outnumbers NGen among OV languages overall by almost 10-to-1, the ratio in Africa is only 2-to-1 and over half (13 out of 25) of the genera containing OV \& NGen languages are in Africa.}

\tabref{tab:dryer:9} provides comparable data on the order of definiteness marker and noun among OV languages.  We again find only a small difference, though it is NDef that outnumbers DefN among OV languages, by 53 genera to 38.

\begin{table}
\begin{tabularx}{\textwidth}{llrrrrr} 
\lsptoprule
& \bfseries Africa & \bfseries Euras & \bfseries Oceania & \bfseries N.Amer & \bfseries S.Amer & \bfseries TOTAL\\
\midrule
OV \& DefN & 3 & [9] & 15 & 4 & [7] & 38\\
OV \& NDef & [12] & 5 & [23] & [9] & 4 & 53\\
Proportion DefN & .20 & .64 & .39 & .31 & .64 & $\bar{x}$=.44\\
\lspbottomrule
\end{tabularx} 
\caption{Order of noun and definiteness marker in OV languages}
\label{tab:dryer:9}
\end{table}

But what is revealing is to compare the proportions from the last lines of Tables \ref{tab:dryer:8} and \ref{tab:dryer:9}, given in \tabref{tab:dryer:10}.

\begin{table}
\begin{tabularx}{\textwidth}{Xlrrrrr} 
\lsptoprule
& \bfseries Africa & \bfseries Eurasia & \bfseries Oceania & \bfseries N.America & \bfseries S.America & \bfseries Mean\\
\midrule 
VO & [.21] & [.79] & [.55] & [.68] & [1.00] & .64\\
OV & .20 & .64 & .39 & .31 & .64 & =.44\\
\lspbottomrule
\end{tabularx} 
\caption{\label{tab:dryer:10} Proportion of genera containing DefN languages among VO vs. OV languages}
\end{table}


Here we find that although the margin of difference in Africa is very small, it is still the case that the proportion of genera containing DefN languages is greater among VO languages in all five areas. This gives us reason to conclude that there is a correlation, albeit a weak one, between the order of verb and object and the order of definiteness marker and noun, with the definiteness marker preceding the noun more often among VO languages than among OV languages.

Given the fact that the most common grammaticalization source for definiteness markers appears to be demonstratives, we might expect to find a similar correlation between the order of verb and object and the order of demonstrative and noun. We do find a clear trend, but it is the opposite correlation. Namely while definiteness markers precede the noun more often among VO languages compared to OV languages, demonstratives tend to follow the noun more often among VO languages compared to OV languages.

Tables \ref{tab:dryer:11} to \ref{tab:dryer:13} provide data supporting this. \tabref{tab:dryer:11} provides relevant data for VO languages. It shows that although NDem order is slightly more common than DemN order, by 118 genera to 92, this order is more common in only three of the five areas (and in fact, if we exclude Africa, it is DemN order that is more common among VO languages, by 84 genera to 66)

\begin{table}
\begin{tabularx}{\textwidth}{llrrrrr} 
\lsptoprule
& \bfseries Africa & \bfseries Euras & \bfseries Oceania & \bfseries N.Amer & \bfseries S.Amer & \bfseries TOTAL\\
\midrule 
VO \& DemN & 8 & 12 & 24 & [24] & [24] & 92\\
VO \& NDem & [52] & [16] & [31] & 12 & 7 & 118\\
Proportion DemN & .13 & .43 & .44 & .67 & .77 & $\bar{x}$=.49\\
\lspbottomrule
\end{tabularx}
\caption{\label{tab:dryer:11}Order of noun and demonstrative in VO languages}
\end{table}


However, \tabref{tab:dryer:12} shows that among OV languages, DemN order is about twice as common as NDem order, by 181 genera to 95, although there are two areas where NDem is more common among OV languages.

\begin{table}
\begin{tabularx}{\textwidth}{llrrrrr}
\lsptoprule
& \bfseries Africa & \bfseries Euras & \bfseries Oceania & \bfseries N.Amer & \bfseries S.Amer & \bfseries TOTAL\\
\midrule
OV \& DemN & 16 & [44] & 45 & [30] & [46] & 181\\
OV \& NDem & [18] & 6 & [57] & 6 & 8 & 95\\
Proportion DemN & .47 & .88 & .44 & .83 & .85 & $\bar{x}$=.70\\
\lspbottomrule
\end{tabularx}
\caption{\label{tab:dryer:12}Order of noun and demonstrative in OV languages}
\end{table}

Again, it is useful to compare the proportions from the last lines of Tables \ref{tab:dryer:11} and \ref{tab:dryer:12}, shown in \tabref{tab:dryer:13}.

\begin{table}
\begin{tabularx}{\textwidth}{Xlrrrrr}
\lsptoprule
& \bfseries Africa & \bfseries Eurasia & \bfseries Oceania & \bfseries N.America & \bfseries S.America & \bfseries Mean\\
\midrule
VO & .13 & .43 & .44 & .67 & .77 & .49\\
OV & [.43] & [.88] & .44 & [.83] & [.85] & .70\\
\lspbottomrule
\end{tabularx}
\caption{\label{tab:dryer:13}Proportion of genera containing DemN languages among VO vs. OV languages} 
\end{table}


\tabref{tab:dryer:13} shows that the proportion of genera containing DemN languages is higher among OV languages in four areas while the proportion is the same in the fifth area (Oceania).\footnote{If we compute the proportions to three decimal places, DemN is also higher among OV languages compared to VO languages in Oceania (by .441. to .434). However, this difference is too small to base any conclusion on.} There is thus a clear trend in the opposite direction from what we found for the order of definiteness marker and noun. Given that the most common grammaticalization source for definiteness markers appears to be demonstratives, this contrast is quite surprising.

I have no explanation for the source of this difference between definiteness markers and demonstratives. But I will share some interesting data from particular languages that conforms to this difference. First, there are a few languages in which the same form is used as a demonstrative and as a marker of definiteness, but this form occurs on different sides of the noun, depending on its function. In Swahili, the forms that are used as distal demonstratives when following the noun function as markers of definiteness when they precede the noun, as shown in \REF{ex:dryer:37}. Since Swahili is SVO, this difference conforms to the contrast in the crosslinguistic data shown above.

\ea\label{ex:dryer:37}
\langinfo{Swahili}{Niger-Congo, Bantoid}{\citealt{Ashton1947}: 59}\\
  \ea \gll  m-tu  \textbf{yu-le}\\
	  \textsc{nc}\textsubscript{1}{}-man  \textbf{\textsc{nc}}\textbf{\textsubscript{1}}\textbf{{}-that}\\
  \glt     ‘that man’

  \ex
  \gll    \textbf{yu-le}  m-tu\\
	  \textbf{\textsc{nc}}\textbf{\textsubscript{1}}\textbf{{}-}\textbf{\textsc{def}}  \textsc{nc}\textsubscript{1}{}-man\\
  \glt     ‘the man’
  \z
\z

In Abui, we find the opposite situation: the form \textit{do} functions as a demonstrative when it precedes the noun, as in \REF{ex:dryer:38a}, but as a marker of definiteness when it follows the noun, as in \REF{ex:dryer:38b}.

\ea\label{ex:dryer:38}
\langinfo{Abui}{Timor-Alor-Pantar: Indonesia}{\citealt{Kratochvil2007}: 111, 114}\\
\ea\label{ex:dryer:38a}
\gll     \textbf{do}  sura  \\  
         \textbf{this}  book\\    
\glt     ‘this book (near me)’  
\ex \label{ex:dryer:38b}
\gll kaai  \textbf{do}\\
    dog  \textbf{\textsc{def}}\\
\glt  ‘the dog (I just talked about)’
\z
\z

Significantly, Abui is an OV language, so the fact that Abui exhibits the opposite pattern from what we saw in Swahili again conforms to the crosslinguistic pattern described above.

The situation in Ute is similar to that in Abui. Namely Ute is OV and the word \textit{'u} functions as a demonstrative when it precedes the noun, as in \REF{ex:dryer:39a}, but as a marker of definiteness when it follows the noun, as in \REF{ex:dryer:39b}.

\ea\label{ex:dryer:39}
\langinfo{Ute}{Uto-Aztecan: United States}{\citealt{Givón2011}: 50, 38}\\
\ea\label{ex:dryer:39a}
\gll     \textbf{'ú}  kava  sá-gha-rʉ-mʉ  qhárʉ-kwa-pʉga.\\
         \textbf{that.\textsc{sbj}}  horse.\textsc{sbj}  white-have-\textsc{nmlz-anim.sbj}  run-go-\textsc{rem}\\
\glt     ‘That white horse ran away.’
\ex\label{ex:dryer:39b}
\gll    ta'wa-chi  \textbf{'u}  sivaatu-chi  paqha-qa.\\
         man-\textsc{anim.sbj}  \textbf{\textsc{def.sbj}}  goat-\textsc{anim.obj}  kill-\textsc{ant}\\
\glt     ‘The man killed a goat.’
\z
\z

The situation in Loniu is somewhat different. In Loniu, the definiteness marker and demonstrative are similar in form, though not identical, with \textit{iy} as the definiteness marker and \textit{iyɔ} as the demonstrative. The two in fact can co-occur as in \REF{ex:dryer:40}, with the definiteness marker preceding the noun, and the demonstrative following the noun.

\ea\label{ex:dryer:40}
\langinfo{Loniu}{Austronesian, Oceanic: Papua New Guinea}{\citealt{Hamel1994}: 100}\\
\gll   iy   amat   iyɔ\\
       \textsc{def}   man   this \\
\glt   ‘this man’
\z

Again, since Loniu is VO, this order difference conforms to the crosslinguistic pattern described above.

And we find similar phenomena in cases where the definiteness marker and demonstrative are completely different in form but can co-occur, with one preceding the noun and one following. In Kana, the definiteness marker precedes the noun while the demonstrative follows, as in \REF{ex:dryer:41}.

\ea\label{ex:dryer:41}
\langinfo{Kana}{Niger-Congo, Delta Cross: Nigeria}{\citealt{Ikoro1996}: 70}\\
\gll   ló   bárí   āmā \\
       \textsc{def}   fish   this \\
\glt   ‘this fish’
\z

Since Kana is VO, this conforms to the crosslinguistic pattern. Contrast this with the situation in Kwoma (Washkuk), which is OV, and in this case it is the demonstrative that precedes the noun and the definiteness marker that follows, as in \REF{ex:dryer:42}.

\ea\label{ex:dryer:42}
\langinfo{Kwoma}{Sepik: Papua New Guinea}{\citealt{Kooyers1974}: 49}\\
\gll   kata  ma  rii\\
       that  man  \textsc{def}\\
\glt   ‘that man’
\z

These differences between demonstratives and definiteness markers are a puzzle if demonstratives are the primary grammaticalization source for definiteness markers. It should be emphasized, however, that although definiteness markers and demonstratives exhibit very different patterns in terms of how they correlate with the order of verb and object, it is still the case that they correlate with each other, i.e. that the order of definiteness marker and noun and the order of demonstrative and noun correlate. This is shown in Tables \ref{tab:dryer:14} and \ref{tab:dryer:15}, excluding languages where the definiteness marker is the same as the demonstrative. \tabref{tab:dryer:14} shows that among DefN languages with definiteness markers that are distinct from demonstratives, it is approximately twice as common for the demonstrative to precede the noun as well, by 41 genera to 20.

\begin{table}
\begin{tabularx}{\textwidth}{Xlrrrrr}
\lsptoprule
& \bfseries Africa & \bfseries Euras & \bfseries Oceania & \bfseries N.Amer & \bfseries S.Amer & \bfseries TOTAL\\
\midrule
DefN \& DemN & 3 & [7] & [12] & [11] & [8] & 41\\
DefN \& NDem & [4] & 3 & 7 & 3 & 3 & 20\\
\lspbottomrule
\end{tabularx}
\caption{\label{tab:dryer:14}Order of noun and demonstrative in DefN languages}
\end{table}

Conversely, \tabref{tab:dryer:15} shows that among NDef languages with definiteness markers that are distinct from demonstratives, it is much more common for the demonstrative to follow the noun as well, by 67 genera to 11.

\begin{table}
\begin{tabularx}{\textwidth}{Xlrrrrr}
\lsptoprule
& \bfseries Africa & \bfseries Euras & \bfseries Oceania & \bfseries N.Amer & \bfseries S.Amer & \bfseries TOTAL\\
\midrule
NDef \& DemN & 4 & 3 & 2 & 1 & 1 & 11\\
NDef \& NDem & [33] & [6] & [19] & [8] & 1 & 67\\
\lspbottomrule
\end{tabularx}
\caption{\label{tab:dryer:15}Order of noun and demonstrative in NDef languages}
\end{table}

While grammaticalization probably plays some role in explaining this correlation, it seems likely that the clear semantic similarity between definiteness markers and demonstratives plays a role as well. There is also a correlation between the order of definiteness marker and noun and the order of indefinite marker and noun, a correlation that is presumably due to semantic similarity or processing, not grammaticalization.

\section{Conclusion}\label{sec:dryer:5}

I have argued that there is evidence that any approach to explaining word order correlations that ignores the role of grammaticalization is inadequate. At the same time, I have argued that while grammaticalization plays a role in explaining some correlations, a pure grammaticalization approach fails as well.

Although I have focused my discussion of SVO \& GenN languages on those with both prepositions and postpositions, further research is needed on SVO \& GenN languages with prepositions as the only or dominant type or with postpositions as the only or dominant type. Grammaticalization theory would predict that SVO \& GenN languages with prepositions will be ones where the primary source of adpositions is verbs, while SVO \& GenN languages with postpositions will be ones where the primary source of adpositions is head nouns in genitive constructions. I suspect that this is true and if so, it would further bolster the argument that grammaticalization plays an important role in explaining correlations involving adpositions. One reason to suspect it is true is the geographical distribution of the two types of languages. My database includes 21 genera containing SVO \& GenN languages with prepositions and 13 of these genera (almost two thirds of them) are in an area stretching from China and Southeast Asia through Austronesian. The fact that so many of the SVO \& GenN languages are in this region is significant since my impression is that the grammaticalization of adpositions from verbs is especially common in this region. Conversely, my database includes 19 genera containing SVO \& GenN languages with postpositions and only two of these genera are in the region mentioned above stretching from China through Austronesian where SVO \& GenN \& Pr languages are common. I suspect that this is because outside that region, it is more common for adpositions to grammaticalize from nouns. However, this is a matter for future research.

\section*{Abbreviations}

The paper abides by the Leipzig Glossing Rules. Additional abbreviations include the following ones:


\begin{tabularx}{.45\textwidth}{lQ}
\textsc{anim}  &animate \\
\textsc{ant}  &anterior\\
\textsc{constr} & construct state\\
\end{tabularx}
\begin{tabularx}{.45\textwidth}{lQ}
\textsc{expec} & expectational\\
\textsc{nc} & noun class\\
\textsc{rem}  &remote\\
\end{tabularx}



\section*{Acknowledgements}

I am indebted to Lea Brown, Karsten Schmidtke-Bode and members of the audience at the 2015 meeting of the DGfS (the German Linguistic Society) for comments on an earlier version of this paper. I also acknowledge funding from The Social Sciences and Humanities Research Council of Canada, the National Science Foundation (in the United States), the Max Planck Institute for Evolutionary Anthropology (in Leipzig, Germany) and the Humboldt Foundation (in Germany).


\sloppy
\printbibliography[heading=subbibliography,notkeyword=this] 
\end{document}
\documentclass[output=paper]{langsci/langscibook} 
\ChapterDOI{10.5281/zenodo.2583810}
\author{Matthew S. Dryer\affiliation{University at Buffalo}}
\title{Grammaticalization accounts of word order correlations}
\abstract{This paper examines the role that grammaticalization plays in explaining word order correlations. It presents some data that only grammaticalization accounts for, but also argues that there are correlations that grammaticalization does not account for. The conclusion is that accounts entirely in terms of grammaticalization or accounts that make no reference to grammaticalization are both inadequate.}
% %% hyphenation points for line breaks
%% Normally, automatic hyphenation in LaTeX is very good
%% If a word is mis-hyphenated, add it to this file
%%
%% add information to TeX file before \begin{document} with:
%% %% hyphenation points for line breaks
%% Normally, automatic hyphenation in LaTeX is very good
%% If a word is mis-hyphenated, add it to this file
%%
%% add information to TeX file before \begin{document} with:
%% %% hyphenation points for line breaks
%% Normally, automatic hyphenation in LaTeX is very good
%% If a word is mis-hyphenated, add it to this file
%%
%% add information to TeX file before \begin{document} with:
%% \include{localhyphenation}
\hyphenation{
affri-ca-te
affri-ca-tes
com-ple-ments
Haw-kins
broad-er
over-view
par-tic-u-lar
spe-cif-ic
adap-tive
lin-guis-ti-cal-ly
Hix-kar-yana
sour-ces
mark-er
ground-ed
evol-ved
Born-kessel-Schlesew-sky
}
\hyphenation{
affri-ca-te
affri-ca-tes
com-ple-ments
Haw-kins
broad-er
over-view
par-tic-u-lar
spe-cif-ic
adap-tive
lin-guis-ti-cal-ly
Hix-kar-yana
sour-ces
mark-er
ground-ed
evol-ved
Born-kessel-Schlesew-sky
}
\hyphenation{
affri-ca-te
affri-ca-tes
com-ple-ments
Haw-kins
broad-er
over-view
par-tic-u-lar
spe-cif-ic
adap-tive
lin-guis-ti-cal-ly
Hix-kar-yana
sour-ces
mark-er
ground-ed
evol-ved
Born-kessel-Schlesew-sky
}
\begin{document}
\maketitle 

\section{Introduction}

There is extensive literature both on identifying word order correlations\is{word-order correlation} (\citealt{Greenberg1963,Hawkins1983,Dryer1992}) and on possible explanations for these correlations. Proposed explanations can be grouped loosely into three types. First, it is proposed that some correlations exist because of some sort of similarity or shared property of the pairs that correlate. An example of this is the hypothesis that the order of \isi{object} and verb correlates with the order of \isi{adposition} and noun phrase because both involve a pair of head and dependent. A second type of explanation is in terms of sentence \isi{processing} (\citealt{Kuno1974,Dryer1992,Dryer2009_Corr,Hawkins1994_Perf,Hawkins2004_Eff,Hawkins2014_VarEff}), under which the types that do not conform to the correlations are less frequent because structures containing the two inconsistent types are more difficult to parse\is{processing}. This would be what \citetv{Haspelmath2019tv} calls a functional-adaptive\is{adaptation} type of explanation. A third line of explanation is in terms of \isi{grammaticalization} (\citealt{Givón1979,HeineReh1984}: 241–244; \citealt{Bybee1988,Aristar1991,DeLancey1994,Collins2012}, \citealtv{Collins2019tv}). For example it is hypothesized that the reason (or a reason) why the order of \isi{adposition} and noun phrase correlates with the order of verb and \isi{object} is that one \isi{grammaticalization} source for \isi{adposition}s is verbs and the order of verb and \isi{object} remains the same when the verb grammaticalizes as an \isi{adposition}. This line of explanation is thus crucially based on the diachronic sources of \isi{adposition}s and hence a type of source-based explanation (\citealtv{Cristofaro2019tv}).

Despite these competing hypotheses for explaining word order correlations\is{word-order correlation}, there is surprisingly little attempt by proponents of an explanation in terms of \isi{grammaticalization} to argue against other approaches or by proponents of other approaches to argue against \isi{grammaticalization}. In fact, proponents of other approaches rarely even mention the possible role of \isi{grammaticalization}. The goal of this paper is to argue that both explanations in terms of \isi{grammaticalization} and explanations in terms of shared features or \isi{processing} are needed in explaining word order correlations\is{word-order correlation}. I will \isi{focus} on the pros and cons of \isi{grammaticalization} explanations, largely ignoring the difference between accounts in terms of shared features and accounts in terms of sentence \isi{processing}.\footnote{In \citet{Dryer1992}, I argue for a \isi{processing} account over an account in terms of heads and dependents for the various correlations between pairs of elements and the order of verb and \isi{object}.}

In \sectref{sec:dryer:2}, I discuss explanations for correlations\is{word-order correlation} involving order of \isi{adposition} and noun phrase and discuss evidence that only a \isi{grammaticalization} approach can account for. Namely I examine SVO  \& GenN languages that have both \isi{preposition}s and \isi{postposition}s and show that not only does an approach involving \isi{grammaticalization} predict languages with both \isi{preposition}s and \isi{postposition}s but it also correctly predicts the semantics associated with each type of \isi{adposition}. 
In \sectref{sec:dryer:3}, I give reasons why \isi{grammaticalization} cannot account for all word order correlations\is{word-order correlation}, concluding that \isi{grammaticalization} and other factors conspire to account for some correlations. 
And in \sectref{sec:dryer:4}, I discuss data involving \isi{word order} properties of \isi{definiteness} markers where \isi{grammaticalization} seems to make the wrong prediction.

\section{A grammaticalization account of the correlations with the order of adposition and noun phrase} \label{sec:dryer:2}

In this section, I present evidence for a \isi{grammaticalization} account for correlations\is{word-order correlation} involving the order of \isi{adposition} and noun phrase that only \isi{grammaticalization} can account for. In \sectref{sec:dryer:2.1}, I discuss evidence of \isi{adposition}s grammaticalizing from verbs. In \sectref{sec:dryer:2.2}, I discuss evidence of a second \isi{grammaticalization} source for \isi{adposition}s, namely head nouns in \isi{genitive} constructions. The next section, \sectref{sec:dryer:2.3}, is, in my view, the most important section of this paper. In that section, I discuss SVO languages which employ GenN \isi{word order}. Grammaticalization theory predicts that in such languages, if \isi{adposition}s arise from both \isi{grammaticalization} sources discussed in \sectref{sec:dryer:2.1} and \sectref{sec:dryer:2.2}, the language will have both \isi{preposition}s and \isi{postposition}s, the former arising from verbs, the latter from head nouns in \isi{genitive} constructions, with particular semantics associated with each. I present evidence from a number of languages showing that this prediction is borne out.

\subsection{Adpositions that grammaticalize from verbs}\label{sec:dryer:2.1}

Let me turn now to one of the best-known word order correlations\is{word-order correlation}, between the order of \isi{object} and verb and the order of \isi{adposition} and noun phrase, where VO languages tend to have \isi{preposition}s while OV languages tend to have \isi{postposition}s (\citealt{Greenberg1963,Dryer1992}). Evidence for this correlation is given in Tables \ref{tab:dryer:1} and \ref{tab:dryer:2}. The data for VO languages is given in \tabref{tab:dryer:1}. The numbers in \tabref{tab:dryer:1} denote numbers of genera containing languages of the given sort, divided into five large continental areas \citep{Dryer1989}. The more frequent type in each area is enclosed in square brackets.

\begin{table}
\begin{tabularx}{\textwidth}{Xrrrrrr} 
\lsptoprule
& \bfseries Africa & \bfseries Eurasia & \bfseries Oceania & \bfseries N.America & \bfseries S.America & \bfseries TOTAL\\
\midrule
VO \& Po & 10{\db} & 6{\db} & 3{\db} & 5{\db} & 15 & 39\\
VO \& Pr & [37] & [25] & [48] & [27] & 15 & 152\\
\lspbottomrule
\end{tabularx}
\caption{\label{tab:dryer:1}Order of adposition and noun phrase in VO languages}
\end{table}



\tabref{tab:dryer:1} shows that \isi{preposition}s outnumber \isi{postposition}s in VO languages by a wide margin in four of the five areas, with the fifth area (South America) having an equal number of genera containing languages with \isi{preposition}s and those containing languages with \isi{postposition}s. Overall, VO \& Pr outnumbers VO \& Po by 152 to 39 genera.

The corresponding data for OV languages is given in \tabref{tab:dryer:2}.
\tabref{tab:dryer:2} shows a stronger preference for \isi{postposition}s in OV languages than the preference for \isi{preposition}s in VO languages shown in \tabref{tab:dryer:1}, in that \tabref{tab:dryer:2} shows only 11 genera containing OV \& Pr languages while \tabref{tab:dryer:1} shows 39 genera containing VO \& Po languages. I discuss an explanation for this difference in terms of \isi{grammaticalization} in \sectref{sec:dryer:3} below.

\begin{table}
\begin{tabularx}{\textwidth}{Xrrrrrr} 
\lsptoprule
& \bfseries Africa & \bfseries Eurasia & \bfseries Oceania & \bfseries N.America & \bfseries S.America & \bfseries TOTAL\\
\midrule
OV \& Po & [25] & [45] & [82] & [28] & [41] & 221\\
OV \& Pr & 3{\db} & 2{\db} & 5{\db} & 0{\db} & 1{\db} & 11\\
\lspbottomrule
\end{tabularx}
\caption{\label{tab:dryer:2}Order of adposition and noun phrase in OV languages}
\end{table}

\newpage An explanation of this correlation\is{word-order correlation} in terms of \isi{grammaticalization} appeals to the fact that verbs are a common source for \isi{adposition}s, so that when a verb is grammaticalized as an \isi{adposition}, the order with the verb followed by \isi{object} is retained as \isi{preposition} followed by noun phrase, while the order with the \isi{object} followed by verb is retained as noun phrase plus \isi{postposition}.

Two examples of this process of \isi{grammaticalization} in \ili{English} are the \isi{preposition}s \textit{including} and \textit{concerning}, as in \REF{ex:dryer:1}.

\ea\label{ex:dryer:1}
{\ili{English}}\\
\ea  Four men, including John, arrived.\\
\ex  I will talk to you later concerning your thesis.\\
\z
\z

\noindent Both of these \isi{preposition}s retain the present \isi{participle} form ending in -\textit{ing}, coming from the verbs \textit{include} and \textit{concern}. 

Grammaticalization of \isi{adposition}s from verbs is common in many other languages and widely described in the literature. The examples in \REF{ex:dryer:2} to \REF{ex:dryer:7} illustrate apparent examples of \isi{grammaticalization} of particular semantic types of \isi{adposition}s from verbs with particular meanings.

\paragraph*{give → for}

\ea\label{ex:dryer:2}
\langinfo{\ili{Efik}}{\ili{Niger-Congo}, Delta Cross: Nigeria}{\citealt{Givón2001}: 163}\\
\gll   nam  utom  emi  \textbf{ni}  mi.\\
       do  work  this  \textbf{give}  me\\
\glt   ‘Do this work \textbf{for} me.’
\z

\paragraph*{give → to (marking addressee)}

\ea\label{ex:dryer:3}
\langinfo{\ili{Yoruba}}{\ili{Niger-Congo}, Defoid: Nigeria}{\citealt{Givón2001}: 163}\\
\gll   mo  so̧ \textbf{fún}  o̧.\\
       I  said \textbf{give}  you\\
\glt   ‘I said \textbf{to} you.’
\z

\paragraph*{go → to (marking goal of motion)}

\ea\label{ex:dryer:4}
\langinfo{\ili{Nupe}}{\ili{Niger-Congo}, Ebira-Nupoid: Nigeria}{\citealt{Givón1979}: 221}\\
\gll   ū  bīcī  \textbf{lō}  dzūká.\\
       he  ran  \textbf{go}  market\\
\glt   ‘He ran \textbf{to} the market.’
\z

\paragraph*{follow → with (\isi{comitative})}

\ea\label{ex:dryer:5}
\langinfo{Mandarin Chinese\il{Chinese (Mandarin)}}{\ili{Sino-Tibetan}, Sinitic: China}{\citealt{LiThompson1981}: 423}\\
\gll   tā  bu  \textbf{gēn}  {wǒ}  jiǎng-hua.\\
       3\textsc{sg}  \textsc{neg}  \textbf{follow}  \textsc{1sg}  speak-speech\\
\glt   ‘He doesn't talk \textbf{with} me.’
\z

\paragraph*{take → \isi{object} \isi{case} marker}

\ea\label{ex:dryer:6}
\langinfo{\ili{Yatye}}{\ili{Niger-Congo}, Idomoid: Nigeria}{\citealt{Givón2001}: 163}\\
\gll     ìywi  \textbf{awá}  \textbf{utsì}  ikù.\\
         boy  \textbf{took} \textbf{door}  shut\\
\glt     ‘The boy shut \textbf{the} \textbf{door}.’
\z

\paragraph*{be at → at}

\ea\label{ex:dryer:7}
\langinfo{Mandarin Chinese \il{Chinese (Mandarin)}}{\ili{Sino-Tibetan}, Sinitic: China}{Yu Li, p.c.}\\
\gll     tā  \textbf{zài}  {guō-li}  chǎo  fàn.\\
         3\textsc{sg}  \textbf{be.at}  pot-in  fry  rice\\
\glt     ‘He is frying rice \textbf{in} the pot.’
\z

\noindent The \isi{grammaticalization} of \isi{adposition}s from verbs provides a possible basis for an explanation of the correlation between the order of verb and \isi{object} and the order of \isi{adposition} and noun phrase.

\subsection{Adpositions that grammaticalize from head nouns in nominal possessive constructions}\label{sec:dryer:2.2}

Another common \isi{grammaticalization} source for \isi{adposition}s (and probably the more common source) is head nouns in \isi{genitive} constructions\is{possessive construction}. \ili{English} has a number of \isi{preposition}s that have arisen from head nouns in \isi{genitive} constructions, including those in \REF{ex:dryer:8}.
\newpage
\ea\label{ex:dryer:8}
{\ili{English}}\\
\ea in the side of NP → inside NP\\
\ex by the side of NP → beside NP\\
\ex by the cause of NP → because of NP\\
\z
\z

\noindent Because these \isi{adposition}s arose from head nouns in a \isi{genitive} construction with NGen order, they ended up as \isi{preposition}s rather than \isi{postposition}s. The opposite situation arose in \ili{Amharic}, where the examples in \REF{ex:dryer:9} illustrate two \isi{postposition}s arising from head nouns in a GenN construction.

\ea\label{ex:dryer:9}
\langinfo{\ili{Amharic}}{\ili{Afro-Asiatic}, \ili{Semitic}: Ethiopia}{\citealt{Givón1971}: 399} \\
\ea
NP + bottom → NP + under\\

\gll   kä-bet    \textbf{tač}    allä.\\
       at-house  \textbf{bottom}  is\\
\glt   ‘He is \textbf{under} the house.’

\ex  NP + reason → NP + because of\\

\gll   bä-ɨssu  \textbf{mɨknɨyat}  näw.\\
       at-he    \textbf{reason}  is\\
\glt   ‘It is \textbf{because} \textbf{of} him.’
\z
\z

This type of \isi{grammaticalization} of \isi{adposition}s from head nouns in \isi{genitive} constructions\is{possessive construction} would explain the correlation\is{word-order correlation} between the order of noun and \isi{genitive} and the order of \isi{adposition} and noun phrase. The data in Tables \ref{tab:dryer:3} and \ref{tab:dryer:4} provides evidence for this correlation. The data in \tabref{tab:dryer:3} shows GenN languages being overwhelmingly \isi{postposition}al, while the data in \tabref{tab:dryer:4} shows NGen languages being overwhelmingly \isi{preposition}al.

\begin{table}
\begin{tabularx}{\textwidth}{Xrrrrrr}
\lsptoprule
& \bfseries Africa & \bfseries Euras & \bfseries Oceania & \bfseries N.Amer & \bfseries S.Amer & \bfseries TOTAL\\
\midrule 
GenN \& Po & [31] & [50] & [73] & [33] & [54] & 241\\
GenN \& Pr & 2{\db} & 7{\db} & 13{\db} & 3{\db} & 4{\db} & 29\\
\lspbottomrule
\end{tabularx}
\caption{\label{tab:dryer:3}Order of adposition and noun phrase in GenN languages}
\end{table}


\begin{table}
\begin{tabularx}{\textwidth}{Xrrrrrr}
\lsptoprule
& \bfseries Africa & \bfseries Euras & \bfseries Oceania & \bfseries N.Amer & \bfseries S.Amer & \bfseries TOTAL\\
\midrule 
NGen \& Po & 7{\db} & 0{\db} & 6{\db} & 0{\db} & 1{\db} & 14\\
NGen \& Pr & [34] & [19] & [29] & [19] & [10] & 111\\
\lspbottomrule
\end{tabularx}
\caption{\label{tab:dryer:4}Order of adposition and noun phrase in NGen languages}
\end{table}


\subsection{An interesting prediction of grammaticalization accounts for adpositions}\label{sec:dryer:2.3}


Sections \sectref{sec:dryer:2.1} and \sectref{sec:dryer:2.2} illustrate two \isi{grammaticalization} sources for \isi{adposition}s, one from verbs, the other from head nouns in \isi{genitive} constructions\is{possessive construction}. In languages which are VO and NGen, both sources will lead to \isi{preposition}s rather than \isi{postposition}s. Conversely, in languages which are OV and GenN, both sources will lead to \isi{postposition}s. But there are many languages which are VO but GenN. \citet{Dryer1997_Six,Dryer2013_Six2} shows that although OV languages tend to be GenN and verb-initial languages tend to be NGen, both orders of noun and \isi{genitive} are common among SVO languages, as shown in \tabref{tab:dryer:5}.

\begin{table}
\begin{tabularx}{\textwidth}{Xrrrrrr} 
\lsptoprule
& \bfseries Africa & \bfseries Euras & \bfseries Oceania & \bfseries N.Amer & \bfseries S.Amer & \bfseries TOTAL\\
\midrule
SVO \& GenN & 11{\db} & 11{\db} & 13 & 2{\db} & [11] & 48\\
SVO \& NGen & [41] & [16] & 13 & [5] & 3{\db} & 78\\
\lspbottomrule
\end{tabularx}
\caption{\label{tab:dryer:5}Order of genitive and noun in SVO languages}
\end{table}


\tabref{tab:dryer:5} shows that NGen is more common overall than GenN among SVO languages by 78 genera to 48. However, the higher number of genera containing SVO \& NGen languages turns out to be entirely due to languages in Africa. Outside Africa, SVO \& NGen and SVO \& GenN are both found in exactly 37 genera. The general conclusion is that there is no evidence of any preference for NGen order over GenN order among SVO languages.

Because of the two \isi{grammaticalization} sources for \isi{adposition}s described in the two preceding sections, \isi{grammaticalization} theory makes an interesting prediction about SVO \& GenN languages. Namely, if \isi{adposition}s arise in any such languages from both \isi{grammaticalization} sources, the language should have both \isi{preposition}s and \isi{postposition}s, those arising from verbs being \isi{preposition}s and those arising from nouns being \isi{postposition}s. The evidence in this section shows \largerpage that this prediction is borne out.

In fact, \isi{grammaticalization} theory makes more specific predictions about what meanings will be associated with \isi{preposition}s and what meanings will be associated with \isi{postposition}s in such languages. The lefthand column in \tabref{extab:dryer:10} summarizes typical meanings associated with \isi{adposition}s that arise from verbs, while the righthand column summarizes typical meanings associated with \isi{adposition}s that arise from head nouns in \isi{genitive} constructions.

\begin{table}
\caption{Typical meanings associated with adpositions} 
\label{extab:dryer:10}
\begin{tabularx}{\textwidth}{lXlQ}
\lsptoprule
\bfseries & \bfseries Typical meanings associated with \isi{adposition}s that come from verbs & \bfseries & \bfseries Typical meanings associated with \isi{adposition}s that come from nouns\\
\midrule 
 & {\isi{benefactive} (‘for’)}

{\isi{instrumental} (‘with’)}

{\isi{comitative} (‘with’)}

{\isi{similative} (‘like’)}

{\isi{allative} (‘to, toward’)}

{general locations\is{locative} (‘at’)}

\isi{adposition}s marking direct \isi{object}s &  & {specific locations like}

{‘under’, ‘behind’, ‘in front of’}

‘because of’\\
\lspbottomrule
\end{tabularx}
\end{table}

Note that it is typically \isi{adposition}s denoting specific locations\is{locative} that arise from nouns. \isi{adposition}s denoting general locations (meaning something like ‘at’) often arise from verbs. Similarly \isi{adposition}s associated with motion away from source\is{ablative} or towards\is{allative} a location also more often arise from verbs. Grammaticalization theory predicts that in an SVO \& GenN language with both \isi{preposition}s and \isi{postposition}s, the \isi{\isi{preposition}}s will tend to have meanings like those in the lefthand column in \tabref{extab:dryer:10}, while the \isi{postposition}s will tend to have meanings like those in the righthand column. This section shows that these predictions are also borne out.

The first language illustrating how these predictions are borne out is \ili{Nǀuuki}. The SVO order of Nǀuuki is illustrated in \REF{ex:dryer:11}, GenN order in \REF{ex:dryer:12}.

\ea\label{ex:dryer:11}
\langinfo{\ili{Nǀuuki}}{Tuu: South Africa}{\citealt{CollinsNamaseb2011}: 10}\\
\gll ǂharuxu    ke  ãi  ʘoe.\\
       Haruxu  \textsc{decl}  eat  meat \\
\glt ‘Haruxu is eating meat.’  
\z

\ea\label{ex:dryer:12}
\langinfo{Nǀuuki}{Tuu: South Africa}{\citealt{CollinsNamaseb2011}: 37}\\
\gll siso  ŋǂona\\
       Siso  knife\\
\glt   ‘Siso’s knife’
\z

\noindent \ili{Nǀuuki} has both \isi{preposition}s and \isi{postposition}s. Examples illustrating a \isi{preposition} \textit{ŋ}ǀ\textit{a} are given in \REF{ex:dryer:13} and \REF{ex:dryer:14}, \REF{ex:dryer:13} illustrating an \isi{instrumental} use, \REF{ex:dryer:14} a \isi{comitative} use.

\ea\label{ex:dryer:13}
\langinfo{\ili{Nǀuuki}}{Tuu: South Africa}{\citealt{CollinsNamaseb2011}: 25}\\
\gll n-a  si  ǀaa  ʘoe  \textbf{ŋǀa}  \textbf{ŋǂona}.\\
       1\textsc{sg-decl}  \textsc{irr}  cut  meat  \textbf{with}  \textbf{knife}  \\
\glt ‘I will cut the meat \textbf{with} \textbf{a} \textbf{knife}.’  
\z

\ea\label{ex:dryer:14}
\langinfo{\ili{Nǀuuki}}{Tuu: South Africa}{\citealt{CollinsNamaseb2011}: 25}\\
\gll ǀaǀaʕe  ke  sĩĩsen  \textbf{ŋǀa} \textbf{ŋǀaŋgusi}.\\
       ǀaǀaʕe  \textsc{decl}  work  \textbf{with}  \textbf{Nǀaŋgusi}  \\
\glt ‘ǀaǀa\textsuperscript{ʕ}e works \textbf{with} \textbf{Nǀaŋgusi}.’  
\z

\noindent In contrast, example \REF{ex:dryer:15} illustrates a \isi{postposition} \textit{xuu} ‘in front of’.

\ea\label{ex:dryer:15}
\langinfo{\ili{Nǀuuki}}{Tuu: South Africa}{\citealt{CollinsNamaseb2011}: 80}\\
\gll ǀx’esi  ǀʔaa  s\~ui  \textbf{ǀoβa}  \textbf{xuu}\\
       necklace  go  sit.down  \textbf{child}  \textbf{front}  \\
\glt ‘The necklace fell \textbf{in} \textbf{front} \textbf{of} \textbf{the} \textbf{child}.’ 
\z

The \isi{preposition}s and \isi{postposition}s in \ili{Nǀuuki} (\citealt{CollinsNamaseb2011}: 24–25) are listed in \tabref{extab:dryer:16}.


\begin{table}
\caption{Prepositions and postpositions in Nǀuuki }
\label{extab:dryer:16}
\begin{tabularx}{\textwidth}{lll lll} 
\lsptoprule
 & \multicolumn{2}{c}{\bfseries Prepositions\is{preposition}} &  & \multicolumn{2}{c}{\bfseries Postpositions\is{postposition}}\\
\midrule
 & ŋǀa & ‘\isi{instrumental}, \isi{comitative}’ &  & ǀǀãʔẽ & ‘in’\\
 & ǀǀa & ‘like’ &  & xuu & ‘in front of’\\
 & ŋ & ‘linker’ &  & tsʔĩi & ‘behind’\\
 &  &  &  & ǀqhaa & ‘next to’\\
\lspbottomrule
\end{tabularx}
\end{table}

\noindent Two of the three \isi{preposition}s, \textit{ŋ}ǀ\textit{a} ‘\isi{instrumental}, \isi{comitative}’ and ǀǀ\textit{a} ‘like’, conform to the semantic types of \isi{adposition}s arising from verbs and the fact that they are \isi{preposition}s rather than \isi{postposition}s can be explained if they have arisen from verbs in a VO language. And all four of the \isi{postposition}s represent specific locations, conforming to what we expect semantically of \isi{adposition}s arising from head nouns in \isi{genitive} constructions; the fact that they are \isi{postposition}s rather than \isi{preposition}s can be explained in that they have arisen from head nouns in a \isi{genitive} construction in a GenN language.

A second example is provided by \ili{Logba}. Like \ili{Nǀuuki}, Logba is SVO, as illustrated in \REF{ex:dryer:17}, and GenN, as in \REF{ex:dryer:18}.

\ea\label{ex:dryer:17}
\langinfo{\ili{Logba}}{\ili{Niger-Congo}, Kwa: Ghana}{\citealt{Dorvlo2008}: 105}\\
\gll Setor  ó-kpe  i-gbeɖi=é.\\
       Setor  \textsc{sg}{}-peel  \textsc{nc}{}-cassava=\textsc{det}\\
\glt   ‘Setor peeled the cassava.’ 
\z

\ea\label{ex:dryer:18}
\langinfo{\ili{Logba}}{\ili{Niger-Congo}, Kwa: Ghana}{\citealt{Dorvlo2008}: 71}\\
\gll Kɔdzo    a-klɔ=a\\
       Kɔdzo    \textsc{nc}{}-goat=\textsc{det}\\
\glt   ‘Kɔdzo’s goat’ 
\z

\noindent Also like \ili{Nǀuuki}, \ili{Logba} has both \isi{preposition}s and \isi{postposition}s. The \isi{preposition} \textit{kpɛ} with \isi{instrumental} or \isi{comitative} meaning is illustrated in \REF{ex:dryer:19}.

\ea\label{ex:dryer:19}
\langinfo{\ili{Logba}}{\ili{Niger-Congo}, Kwa: Ghana}{\citealt{Dorvlo2008}: 96}\\
\gll Udzi=é  ó-glɛ  uzugbo  \textbf{kpɛ}  \textbf{a-futa}.\\
       woman=\textsc{det}  \textsc{sg}{}-tie  head  \textbf{with}  \textbf{\textsc{nc}}\textbf{{}-cloth}  \\
\glt ‘The lady tied her head \textbf{with} \textbf{a} \textbf{cloth}.’  
\z

\noindent In contrast, an example illustrating a \isi{postposition} \textit{etsi} ‘under’ is given in \REF{ex:dryer:20}.

\ea\label{ex:dryer:20}
\langinfo{\ili{Logba}}{\ili{Niger-Congo}, Kwa: Ghana}{\citealt{Dorvlo2008}: 98}\\
\gll i-datɔ=a  í-tsi  a-fúta=á  etsi.\\
       \textsc{nc}{}-spoon=\textsc{det}  \textsc{sg}{}-be.in  \textbf{\textsc{nc-}}\textbf{cloth}=\textbf{\textsc{det}}  \textbf{under}\\
\glt   ‘The spoon is \textbf{under} \textbf{the} \textbf{cloth}.’ 
\z

In \tabref{extab:dryer:21} is a list of the \isi{preposition}s and \isi{postposition}s of \ili{Logba} (\citealt{Dorvlo2008}: 95, 98). While one of the \isi{preposition}s has a meaning more commonly associated with \isi{adposition}s that arise from nouns (\textit{na} ‘on’), the other \isi{preposition}s all have meanings that \isi{grammaticalization} theory predicts for \isi{adposition}s arising from verbs and all the \isi{postposition}s have meanings involving specific locations, the types of meanings that \isi{grammaticalization} predicts for \isi{adposition}s that arise from nouns.

\begin{table}
\caption{Prepositions and postpositions of Logba }
\label{extab:dryer:21}
\begin{tabularx}{\textwidth}{@{}l@{}ll lll}
\lsptoprule
 & \multicolumn{2}{c}{\bfseries Prepositions\is{preposition}} &  & \multicolumn{2}{c}{\bfseries Postpositions\is{postposition}}\\
 \midrule 
 & fɛ & ‘at’ &  & nu & ‘inside’\\
 & na & ‘on’ &  & etsi & ‘under’\\
 & kpɛ & ‘\isi{instrumental}, \isi{comitative}’ &  & tsú & ‘on’\\
 & gu & ‘about’ &  & ité & ‘in front of’\\
 & dzígu & ‘from’ &  & zugbó & ‘on’\\
 &  &  &  & yó & {‘surface \isi{contact}’}\newline (e.g. on a wall)\\
 &  &  &  & anú & ‘at tip of, at edge of’\\
 &  &  &  & otsoe & ‘on the side of’\\
 &  &  &  & amá & ‘behind’\\
\lspbottomrule
\end{tabularx}
\end{table}

%should the following paragraph remain deleted?
% % While one of the prepositions has a meaning more commonly associated with adpositions that arise from nouns (\textit{na} ‘on’), the other prepositions all have meanings that \isi{grammaticalization} theory predicts for adpositions arising from verbs and all the postpositions have meanings involving specific locations, the types of meanings that \isi{grammaticalization} predicts for adpositions that arise from nouns.


The third SVO \& GenN language with both \isi{preposition}s and \isi{postposition}s is \ili{Eastern Kayah Li}, a Karenic language in the \ili{Sino-Tibetan} family spoken in Myanmar and Thailand. The \isi{preposition}s and \isi{postposition}s of Eastern Kayah Li are listed in \tabref{extab:dryer:22} (\citealt{Solnit1997}: 209–214). Apart from three \isi{preposition}s with unusual meanings (‘as much as’, ‘as big as’, ‘as long as’), the rest of the \isi{preposition}s and all of the \isi{postposition}s have meanings conforming to the semantics typically associated with \isi{adposition}s arising from verbs and \isi{adposition}s arising from head nouns in \isi{genitive} constructions\is{possessive construction} respectively.

\begin{table}
\caption{Prepositions and postpositions of Eastern Kayah Li}
\label{extab:dryer:22}
\begin{tabularx}{\textwidth}{lll@{}Xll}
\lsptoprule
  & \multicolumn{2}{c}{\bfseries Prepositions\is{preposition}} &  & \multicolumn{2}{c}{\bfseries Postpositions\is{postposition}}\\
\midrule
 & dɤ & ‘at’ &  & kū & ‘inside’\\
 & mú & ‘at’ &  & klɔ & ‘outside’\\
 & bɤ & ‘at’ &  & khu & ‘on, above’\\
 & bá & ‘as much as’ &  & kɛ {\textasciitilde} kɛdē & ‘down inside’\\
 & tí & ‘as big as’ &  & khʌ & ‘at apex of’\\
 & tɤ {\textasciitilde} thɤ & ‘as long as’ &  & lē & ‘under, downhill from’\\
 & phú {\textasciitilde} hú & ‘like’ &  & chá & ‘near’\\
 &  &  &  & ŋē {\textasciitilde} béseŋē & ‘in front of’\\
 &  &  &  & khjā {\textasciitilde} békhjā & ‘behind’\\
 &  &  &  & lo & ‘on non-horizontal surface’\\
 &  &  &  & klē & ‘in (an area)’\\
 &  &  &  & rɔklē & ‘beside’\\
 &  &  &  & ple {\textasciitilde} ple kū & ‘in narrow space between’\\
 &  &  &  & cɔkū & ‘in middle of, between’\\
 &  &  &  & thɯ & ‘on edge of’\\
 &  &  &  & təkjā & ‘in the direction of’\\
\lspbottomrule
\end{tabularx}
\end{table}

The fourth SVO \& GenN language exhibiting a similar pattern is \ili{Jabem}, an \ili{Oceanic} language in the \ili{Austronesian} family spoken in Papua New Guinea. In \tabref{extab:dryer:23} is a list of the \isi{preposition}s and \isi{postposition}s of \ili{Jabem} (\citealt{Dempwolff1939,BradshawCzobor2005}: 42–44; \citealt{Ross2002}: 291). While all the \isi{postposition}s again have meanings denoting specific locations, as we would expect of \isi{adposition}s arising from head nouns in \isi{genitive} constructions\is{possessive construction}, three of the \isi{preposition}s also have meanings of that sort (‘next to’, ‘close to’). In fact, Dempwolff specifically suggests that these \isi{preposition}s arose from verbs (suggesting, for example, that \textit{tamiŋ} ‘next to’ comes from a verb meaning ‘to be close upon’).\footnote{I base this on \citegen{BradshawCzobor2005} \ili{English} translation of \citet{Dempwolff1939}.}

\begin{table}
\caption{Prepositions and postpositions of Jabem }
\label{extab:dryer:23}
\begin{tabularx}{.8\textwidth}{lll Xll} 
\lsptoprule
 & \multicolumn{2}{c}{\bfseries Prepositions\is{preposition}} &  & \multicolumn{2}{c}{\bfseries Postpositions\is{postposition}}\\
\midrule
 & tamiŋ & ‘next to, onto’ &  & lêlôm & ‘inside’\\
 & baŋ & ‘close to’ &  & lôlôc & ‘on top of’\\
 & paŋ & ‘close to’ &  & làbu & ‘under’\\
 & ŋa & ‘\isi{instrumental}’ &  & sawa & ‘between’\\
 & a\textsuperscript{ŋ}ga & ‘from’ &  & lùŋ & ‘in middle of’\\
 &  &  &  & nêm & ‘in front of’\\
 &  &  &  & mu & ‘behind’\\
 &  &  &  & gala & ‘near’\\
 &  &  &  & tali & ‘at edge of’\\
\lspbottomrule
\end{tabularx}
\end{table}

\newpage
% should the following paragraph remain deleted? 
% % While all the postpositions again have meanings denoting specific locations, as we would expect of adpositions arising from head nouns in \isi{genitive} constructions, three of the prepositions also have meanings of that sort (‘next to’, ‘close to’). In fact, Dempwolff specifically suggests that these prepositions arose from verbs (suggesting, for example, that \textit{tamiŋ} ‘next to’ comes from a verb meaning ‘to be close upon’).\footnote{I base this on \citegen{BradshawCzobor2005} \ili{English} translation of \citet{Dempwolff1939}.}
In \tabref{extab:dryer:24} to \ref{extab:dryer:29} are lists of \isi{preposition}s and \isi{postposition}s from six other SVO \& GenN languages that have both. All show patterns similar to those in the four languages described above in this section, with the \isi{preposition}s having meanings associated with \isi{adposition}s arising from verbs and the \isi{postposition}s with meanings associated with \isi{adposition}s arising from nouns.




\begin{table}
\caption{\ili{ǂHoã} (Kxa: Botswana, \citealt{CollinsGruber2014}: 101–105)}
\label{extab:dryer:24} 

\begin{tabularx}{\textwidth}{lll lll} 
\lsptoprule
 & \multicolumn{2}{c}{\bfseries Prepositions\is{preposition}} &  & \multicolumn{2}{c}{\bfseries Postpositions\is{postposition}}\\
\midrule
 & kì & ‘linker’ &  & na & ‘in’\\
 & ke & ‘\isi{comitative}’ &  & za & ‘by, beside’\\
 &  &  &  & ǀǀq'am & ‘above’\\
 &  &  &  & ǂkȁ & ‘below’\\
 &  &  &  & ǂ’hàã & ‘in front of’\\
 &  &  &  & kya“m & ‘near’\\
\lspbottomrule
\end{tabularx}
\end{table}

\begin{table}
\caption{\ili{Koromfe} (\ili{Niger-Congo}, Gur: Burkina Faso, Mali, \citealt{Rennison1997_Koromfe,Rennison2017})}
\label{extab:dryer:25} 
\fittable{
\begin{tabular}{lll lll} 
\lsptoprule
 & \multicolumn{2}{c}{\bfseries Prepositions\is{preposition}} &  & \multicolumn{2}{c}{\bfseries Postpositions\is{postposition}}\\
\midrule
 & la & ‘\isi{instrumental}, \isi{comitative}’ &  & nɛ & ‘\isi{benefactive}, purpose, about’\\
 & hal & ‘until’ &  & kana & ‘like’\\
 &  &  &  & dɔba & ‘on top of’\\
 &  &  &  & hɛrəga & ‘beside, near’\\
 &  &  &  & hogo & ‘under’\\
 &  &  &  & jɪka nɛ & ‘in front of’\\
 &  &  &  & joro & ‘in, inside’\\
 &  &  &  & bɛllɛ & ‘behind’\\
 &  &  &  & tʊllɛ & ‘in the middle of, between’\\
\lspbottomrule
\end{tabular}
}
\end{table}

\begin{table}
\caption{Mandarin Chinese \ili{Chinese (Mandarin)} (\ili{Sino-Tibetan}, Sinitic: China, \citealt{LiThompson1981})}
\label{extab:dryer:26} 
\begin{tabularx}{\textwidth}{lll lll} 
\lsptoprule
& \multicolumn{2}{c}{\bfseries Prepositions\is{preposition} (or coverbs)} &  & \multicolumn{2}{c}{\bfseries Postpositions\is{postposition} (or \isi{locative} particles)}\\
\midrule
 & gēn & ‘with (\isi{comitative})’ &  & shàng & ‘on top of, above’\\
 & gěi & ‘for’ (\isi{benefactive}) &  & xià & ‘below’\\
 & bǎ & \isi{object} marker &  & lǐ & ‘in, inside’\\
 & duì & ‘toward’ &  & wài & ‘outside’\\
 & cóng & ‘from’ &  & qián & ‘in front of’\\
 & zài & ‘at’ &  & hòu & ‘behind’\\
 & tì & ‘instead of’ &  & páng & ‘beside’\\
 & bèi & ‘by’ &  & dōngbu & ‘east of’\\
 & àn & ‘according to’ &  & zhèr & ‘this side of’\\
 & dào & ‘to’ &  & qián & ‘in front of’\\
 &  &  &  & hòu & ‘behind’\\
 &  &  &  & páng & ‘beside’\\
 &  &  &  & zhōngjian & ‘in the centre of’\\
\lspbottomrule
\end{tabularx}
\end{table}

\begin{table}
\caption{\ili{Koyra Chiini} (Songhay: Mali, \citealt{Heath1999}: 104–109)}
\label{extab:dryer:27}
\begin{tabularx}{\textwidth}{lll lll}
\lsptoprule
  & \multicolumn{2}{c}{\bfseries Prepositions\is{preposition}} &  & \multicolumn{2}{c}{\bfseries Postpositions\is{postposition}}\\
\midrule
 & nda & ‘\isi{comitative}, \isi{instrumental}’ &  & se & ‘\isi{dative}’\\
 & bilaa & ‘without’ &  & ra & ‘\isi{locative}’\\
 & hal & ‘until’ &  & ga & ‘beside, from’\\
 & jaa & ‘since’ &  & doo & ‘at the place of’\\
 & bara & ‘except’ &  & banda & ‘behind’\\
 & kala & ‘except’ &  & beene & ‘above’\\
 &  &  &  & čire & ‘under’\\
 &  &  &  & kuna & ‘in’\\
 &  &  &  & jere & ‘beside’\\
 &  &  &  & jine & ‘in front of’\\
 &  &  &  & maasu & ‘inside’\\
 &  &  &  & tenje & ‘facing’\\
\lspbottomrule
\end{tabularx}
\end{table}

\begin{table}
\caption{\ili{Taba} (\ili{Austronesian}, South Halmahera: Indonesia, \citealt{Bowden2001}: 109–111)}
\label{extab:dryer:28}
\begin{tabularx}{\textwidth}{lll lll}
\lsptoprule
& \multicolumn{2}{c}{\bfseries Prepositions\is{preposition}} &  & \multicolumn{2}{c}{\bfseries Postposition}\\
\midrule
 & ada & ‘\isi{comitative}, \isi{instrumental}’ &  & li & ‘on, in, at’\footnote{The fact that the one \isi{postposition} in \ili{Taba} has general \isi{locative} meaning does not fit the expectations for a \isi{postposition} in a GenN language. But the fact that it is \isi{locative} while the \isi{preposition}s are not does fit loosely. It is possible that it originally had a narrower \isi{locative} meaning that has become bleached\is{bleaching}.}\\
 & pake & ‘\isi{instrumental}’ &  &  & \\
 & untuk & ‘\isi{benefactive}’ &  &  & \\
 & lo & ‘like’ &  &  & \\
\lspbottomrule
\end{tabularx} 
\end{table}

\begin{table}
\caption{\ili{Dagbani} (\ili{Niger-Congo}, Gur: Ghana, \citealt{Olawsky1999})}
\label{extab:dryer:29}

\begin{tabularx}{\textwidth}{lll lll} 
\lsptoprule
  & \multicolumn{2}{c}{\bfseries Prepositions\is{preposition}} &  & \multicolumn{2}{c}{\bfseries Postpositions\is{postposition}}\\
\midrule
 & ni & ‘\isi{comitative}, \isi{instrumental}’ &  & nyaaŋa & ‘behind’\\
 & jɛndi & ‘about, concerning’ &  & zuɣu & ‘on top of’\\
 &  &  &  & gbinni & ‘under’\\
 &  &  &  & sani & ‘towards’\\
 &  &  &  & sunsuuni & ‘in the middle of’\\
 &  &  &  & ni & ‘in, at, to’\\
 &  &  &  & puuni & ‘inside’\\
 &  &  &  & polo & ‘in the direction of’\\
 &  &  &  & lɔŋni & ‘under’\\
\lspbottomrule
\end{tabularx}
\end{table}

\clearpage 
The languages illustrated in \tabref{extab:dryer:16} to \tabref{extab:dryer:29}
above are instances of SVO languages with GenN order and both \isi{preposition}s and \isi{postposition}s. Though less common, there are also languages of the opposite sort, OV languages with NGen order and both \isi{preposition}s and \isi{postposition}s, where the semantics associated with \isi{preposition}s and \isi{postposition}s respectively is the opposite of that found in SVO \& GenN languages. An example is Iraqw. Example \REF{ex:dryer:30} illustrates the \isi{preposition} \textit{daandú} ‘behind’. That it has grammaticalized from the head noun in a \isi{genitive} construction\is{possessive construction} is clear from the fact that it occurs in \isi{construct state}, the morphological form that head nouns take in \isi{genitive} constructions.

\ea\label{ex:dryer:30}
\langinfo{\ili{Iraqw}}{\ili{Afro-Asiatic}, \ili{Cushitic}: Tanzania}{\citealt{Mous1993}: 97}\\
\gll   looʾa  i  \textbf{daandú}  \textbf{hunkáy}.\\
       sun  3\textsc{sbj}  \textbf{behind.\textsc{constr}}  \textbf{cloud} \\
\glt ‘The sun is \textbf{behind} \textbf{the} \textbf{cloud}.’
\z

\noindent In contrast, example \REF{ex:dryer:31} illustrates a \isi{postposition}al clitic =\textit{i} ‘directional’ that attaches to the last word in the noun phrase. In \REF{ex:dryer:31} it attaches to the noun \textit{doʾ} ‘house’, the \isi{possessor} of \textit{afkú} ‘mouth’ (‘door’), but it is marking the entire noun phrase \textit{afkú} \textit{doʾ} ‘mouth (door) of the house’ as the goal of the motion \isi{allative} denoted by the verb \textit{qaas} ‘put’.

\ea\label{ex:dryer:31}
\langinfo{\ili{Iraqw}}{\ili{Afro-Asiatic}, \ili{Cushitic}: Tanzania}{\citealt{Mous1993}: 252}\\
\gll famfeʾamo  u-n  \textbf{af-kú}  \textbf{doʾ=i}  qaas-áan.\\
       snake  \textsc{masc.obj-expec}  \textbf{mouth-\textsc{constr.masc}}  \textbf{house=\textsc{dir}}  put-\textsc{1pl}  \\
\glt ‘Let us put a snake \textbf{on} \textbf{the} \textbf{door} \textbf{of} \textbf{the} \textbf{house}.’
\z

In \tabref{extab:dryer:32} is a list of \isi{preposition}s and \isi{postposition}s in \ili{Iraqw} (\citealt{Mous1993}: 95–107). Setting aside momentarily the first three \isi{preposition}s in \tabref{extab:dryer:32}, the semantics associated with the \isi{preposition}s and \isi{postposition}s in \ili{Iraqw} is the reverse of what we found in \REF{ex:dryer:11} to \REF{extab:dryer:29} for SVO \& GenN languages. Namely, in \tabref{extab:dryer:32}, it is the \isi{preposition}s which denote specific locations, while the \isi{postposition}s have meanings that are generally associated with \isi{adposition}s arising from verbs.

\begin{table}
\caption{Prepositions and postpositions in \ili{Iraqw}}
\label{extab:dryer:32}
\begin{tabularx}{\textwidth}{lll Xll}
\lsptoprule
 & \multicolumn{2}{c}{\bfseries Prepositions\is{preposition}} &  & \multicolumn{2}{c}{\bfseries Postpositions\is{postposition}}\\
\midrule
 & ar & ‘\isi{instrumental}’ &  & =(a)r & ‘\isi{instrumental}, \isi{comitative}’\\
 & as & ‘because of’ &  & =sa & ‘because of’\\
 & ay & ‘to’ &  & =i & ‘to’\\
 & dír & ‘to' &  & =wa & ‘from’\\
 & amór & ‘at’ &  &  & \\
 & daandú & ‘on’ &  &  & \\
 & alá & ‘behind’ &  &  & \\
 & gurúu & ‘inside’ &  &  & \\
 & gamú & ‘under’ &  &  & \\
 & bihháa & ‘beside’ &  &  & \\
 & tlaʿá(ng) & ‘between’ &  &  & \\
 & tseeʿá & ‘outside’ &  &  & \\
 & afíqoomár & ‘until’ &  &  & \\
 & gawá & ‘on’ &  &  & \\
 & geerá & ‘before’ &  &  & \\
 & afá & ‘at the edge of’ &  &  & \\
 & bará & ‘in’ &  &  & \\
\lspbottomrule
\end{tabularx}
\end{table}

% should the following paragraph remain deleted? 
% % Setting aside momentarily the first three prepositions in \tabref{extab:dryer:32}, the semantics associated with the prepositions and postpositions in Iraqw is the reverse of what we found in \REF{ex:dryer:11} to \REF{extab:dryer:29} for SVO \& GenN languages. Namely, in \tabref{extab:dryer:32}, it is the prepositions which denote specific locations, while the postpositions have meanings that are generally associated with adpositions arising from verbs.
%

The first three \isi{preposition}s in \tabref{extab:dryer:32} have the same meanings as the first three \isi{postposition}s in the table. Their meanings are thus ones that we might have expected to be associated with \isi{postposition}s in an OV language. These \isi{preposition}s take the form of /a/ plus the corresponding \isi{postposition}al clitics. \citet[102]{Mous1993} speculates that the /a/ in these forms may have originally been the \isi{copula} \textit{a}. It is possible that these \isi{preposition}s have arisen by \isi{analogy} to other \isi{preposition}s in the language.

A second instance of an OV \& NGen language with both \isi{preposition}s and \isi{postposition}s is \ili{Kanuri}. Example \REF{ex:dryer:33} illustrates the \isi{locative}-\isi{instrumental} \isi{postposition}al clitic =\textit{lan} attaching to a postnominal \isi{modifier} \textit{Musa=be} ‘Musa’s’, marking \textit{fər} \textit{Musa=be} ‘Musa’s horse’ as an \isi{instrumental}.\footnote{There are thus two \isi{postposition}al clitics in the phonological word \textit{Musa=be=lan} in \tabref{extab:dryer:32}, the \textit{=be} marking Musa as \isi{possessor} of \textit{fər} ‘horse’ and the =\textit{lan} marking  \textit{fər} \textit{Musa=be} ‘Musa’s horse’ as an \isi{instrumental}.}

\ea\label{ex:dryer:33}
\langinfo{\ili{Kanuri}}{Saharan: Nigeria, Niger}{\citealt{Hutchinson1976}: 5}\\
\gll [\textbf{fər}  \textbf{Musa=be}]\textbf{=lan}  kadio.\\
       [\textbf{horse}  \textbf{Musa=\textsc{gen}}]=\textbf{\textsc{ins}}  come.\textsc{pst}.3\textsc{sg}  \\
\glt ‘He came \textbf{on/by} \textbf{Musa’s} \textbf{horse}.’
\z



\noindent \ili{Kanuri} also has \isi{preposition}s, like \textit{suro} ‘inside’ in \REF{ex:dryer:34}.

\ea\label{ex:dryer:34}
\langinfo{\ili{Kanuri}}{Saharan: Nigeria, Niger}{\citealt{Hutchinson1976}: 80}\\
\gll   \textbf{suro}  \textbf{fato=be=ro}  kargawo.\\
       \textbf{inside}  \textbf{house=\textsc{gen}}\textbf{=to}  enter.\textsc{pst}.3\textsc{sg} \\
\glt   ‘He went \textbf{into} \textbf{the} \textbf{house}.’
\z

\noindent Note that \textit{suro} retains its nominal nature in \REF{ex:dryer:34}, in that its \isi{complement} \textit{fato} ‘house’ is marked as a \isi{possessor}, with the \isi{genitive} \isi{postposition}al clitic \textit{=be}, and the entire phrase marked with the \isi{postposition}al clitic \textit{=ro} ‘to’, so that \REF{ex:dryer:34} could be glossed as ‘He went to the inside of the house’. To what extent these locational nouns have grammaticalized\is{grammaticalization} as \isi{preposition}s is not clear. Even if they have not grammaticalized much yet, they illustrate how an OV \& NGen language could acquire \isi{preposition}s.

In \tabref{extab:dryer:35} is a list of \isi{preposition}s and \isi{postposition}s of \ili{Kanuri} (\citealt{Hutchinson1981}: 257–263).


\begin{table}
\caption{Prepositions and postpositions of Kanuri }
\label{extab:dryer:35}
\fittable{
\begin{tabular}{lll@{}lll} 
\lsptoprule
 & \multicolumn{2}{c}{\bfseries Prepositions\is{preposition}} &  & \multicolumn{2}{c}{\bfseries Postpositions\is{postposition}}\\
\midrule
 & bótówò & ‘next to’ &  & =(là)n & ‘\isi{locative}, \isi{instrumental}’\\
 & cî & ‘at edge of’ &  & =rò & ‘\isi{benefactive}, \isi{indirect object}, to’\\
 & dàryé & ‘at the end of’ &  & =mbèn & ‘through, towards’\\
 & dáwù & ‘in middle of’ &  &  & \\
 & fúwù & ‘in front of’ &  &  & \\
 & fərtə & ‘at base of’ &  &  & \\
 & gəré & ‘next to’ &  &  & \\
 & kátè & ‘between’ &  &  & \\
 & kəlâ & ‘on top of’ &  &  & \\
 & ngáwò & ‘behind, after’ &  &  & \\
 & sədíà {\textasciitilde} cídíà & ‘under’ &  &  & \\
 & súró & ‘inside, during’ &  &  & \\
\lspbottomrule
\end{tabular}
}
\end{table}

\noindent The meanings associated with the \isi{preposition}s in \ili{Kanuri} are similar to those of the \isi{preposition}s in \ili{Iraqw}, but are also similar to the meanings of the \isi{postposition}s in the various SVO \& GenN languages discussed above. Conversely, the meanings associated with the \isi{postposition}s in \ili{Kanuri} are similar to those of the \isi{postposition}s in \ili{Iraqw} and also similar to the meanings of the \isi{preposition}s in the various SVO \& GenN languages discussed above

There is another instance of a language with both \isi{preposition}s and \isi{postposition}s that provides an interesting variation of the argument in this section, namely \ili{English}. While \ili{English} is predominantly a \isi{preposition}al language, it has at least two \isi{postposition}s, \textit{ago} and \textit{notwithstanding}, as in \REF{ex:dryer:36}.\footnote{\textit{Notwithstanding} also occurs as a \isi{preposition}. The \isi{postposition}al use is apparently the original use. I suspect that the use as a \isi{preposition} arose due to its semantic similarity to another \isi{preposition} \textit{despite}.}

\ea\label{ex:dryer:36}
\langinfo{}{}{\ili{English}}\\
\ea  I saw him three weeks ago.\\
\ex  I went to the concert, the doctor’s advice notwithstanding.\\
\z
\z

\noindent What is unusual about these two \isi{postposition}s in \ili{English} is that although both are apparently \isi{grammaticalization}s of verbs, they are ones where what is now the \isi{object} of that \isi{postposition} was originally the \isi{subject} of the verb (rather than the \isi{object}, the more common situation with grammaticalizations from verbs). According to the Merriam Webster online dictionary,\footnote{\url{https://www.merriam-webster.com/dictionary}} \textit{ago} comes from an obsolete verb meaning ‘pass’ so that \textit{three} \textit{weeks} \textit{ago}  derives from \textit{three} \textit{weeks} \textit{have} \textit{passed}, where \textit{three} \textit{weeks} was originally the \isi{subject} of this verb. And \textit{notwithstanding} comes from \textit{not} plus a form of the verb meaning ‘withstand’ in the sense of ‘providing an obstacle for’; again, what is now the \isi{object} of the \isi{postposition} \textit{notwithstanding} was originally the \isi{subject} of the verbal expression. The fact that these two words arose as \isi{postposition}s rather than as \isi{preposition}s reflects the fact that \isi{subject}s normally preceded the verb, even in earlier varieties of \ili{English} when \isi{word order} was more flexible. Again, only a \isi{grammaticalization} account explains these.

The evidence in this section involves data that only \isi{grammaticalization} can explain. An explanation in terms of \isi{grammaticalization} for the correlation\is{word-order correlation} between the order of verb and \isi{object} and order of \isi{adposition} and noun phrase as well as the correlation between the order of noun and \isi{genitive} and order of \isi{adposition} and noun phrase predicts that we should find both \isi{preposition}s and \isi{postposition}s in the same language where the former derive from verbs and the latter from head nouns in \isi{genitive} constructions\is{possessive construction}, as well as predicting the semantic differences between the two types of \isi{adposition}. The evidence in this section shows how these predictions are borne out. There is no obvious way in which accounts in terms of \isi{processing} or similarity could explain this data.

\section{What grammaticalization does not explain}\label{sec:dryer:3}

The preceding section provides evidence that \isi{grammaticalization} explains, at least partly, the correlation\is{word-order correlation} between the order of verb and \isi{object} and order of \isi{adposition} and noun phrase as well as the correlation between the order of noun and \isi{genitive} and order of \isi{adposition} and noun phrase. In this section, I discuss the question whether \isi{grammaticalization} fully explains word order correlations\is{word-order correlation} and argue that it does not. I first discuss word order correlations\is{word-order correlation} for which there does not seem to be any good explanation in terms of \isi{grammaticalization}. \tabref{tab:dryer:6} provides a list of pairs of elements that are shown by \citet{Dryer1992} to correlate with the order of verb and \isi{object}, where the verb patterner refers to elements that occur first in these pairs more often among VO languages than among OV languages (and where the \isi{object} patterner refers to the other member of the pair).

\begin{table}
\begin{tabularx}{\textwidth}{XXl}
\lsptoprule
verb patterner &  \isi{object} patterner  & example\\
\midrule 
verb &  \isi{adposition}al phrase  	&\textit{slept} + \textit{on} \textit{the} \textit{floor}   \\
verb & \isi{manner} adverb 		& \textit{ran} + \textit{slowly}                            \\
\isi{copula} verb & predicate  	&\textit{is} + \textit{a} \textit{teacher}                  \\
‘want'\is{desiderative} & VP  			&\textit{wants} + \textit{to} \textit{see} \textit{Mary}    \\
noun & \isi{relative clause}  	&\textit{movies} + \textit{that} \textit{we} \textit{saw}   \\
\isi{adjective} & standard of \isi{comparison}  &\textit{taller} + \textit{than} \textit{Bob}           \\
\isi{complementizer} &  clause  	&\textit{that} + \textit{John} \textit{is} \textit{sick}    \\
question\is{interrogative} particle & sentence     & {}                      \\
adverbial \isi{subordinator} & clause\is{adverbial clause} & \textit{because} + \textit{Bob} \textit{has} \textit{left}\\
\lspbottomrule
\end{tabularx}

\caption{\label{tab:dryer:6}Pairs of elements that correlate with the order of verb and object}
\end{table}

For none of these pairs of elements that correlate with the order of verb and \isi{object} is there a convincing explanation in terms of \isi{grammaticalization}. For example, the order of verb and \isi{adposition}al phrase most likely correlates with the order of verb and \isi{object} because of semantic similarities between these two pairs of elements or because of \isi{processing} factors. It is hard to imagine an explanation in terms of \isi{grammaticalization} for this correlation.

I devote the remainder of this section to discussing the correlation\is{word-order correlation} between the order of verb and \isi{object} and the order of noun and \isi{genitive}. While there have been attempts to explain this correlation in terms of \isi{grammaticalization}, I claim here that such attempts fall short of providing a plausible explanation. A good summary of this approach is provided by \citetv{Collins2019tv}. However, most of the cases discussed by Collins are highly speculative, especially compared to the evidence for \isi{adposition}s deriving from verbs or nouns. The arguments involve cases where the constructions now used for main clauses are claimed to have originated from \isi{nominalization}s (where a construction like \textit{John’s} \textit{seeing} \textit{Peter} is claimed to have replaced an existing \isi{finite} construction like \textit{John} \textit{saw} \textit{Peter}).\footnote{Some of Collins’ arguments are particularly unconvincing. He cites data from \ili{Angas} showing \isi{nominalization}s being used for \isi{complement}s of the verb meaning ‘want’\is{desiderative}. But this only shows that some languages express such complements using nominalizations; it provides no evidence of nominalizations coming to be used as main clauses. He also cites the large number of \ili{Austronesian} languages as evidence for the frequency by which nominalizations become main clauses. But quite apart from the fact that Collins provides no evidence to support his claim that it is generally accepted that nominalizations came to be used as main clauses in \ili{Austronesian}, the size of the family is not relevant; what is relevant is the number of instances of changes of this sort. A number of proposals that main clause constructions originated as nominalizations are based largely on the fact that the same \isi{case} marker is used for both \isi{possessor}s and \isi{subject}s (or \isi{transitive} subjects). But
\label{p:dryer:manyotherways}
there are many ways by which this can arise without \isi{nominalization}s coming to be used as main clauses.} Assuming that the \isi{word order} in \isi{nominalization}s reflects the order of noun and \isi{genitive} (an assumption that is probably valid), the new construction will employ an order of verb and \isi{object} that reflects the order of noun and \isi{genitive}.\footnote{It will also determine the order of verb and \isi{subject}, especially for \isi{intransitive} verbs. There are issues arising here that are beyond the scope of this paper. And while I find the evidence that \isi{grammaticalization} explains the correlation\is{word-order correlation} between the order of verb and \isi{object} and the order of noun and \isi{genitive}\is{possessive construction} unconvincing, I must concede that it would account for the large number of SVO \& GenN languages. In other words, it would account for the fact that the order of noun and \isi{genitive} is one of the few orders that correlates\is{word-order correlation}not only with the order of verb and \isi{object} but also with the order of verb and \isi{subject} \citep{Dryer2013_Six2}.}

While there probably have been some instances in which a \isi{nominalization} construction came to be used as the primary construction for main clauses, there is little evidence of this in most families and the correlation\is{word-order correlation} between the order of verb and \isi{object} and the order of noun and \isi{genitive} seems far too strong to be explained purely in this way. Consider the data in \tabref{tab:dryer:7} on the relative frequency of the different orders of noun and \isi{genitive} in OV languages.

\begin{table}
\begin{tabularx}{\textwidth}{Xrrrrrr}
\lsptoprule
& \bfseries Africa & \bfseries Euras & \bfseries Oceania & \bfseries N.Amer & \bfseries S.Amer & \bfseries TOTAL\\
\midrule
OV \& GenN & [26] & [46] & [87] & [34] & [54] & 247\\
OV \& NGen & 13{\db} & 1{\db} & 10{\db} & 0{\db} & 1{\db} & 25\\
\lspbottomrule
\end{tabularx}
\caption{\label{tab:dryer:7}Order of noun and genitive in OV languages}
\end{table}

\tabref{tab:dryer:7} shows that GenN order outnumbers NGen by 247 to 25 genera, a ratio of almost 10-to-1. The evidence for \isi{nominalization}s coming to be used as main clauses is far too meagre to account for such a strong correlation\is{word-order correlation}.

It should be noted that the order of noun and \isi{genitive} correlates\is{word-order correlation} with the order of verb and \isi{object} less strongly than the order of \isi{adposition} and noun phrase correlates\is{word-order correlation} with either the order of verb and \isi{object} or the order of noun and \isi{genitive}: Tables \ref{tab:dryer:1} and \ref{tab:dryer:2} above show a particularly strong correlation\is{word-order correlation} between the order of verb and \isi{object} and the order of \isi{adposition} and noun phrase; Tables \ref{tab:dryer:3} and \ref{tab:dryer:4} show an even stronger correlation\is{word-order correlation} between the order of noun and \isi{genitive} and the order of \isi{adposition} and noun phrase. But the large number of SVO \& GenN languages shows that the correlation between the order of verb and \isi{object} and the order of noun and \isi{genitive} is less strong.

One possible explanation for why the correlation\is{word-order correlation} between the order of verb and \isi{object} and the order of noun and \isi{genitive} is weaker is that all three of these correlations are due in part to factors other than \isi{grammaticalization} (such as the \isi{processing} explanations of \citealt{Dryer1992} and \citealt{Hawkins1994_Perf,Hawkins2004_Eff,Hawkins2014_VarEff}), but that \isi{grammaticalization} augments the correlation between the order of verb and \isi{object} and the order of \isi{adposition} and noun phrase as well as the correlation between the order of noun and \isi{genitive} and the order of \isi{adposition} and noun phrase. In other words, it may be a mistake to try to choose between \isi{grammaticalization} and other factors in explaining word order correlations\is{word-order correlation}; they may conspire to lead to these stronger correlations.

In fact, data presented by \citet{Dryer1992_Greenb,Dryer2013_Six2} suggests that the correlation\is{word-order correlation} between the order of verb and \isi{object} and the order of \isi{adposition} and noun phrase as well as the correlation between the order of noun and \isi{genitive} and the order of \isi{adposition} and noun phrase are stronger than most of the correlations in \tabref{tab:dryer:6} above. Since there do not appear to be promising explanations for those correlations\is{word-order correlation}in terms of \isi{grammaticalization}, the fact that the two correlations involving \isi{adposition}s are particularly strong suggests again that both \isi{grammaticalization} and other factors play a role in explaining those correlations.

Note also that \isi{grammaticalization} explains the fact mentioned above in \sectref{sec:dryer:2.1} that the preference for \isi{postposition}s among OV languages is stronger than the preference for \isi{preposition}s among VO languages. Namely, OV languages are overwhelmingly GenN so that both sources for \isi{adposition}s lead to \isi{postposition}s in OV languages. In contrast there are many SVO languages with GenN order. In such languages the \isi{adposition}s derived from head nouns will be \isi{postposition}s, so that (assuming some such languages lack \isi{adposition}s derived from verbs) \largerpage we expect to find SVO languages with \isi{postposition}s.

\section{Order of noun and definiteness marker}\label{sec:dryer:4}

In this section, I discuss a different type of problem for \isi{grammaticalization} accounts of word order correlations\is{word-order correlation}. In the cases discussed in \sectref{sec:dryer:3}, \isi{grammaticalization} simply fails to predict a word order correlations\is{word-order correlation} which can be shown to be real. In the case discussed in this section, \isi{grammaticalization} makes a prediction that turns out not to hold, involving the order of \isi{definiteness} marker and noun.

The most common \isi{grammaticalization} source for markers of \isi{definiteness} appears to be demonstratives. In fact my database contains 102 instances of languages that use \isi{demonstrative}s as markers of \isi{definiteness}, compared to 274 languages with markers of \isi{definiteness} that are distinct from \isi{demonstrative}s. Both the order of \isi{definiteness} marker and noun and the order of \isi{demonstrative} and noun exhibit weak correlations\is{word-order correlation} with the order of verb and \isi{object}, but what is surprising from the perspective of \isi{grammaticalization} is that they exhibit opposite correlations. Namely, \isi{definiteness} markers \textit{precede} the noun more often in VO languages than in OV languages, while demonstratives \textit{follow} the noun more often in VO languages than in OV languages.

Consider first \isi{definiteness} markers in VO languages. \tabref{tab:dryer:8} provides data on the order of \isi{definiteness} marker and noun in VO languages. The last line in \tabref{tab:dryer:8} gives the proportion of the number on the first line as a proportion of the sum of the number on the first line and the number on the second line. For example, the .21 on the third line in \tabref{tab:dryer:8} under Africa represents 8 as a proportion of 39 (the sum of 8 and 31). I use these proportions in the discussion below.

\begin{table}
\begin{tabularx}{\textwidth}{lrrrrrr} 
\lsptoprule
& \bfseries Africa & \bfseries Euras & \bfseries Oceania & \bfseries N.Amer & \bfseries S.Amer & \bfseries TOTAL\\
\midrule
VO \& DefN & 8{\db} & [11] & [16] & [17] & [7] & 59\\
VO \& NDef & [31] & 3{\db} & 13{\db} & 8{\db} & 0{\db} & 55\\
Proportion DefN & .21 & .79 & .55 & .68 & 1.00 & $\bar{x}$=.64\\
\lspbottomrule
\end{tabularx}
\caption{\label{tab:dryer:8}Order of noun and definiteness marker in VO languages} 
\end{table}

\tabref{tab:dryer:8} shows the two orders of \isi{definiteness} marker and noun to be about equally common among VO languages, with DefN order found in languages in 59 genera and NDef order found in languages in 55 genera. This is a case, however, where the total numbers of genera are somewhat misleading, since one area, Africa, exhibits a very different pattern from what we find in the other four areas. In Africa, genera containing VO languages in which the \isi{definiteness} marker follows the noun outnumber genera containing VO languages in which the \isi{definiteness} marker precedes the noun by 31 to 8. In the other four areas, in contrast, it is more common among VO languages for the \isi{definiteness} marker to precede the noun; in fact, in three of the areas (Eurasia, North America, and South America), DefN order is more than twice as common as NDef order. The mean of the proportions over the five areas, namely .64, also reflects a preference for DefN order among VO languages. Another way to see this is that if we exclude Africa, DefN outnumbers NDef among VO languages by 51 to 24.\footnote{The higher preference for NDef order among VO languages in Africa reflects a general difference between Africa and the rest of the world in that postnominal \isi{modifier}s are more common in Africa than elsewhere \citep{Dryer2010}. \tabref{tab:dryer:7} above shows a similar difference between Africa and the rest of the world: while GenN outnumbers NGen among OV languages overall by almost 10-to-1, the ratio in Africa is only 2-to-1 and over half (13 out of 25) of the genera containing OV \& NGen languages are in Africa.}

\tabref{tab:dryer:9} provides comparable data on the order of \isi{definiteness} marker and noun among OV languages.  We again find only a small difference, though it is NDef that outnumbers DefN among OV languages, by 53 genera to 38.

\begin{table}
\begin{tabularx}{\textwidth}{lrrrrrr} 
\lsptoprule
& \bfseries Africa & \bfseries Euras & \bfseries Oceania & \bfseries N.Amer & \bfseries S.Amer & \bfseries TOTAL\\
\midrule
OV \& DefN & 3{\db} & [9] & 15{\db} & 4{\db} & [7] & 38\\
OV \& NDef & [12] & 5{\db} & [23] & [9] & 4{\db} & 53\\
Proportion DefN & .20 & .64 & .39 & .31 & .64 & $\bar{x}$=.44\\
\lspbottomrule
\end{tabularx} 
\caption{Order of noun and definiteness marker in OV languages}
\label{tab:dryer:9}
\end{table}

But what is revealing is to compare the proportions from the last lines of Tables \ref{tab:dryer:8} and \ref{tab:dryer:9}, given in \tabref{tab:dryer:10}.

\begin{table}
\begin{tabularx}{\textwidth}{Xrrrrrr} 
\lsptoprule
& \bfseries Africa & \bfseries Eurasia & \bfseries Oceania & \bfseries N.America & \bfseries S.America & \bfseries Mean\\
\midrule 
VO & [.21] & [.79] & [.55] & [.68] & [1.00] & .64\\
OV & .20{\db} & .64{\db} & .39{\db} & .31{\db} & .64{\db} & =.44\\
\lspbottomrule
\end{tabularx} 
\caption{\label{tab:dryer:10} Proportion of genera containing DefN languages among VO vs. OV languages}
\end{table}

\noindent Here we find that although the margin of difference in Africa is very small, it is still the case that the proportion of genera containing DefN languages is greater among VO languages in all five areas. This gives us reason to conclude that there is a correlation\is{word-order correlation}, albeit a weak one, between the order of verb and \isi{object} and the order of \isi{definiteness} marker and noun, with the \isi{definiteness} marker preceding the noun more often among VO languages than among OV languages.

Given the fact that the most common \isi{grammaticalization} source for \isi{definiteness} markers appears to be \isi{demonstrative}s, we might expect to find a similar correlation\is{word-order correlation} between the order of verb and \isi{object} and the order of \isi{demonstrative} and noun. We do find a clear trend, but it is the opposite correlation. Namely while \isi{definiteness} markers precede the noun more often among VO languages compared to OV languages, \isi{demonstrative}s tend to follow the noun more often among VO languages compared to OV languages.

Tables \ref{tab:dryer:11} to \ref{tab:dryer:13} provide data supporting this. \tabref{tab:dryer:11} provides relevant data for VO languages. It shows that although NDem order is slightly more common than DemN order, by 118 genera to 92, this order is more common in only three of the five areas (and in fact, if we exclude Africa, it is DemN order that is more common among VO languages, by 84 genera to 66).

\begin{table}
\begin{tabularx}{\textwidth}{lrrrrrr} 
\lsptoprule
& \bfseries Africa & \bfseries Euras & \bfseries Oceania & \bfseries N.Amer & \bfseries S.Amer & \bfseries TOTAL\\
\midrule 
VO \& DemN & 8{\db} & 12{\db} & 24{\db} & [24] & [24] & 92\\
VO \& NDem & [52] & [16] & [31] & 12{\db} & 7{\db} & 118\\
Proportion DemN & .13 & .43 & .44 & .67 & .77 & $\bar{x}$=.49\\
\lspbottomrule
\end{tabularx}
\caption{\label{tab:dryer:11}Order of noun and demonstrative in VO languages}
\end{table}


\noindent However, \tabref{tab:dryer:12} shows that among OV languages, DemN order is about twice as common as NDem order, by 181 genera to 95, although there are two areas where NDem is more common among OV languages.

\begin{table}
\begin{tabularx}{\textwidth}{lrrrrrr}
\lsptoprule
& \bfseries Africa & \bfseries Euras & \bfseries Oceania & \bfseries N.Amer & \bfseries S.Amer & \bfseries TOTAL\\
\midrule
OV \& DemN & 16{\db} & [44] & 45{\db} & [30] & [46] & 181\\
OV \& NDem & [18] & 6{\db} & [57] & 6{\db} & 8{\db} & 95\\
Proportion DemN & .47 & .88 & .44 & .83 & .85 & $\bar{x}$=.70\\
\lspbottomrule
\end{tabularx}
\caption{\label{tab:dryer:12}Order of noun and demonstrative in OV languages}
\end{table}

Again, it is useful to compare the proportions from the last lines of Tables \ref{tab:dryer:11} and \ref{tab:dryer:12}, shown in \tabref{tab:dryer:13}.

\begin{table}
\begin{tabularx}{\textwidth}{Xlrrrrr}
\lsptoprule
& \bfseries Africa & \bfseries Eurasia & \bfseries Oceania & \bfseries N.America & \bfseries S.America & \bfseries Mean\\
\midrule
VO & .13{\db} & .43{\db} & .44 & .67{\db} & .77{\db} & .49\\
OV & [.43] & [.88] & .44 & [.83] & [.85] & .70\\
\lspbottomrule
\end{tabularx}
\caption{\label{tab:dryer:13}Proportion of genera containing DemN languages among VO vs. OV languages} 
\end{table}

\noindent \tabref{tab:dryer:13} shows that the proportion of genera containing DemN languages is higher among OV languages in four areas while the proportion is the same in the fifth area (Oceania).\footnote{If we compute the proportions to three decimal places, DemN is also higher among OV languages compared to VO languages in Oceania (by .441. to .434). However, this difference is too small to base any conclusion on.} There is thus a clear trend in the opposite direction from what we found for the order of \isi{definiteness} marker and noun. Given that the most common \isi{grammaticalization} source for \isi{definiteness} markers appears to be \isi{demonstrative}s, this contrast is quite surprising.


I have no explanation for the source of this difference between \isi{definiteness} markers and \isi{demonstrative}s. But I will share some interesting data from particular languages that conforms to this difference. First, there are a few languages in which the same form is used as a \isi{demonstrative} and as a marker of \isi{definiteness}, but this form occurs on different sides of the noun, depending on its function. In \ili{Swahili}, the forms that are used as distal \isi{demonstrative}s when following the noun function as markers of \isi{definiteness} when they precede the noun, as shown in \REF{ex:dryer:37}. Since \ili{Swahili} is SVO, this difference conforms to the contrast in the crosslinguistic data shown above.

\ea\label{ex:dryer:37}
\langinfo{\ili{Swahili}}{\ili{Niger-Congo}, Bantoid}{\citealt{Ashton1947}: 59}\\
  \ea \gll  m-tu  \textbf{yu-le}\\
	  \textsc{nc}\textsubscript{1}{}-man  \textbf{\textsc{nc}}\textbf{\textsubscript{1}}\textbf{{}-that}\\
  \glt     ‘that man’

  \ex
  \gll    \textbf{yu-le}  m-tu\\
	  \textbf{\textsc{nc}}\textbf{\textsubscript{1}}\textbf{{}-}\textbf{\textsc{def}}  \textsc{nc}\textsubscript{1}{}-man\\
  \glt     ‘the man’
  \z
\z

In \ili{Abui}, we find the opposite situation: the form \textit{do} functions as a \isi{demonstrative} when it precedes the noun, as in \REF{ex:dryer:38a}, but as a marker of \isi{definiteness} when it follows the noun, as in \REF{ex:dryer:38b}.

\ea\label{ex:dryer:38}
\langinfo{\ili{Abui}}{Timor-Alor-Pantar: Indonesia}{\citealt{Kratochvil2007}: 111, 114}\\
\ea\label{ex:dryer:38a}
\gll     \textbf{do}  sura  \\  
         \textbf{this}  book\\    
\glt     ‘this book (near me)’  
\ex \label{ex:dryer:38b}
\gll kaai  \textbf{do}\\
    dog  \textbf{\textsc{def}}\\
\glt  ‘the dog (I just talked about)’
\z
\z

\noindent Significantly, \ili{Abui} is an OV language, so the fact that \ili{Abui} exhibits the opposite pattern from what we saw in \ili{Swahili} again conforms to the crosslinguistic pattern described above.

The situation in \ili{Ute} is similar to that in \ili{Abui}. Namely \ili{Ute} is OV and the word \textit{'u} functions as a \isi{demonstrative} when it precedes the noun, as in \REF{ex:dryer:39a}, but as a marker of \isi{definiteness} when it follows the noun, as in \REF{ex:dryer:39b}.

\ea\label{ex:dryer:39}
\langinfo{\ili{Ute}}{\ili{Uto-Aztecan}: United States}{\citealt{Givón2011}: 50, 38}\\
\ea\label{ex:dryer:39a}
\gll     \textbf{'ú}  kava  sá-gha-rʉ-mʉ  qhárʉ-kwa-pʉga.\\
         \textbf{that.\textsc{sbj}}  horse.\textsc{sbj}  white-have-\textsc{nmlz-anim.sbj}  run-go-\textsc{rem}\\
\glt     ‘That white horse ran away.’
\ex\label{ex:dryer:39b}
\gll    ta'wa-chi  \textbf{'u}  sivaatu-chi  paqha-qa.\\
         man-\textsc{anim.sbj}  \textbf{\textsc{def.sbj}}  goat-\textsc{anim.obj}  kill-\textsc{ant}\\
\glt     ‘The man killed a goat.’
\z
\z

The situation in \ili{Loniu} is somewhat different. In \ili{Loniu}, the \isi{definiteness} marker and \isi{demonstrative} are similar in form, though not identical, with \textit{iy} as the \isi{definiteness} marker and \textit{iyɔ} as the \isi{demonstrative}. The two in fact can co-occur as in \REF{ex:dryer:40}, with the \isi{definiteness} marker preceding the noun, and the \isi{demonstrative} following the noun.

\ea\label{ex:dryer:40}
\langinfo{\ili{Loniu}}{\ili{Austronesian}, \ili{Oceanic}: Papua New Guinea}{\citealt{Hamel1994}: 100}\\
\gll   iy   amat   iyɔ\\
       \textsc{def}   man   this \\
\glt   ‘this man’
\z

\noindent Again, since \ili{Loniu} is VO, this order difference conforms to the crosslinguistic pattern described above.

And we find similar phenomena in cases where the \isi{definiteness} marker and \isi{demonstrative} are completely different in form but can co-occur, with one preceding the noun and one following. In \ili{Kana}, the \isi{definiteness} marker precedes the noun while the \isi{demonstrative} follows, as in \REF{ex:dryer:41}.

\ea\label{ex:dryer:41}
\langinfo{\ili{Kana}}{\ili{Niger-Congo}, Delta Cross: Nigeria}{\citealt{Ikoro1996}: 70}\\
\gll   ló   bárí   āmā \\
       \textsc{def}   fish   this \\
\glt   ‘this fish’
\z

\noindent Since \ili{Kana} is VO, this conforms to the crosslinguistic pattern. Contrast this with the situation in \ili{Kwoma} (\ili{Washkuk}), which is OV, and in this case it is the \isi{demonstrative} that precedes the noun and the \isi{definiteness} marker that follows, as in \REF{ex:dryer:42}.

\ea\label{ex:dryer:42}
\langinfo{\ili{Kwoma}}{Sepik: Papua New Guinea}{\citealt{Kooyers1974}: 49}\\
\gll   kata  ma  rii\\
       that  man  \textsc{def}\\
\glt   ‘that man’
\z

These differences between demonstratives and \isi{definiteness} markers are a puzzle if demonstratives are the primary \isi{grammaticalization} source for \isi{definiteness} markers. It should be emphasized, however, that although \isi{definiteness} markers and \isi{demonstrative}s exhibit very different patterns in terms of how they correlate\is{word-order correlation} with the order of verb and \isi{object}, it is still the case that they correlate with each other, i.e. that the order of \isi{definiteness} marker and noun and the order of \isi{demonstrative} and noun correlate. This is shown in Tables \ref{tab:dryer:14} and \ref{tab:dryer:15}, excluding languages where the \isi{definiteness} marker is the same as the \isi{demonstrative}. \tabref{tab:dryer:14} shows that among DefN languages with \isi{definiteness} markers that are distinct from \isi{demonstrative}s, it is approximately twice as common for the \isi{demonstrative} to precede the noun as well, by 41 genera to 20.

\begin{table}
\begin{tabularx}{\textwidth}{Xrrrrrr}
\lsptoprule
& \bfseries Africa & \bfseries Euras & \bfseries Oceania & \bfseries N.Amer & \bfseries S.Amer & \bfseries TOTAL\\
\midrule
DefN \& DemN & 3{\db} & [7] & [12] & [11] & [8] & 41\\
DefN \& NDem & [4] & 3 & 7{\db} & 3{\db} & 3{\db} & 20\\
\lspbottomrule
\end{tabularx}
\caption{\label{tab:dryer:14}Order of noun and demonstrative in DefN languages}
\end{table}

Conversely, \tabref{tab:dryer:15} shows that among NDef languages with \isi{definiteness} markers that are distinct from \isi{demonstrative}s, it is much more common for the \isi{demonstrative} to follow the noun as well, by 67 genera to 11.

\begin{table}
\begin{tabularx}{\textwidth}{Xlrrrrr}
\lsptoprule
& \bfseries Africa & \bfseries Euras & \bfseries Oceania & \bfseries N.Amer & \bfseries S.Amer & \bfseries TOTAL\\
\midrule
NDef \& DemN & 4{\db} & 3{\db} & 2{\db} & 1{\db} & 1 & 11\\
NDef \& NDem & [33] & [6] & [19] & [8] & 1 & 67\\
\lspbottomrule
\end{tabularx}
\caption{\label{tab:dryer:15}Order of noun and demonstrative in NDef languages}
\end{table}

While \isi{grammaticalization} probably plays some role in explaining this correlation\is{word-order correlation}, it seems likely that the clear semantic similarity between \isi{definiteness} markers and \isi{demonstrative}s plays a role as well. There is also a correlation\is{word-order correlation} between the order of \isi{definiteness} marker and noun and the order of indefinite\is{definiteness} marker and noun, a correlation that is presumably due to semantic similarity or \isi{processing}, not \isi{grammaticalization}.

\section{Conclusion}\label{sec:dryer:5}

I have argued that there is evidence that any approach to explaining word order correlations\is{word-order correlation} that ignores the role of \isi{grammaticalization} is inadequate. At the same time, I have argued that while \isi{grammaticalization} plays a role in explaining some correlations, a pure \isi{grammaticalization} approach fails as well.

Although I have focused my discussion of SVO \& GenN languages on those with both \isi{preposition}s and \isi{postposition}s, further research is needed on SVO \& GenN languages with \isi{preposition}s as the only or dominant type or with \isi{postposition}s as the only or dominant type. Grammaticalization theory would predict that SVO \& GenN languages with \isi{preposition}s will be ones where the primary source of \isi{adposition}s is verbs, while SVO \& GenN languages with \isi{postposition}s will be ones where the primary source of \isi{adposition}s is head nouns in \isi{genitive} constructions\is{possessive construction}. I suspect that this is true and if so, it would further bolster the argument that \isi{grammaticalization} plays an important role in explaining correlations\is{word-order correlation} involving \isi{adposition}s. One reason to suspect it is true is the geographical distribution of the two types of languages. My database includes 21 genera containing SVO \& GenN languages with \isi{preposition}s and 13 of these genera (almost two thirds of them) are in an area\is{contact} stretching from China and Southeast Asia through \ili{Austronesian}. The fact that so many of the SVO \& GenN languages are in this region is significant since my impression is that the \isi{grammaticalization} of \isi{adposition}s from verbs is especially common in this region. Conversely, my database includes 19 genera containing SVO \& GenN languages with \isi{postposition}s and only two of these genera are in the region mentioned above stretching from China through \ili{Austronesian} where SVO \& GenN \& Pr languages are common. I suspect that this is because outside that region, it is more common for \isi{adposition}s to grammaticalize from nouns. However, this is a matter for future research.

\section*{Abbreviations}

The paper abides by the Leipzig Glossing Rules. Additional abbreviations include the following ones:\\


\begin{tabularx}{.45\textwidth}{lQ}
\textsc{anim}  &{animate} \\
\textsc{ant}  &{anterior}\\
\textsc{constr} & construct state\\
\end{tabularx}
\begin{tabularx}{.45\textwidth}{lQ}
\textsc{expec} & expectational\\
\textsc{nc} & noun class\\
\textsc{rem}  &remote\\
\end{tabularx}



\section*{Acknowledgements}

I am indebted to Lea Brown, Karsten Schmidtke-Bode and members of the audience at the 2015 meeting of the DGfS (the \ili{German} Linguistic Society) for comments on an earlier version of this paper. I also acknowledge funding from The Social Sciences and Humanities Research Council of Canada, the National Science Foundation (in the United States), the Max Planck Institute for Evolutionary Anthropology (in Leipzig, Germany) and the Humboldt Foundation (in Germany).


\sloppy
\printbibliography[heading=subbibliography,notkeyword=this] 
\end{document}
\documentclass[output=paper]{langsci/langscibook} 
\author{Sonia Cristofaro\affiliation{University of Pavia}}
\title{Taking diachronic evidence seriously: Result-oriented
  vs. source-oriented explanations of typological universals}
\abstract{Classical explanations of typological universals are
result-oriented, in that particular grammatical configurations are assumed
to arise because of principles of optimization of grammatical
structure that favor those configurations as opposed to
others. These explanations, however, are based on the synchronic
properties of individual configurations, not the actual diachronic
processes that give rise to these configurations
cross-linguistically. The paper argues that the available evidence
about these processes challenges result-oriented explanations of
typological universals in two major ways. First, individual
grammatical configurations arise because of principles pertaining to
the properties of particular source constructions and developmental
mechanisms, rather than properties of the configuration in
itself. Second,  individual configurations arise through several distinct
  diachronic processes, which do not obviously reflect some
  general principle. These facts  point to a
new research agenda for typology, one  focusing on what source constructions and developmental
mechanisms play a role in the shaping of individual cross-linguistic
patterns, rather
  than the synchronic properties of the  pattern in itself.}
\shorttitlerunninghead{Taking diachronic evidence seriously}
\begin{document}
\maketitle 
  
 
 

\section{Introduction}\label{intro}
In the functional-typological approach that originated from the work
of Joseph Greenberg, language universals (henceforth, typological
universals) are skewed cross-linguistic
  distributional patterns whereby languages recurrently display
  certain grammatical configurations as opposed to others. Explanations for these patterns are usually result-oriented, in the 
sense  that at least some of the relevant configurations are assumed
to arise because of some postulated principle of grammatical
structure, which favors those particular configurations and disfavor
other logically possible ones.

For example, a
  number of word order correlations have been explained by assuming that
speakers will recurrently select particular word orders as opposed
  to others because these orders lead to syntactic structures that are
  easier to process (\citealt{Hawkins2004},  among others). Another case in point is
  provided by explanations of the use of explicit marking for
  different grammatical meanings, for example the
  use of overt marking for different number values, or that of 
 dedicated case marking for different NP types
occurring in particular argument roles. Cross-linguistically, explicit
marking may be restricted to less frequent meanings, for example plural rather than singular, animate rather than
inanimate P arguments, or inanimate rather than animate A arguments, but
is usually not restricted to more frequent meanings. This has been assumed to reflect 
a principle of economy whereby
speakers will tend to use explicit marking only when they really need
to do so. Explicit marking can be restricted to less frequent meanings
because more frequent ones are easier to identify, and hence less in
need to be disambiguated
\citep{Greenberg1966,Corbett2000,TU2};
\citealt{Martinmarkedness,Haspelmath2008}). 

These explanations are based on the synchronic properties of the
relevant distributional patterns, not the actual diachronic processes
that shape these distributions from one language to another. For
example, assumptions about the role of
processing ease in determining particular word order correlations are based on the synchronic syntactic configurations
  produced by particular word orders, not the actual diachronic
  origins of these orders from one language to
  another. Similarly, the idea that the use of explicit marking reflects economy is based on
the synchronic cross-linguistic distribution of particular
constructions across different contexts (e.g. zero vs. overt marking
across singular and plural, dedicated case marking across animate and
inanimate A and P arguments), not the actual diachronic processes that give rise to this distribution in
individual languages. 


This paper discusses various types of diachronic evidence about the
cross-linguistic origins of two phenomena that have been described in
terms of typological universals, the distribution of accusative vs.
ergative case marking alignment across different NP types and that of
zero vs. overt marking across singular and plural. 

This evidence, it will
be argued, challenges classical, result-oriented explanations of
typological universals in two major ways.  First, recurrent grammatical
configurations cross-linguistically do not appear to arise because of principles that favor
those particular configurations in themselves. This challenges the idea that these principles play a
role in the emergence of the distributional patterns described by the
relevant universals. 
Second, individual configurations arise through several distinct
  diachronic processes, which do not obviously reflect some
  general principle. This challenges the idea that explanations for
  particular distributional patterns can be read off from the
  synchronic properties of the relevant grammatical configurations,
  because these properties can originate differently in different
  cases. 
These facts call for a source-oriented approach to
  typological universals, one in which the patterns described by
  individual universals are accounted for in terms of the actual
  diachronic processes that give rise to the pattern, rather
  than the synchronic properties of the  pattern in itself.


\section{The animacy/referential hierarchy: Some possible origins of
  alignment splits in case marking}\label{alignment}

One of the most famous typological universals is the animacy/referential hierarchy in \refP{hierarchy}:


\exNG{hierarchy}{1st person pronouns  $>$ 2nd person pronouns $>$ 3rd person pronouns
 $>$ human $>$ animate
  $>$ inanimate  (\fatcitNP{TU2}{130}, among others)}

Among other phenomena, this hierarchy captures some recurrent splits
in the distribution of
accusative and ergative case marking alignment
 across different NP types.
 Accusative alignment
  can be restricted to a left end portion of the hierarchy (e.g.  pronouns, pronouns and animate
nouns), but is usually not restricted to a right end portion of the hierarchy (e.g. inanimate nouns, nouns as opposed to
pronouns). Conversely, ergative alignment  is sometimes restricted to a right end portion of the hierarchy (e.g. inanimate
nouns, nouns as opposed to pronouns, nouns and 3rd person pronouns), but is usually not restricted to a left end portion of the hierarchy
(1st/2nd person pronouns, pronouns as opposed to nouns, pronouns and animate nouns).

A classical result-oriented explanation  for this distribution invokes the economy
principle mentioned in Section \ref{intro}. Speakers  tend to use dedicated case marking  only when
it is really needed, that is, when some grammatical role is more in
need of disambiguation. The NPs towards the right end of the
hierarchy (inanimates, nouns as opposed to pronouns) are more likely
to occur as P arguments, hence, when they do, the P role is relatively
easy to identify, and hence less in need of disambiguation. Dedicated
case marking for P arguments, leading to accusative alignment, may
then be limited to the NPs towards the left end of the hierarchy
(pronouns, animate nouns). By contrast, these NPs are more likely to
occur as A arguments, hence, when they do, the A role is less in need
of disambiguation. Dedicated case marking for A arguments, leading to
ergative alignment, may then be limited to the NPs towards the right
end of the hierarchy
 (\citealt{Silverstein1976,Dixonergativity,Dixon1994,Comrie2,DeLancey1981,Song2001,TU2}).\footnote{Following a standard
    practice in typology (see, for example, \citealt{Comrie2} or \citealt{Dixon1994}), the labels A, P and S are used throughout
    the paper to refer to the two arguments of transitive verbs and
    the only argument of intransitive verbs.} 

This explanation, however, is not  supported by the available
diachronic evidence about the origins of the relevant grammatical configurations
cross-linguistically. 

In many cases where accusative or ergative alignment is restricted to
particular NP types, the relevant alignment pattern is a result of the
development of an accusative or ergative
marker through the reinterpretation of a pre-existing
element with similar distributional restrictions.  In some cases, for example, accusative markers restricted
to pronominal, animate or  definite direct objects are 
structurally identical to topic markers. This is illustrated  in
\refP{kanuri} for Kanuri. 


%\EX{Kanuri (Nilo-Saharan)}
\ea\label{kanuri}
Kanuri (Nilo-Saharan; \fatcitNP{Kanuri}{52})\\
\ea
  \gll Músa sh{í}-\textbf{{ga}} cúro.\\
  Musa 3\textsc{sg-acc} saw\\
  \glt `Musa saw him.' 
  \ex
  \gll wú-\textbf{{ga}}\\
  1\textsc{sg}-as.for\\
  \glt `as for me' 
%\fatcit{Kanuri}{52}

\z
\z
%\nomenclature{SG}{singular}

In such cases, the accusative marker
plausibly originates from the topic
marker in contexts where the latter refers to a P argument and  is
reinterpreted as a marker for this argument (`As for X' $>$ `X ACC': see, for example, \citealt{Rohlfs1984} and \citealt{Pensado1995} for Romance
languages, and \citealt{Konig2008} for several African languages). 
Topics are usually pronominal, animate and
definite, so topic markers
are mainly used in the same contexts as the resulting accusative
markers. 

 Ergative markers not applying to first and second person
  pronouns have been shown to originate
from various types of source elements not applying to these pronouns
  either. Sometimes, for example,  the ergative marker is derived from an indexical
element, such as a demonstrative or a third person pronoun, as
illustrated in \refP{bagandji} for Bagandji.
\citet{McGregor2006,McGregor2008} argues that in such cases the indexical
element is
originally used to emphasize the referent of the A argument, as this
referent is a new or unexpected agent. This strategy does not
    apply to  first and second person pronouns because the referents
    of these pronouns are typically expected agents.
%mcgregor 2006: 400 pdf

%\EX{Bagandji (Australian)}
\ea\label{bagandji}
Bagandji (Australian: \fatcitNP{Bagandji}{63})\\
\gll Yaḍu-\textbf{ḏuru} gāndi-d-uru-ana.\\
wind-\textsc{dem/erg} carry-\textsc{fut-3sg.sbj-3sg.obj}\\
\glt `{\bf This} wind will carry it along / The wind will carry it along.' 
 
\z



In other cases, the ergative marker is derived from a marker used to encode instruments
 in  transitive sentences with no overt third person arguments. In these sentences, the
  instrument can be reinterpreted as an agent, thus evolving into the A
  argument of the sentence. As a result,  the
 marker originally used for the instrument becomes an ergative marker. This process has
  been reconstructed by \citet{Mithun2005} for Hanis Coos, illustrated in
  \refP{hanis}.  Instruments are typically inanimate, so the relevant markers 
do not usually occur with first and second person pronouns.

\ea 
\label{hanis}
Hanis Coos (Coosan; \fatcitNP{Mithun2005}{84})\\
\gll K'w​ɨn-t \textbf{̣x}=m​ɨl:aqətš.\\
shoot-\textsc{tr} \textsc{obl/erg}=arrow\\
\glt `An arrow shot (him).' (from `(He) shot at him with an arrow')

\z

Restrictions in the distribution of particular
alignment patterns across different NP types can also be a result of
phonological processes targeting a subset of these
NPs. 
In English, for example, accusative alignment
  became restricted to pronouns as a result of nouns losing the
  relevant inflectional distinctions due to sound change, as illustrated in Table
  \ref{blake}. 



\begin{table}
  
    \begin{tabular}{lll}
    \lsptoprule
 &1st person&` name'\\
 \midrule
      NOM&\textbf{\textit{ik}} &{\em name}\\
      ACC &\textbf{\textit{mē}} &\textbf{\textit{name}} (from \textbf{\textit{naman}})\\
      \lspbottomrule
\end{tabular}
  
\caption{Pronominal and nominal declension in late Middle English \protect\fatcit{Blake2001}{177--179} \label{blake}}
\end{table}

In Louisiana Creole, A, S and P arguments were originally
undifferentiated for both nouns and pronouns. Pronominal A and S forms,
however, underwent phonological reduction, plausibly due to their high discourse frequency. As a result, as can be seen from Table \ref{apics}, pronouns developed
 distinct forms for A and S
    arguments on the one hand and P arguments on the other, while nominal A, S, and P arguments
    remained undifferentiated. This led to an accusative case marking
    alignment pattern restricted to pronouns \citep{HaspelmathAPiCS}.
   

\begin{table}
  
\begin{tabular}{llll}
\lsptoprule
 & &Subject &Object\\
 \midrule
 Louisiana Creole &1SG &\textbf{\textit{mo}} &{\em mwa}\\
 &2SG &\textbf{\textit{to}} &{\em twa}\\
 \lspbottomrule
\end{tabular}
  
  \caption{Pronominal declension in Louisiana Creole \citep{HaspelmathAPiCS}}\label{apics}
  \end{table}

These various processes do not appear to be triggered by the fact
that, in the resulting grammatical configurations, dedicated case marking is
restricted to roles more in need of disambiguation. 
In some cases, a
pre-existing element  is reinterpreted as a
marker for a co-occurring argument. Topic markers
are reinterpreted as markers for a co-occurring P argument, and
demonstratives and third person pronouns are reinterpreted as markers
for a co-occurring A argument.  This is a metonymization process triggered by the
  contextual co-occurrence of the relevant elements. In other cases,
  a pre-existing element evolves into a case marker  as a
  result of the reanalysis of the argument structure of the
  construction. Such processes are plausibly due to meaning
  similarities between the source construction and the resulting
  construction, for example, instruments can be reinterpreted as
  agents  because of their role in the action being described,
  particularly in the absence of an overtly expressed
  agent. 
 In yet other cases, an existing alignment pattern becomes restricted to
particular NP types because other NPs, due to their phonological
properties, lose their inflectional distinctions as a result of
regular sound change. Finally, particular NPs may develop distinct forms
for some argument roles as a result of the original forms undergoing
phonological reduction due to their discourse frequency. 

Restrictions in the distribution of accusative and ergative aligment,
as determined by individual processes, directly follow from  restrictions in the distribution of
various source constructions, or in the domain of application of particular develop-
mental mechanisms (such as particular phonological processes). These restrictions too, then, cannot actually
be taken as evidence  for principles that favor the resulting
grammatical configurations independently of particular source
constructions and developmental mechanisms. This is further
supported by the fact that, when the source constructions or the
developmental mechanisms involved are not subject to particular
distributional restrictions, the distribution of accusative or
ergative alignment does not display those restrictions either.

For example, accusative
markers sometimes originate from  `take' verbs in constructions of the type `Take
X and Verb (X)', where the `take' verb is reanalyzed as a marker
for its former
       P argument (\citealt{Lord1993,Chappell2013}, among
         several others). The P arguments of `take' verbs can be pronominal, nominal, animate or inanimate (e.g. `take him, it, the child, the sword'), 
and the resulting accusative markers apply to all of these
argument types. This is illustrated for Twi 
%and 
%Mandarin
  %       Chinese 
in 
 \refP{twi},
%and \refP{mandarin}, 
where the accusative marker {\em de},
         derived from a `take' verb, applies to both animate and inanimate P arguments.



%\EX{Twi (Niger-Congo)}
\ea\label{twi}
Twi (Niger-Congo; \fatcitNP{Lord1993}{66, 79})\\
\ea
\gll Wo̱-\textbf{{de}} no ye̱e̱ o̱safohéne.\\
they-\textsc{acc} him make captain\\
\glt `They made him captain.' 

\ex
\gll O-\textbf{{ de}} afoa ce boha-m.\\
he-\textsc{acc} sword put scabbard-inside\\
\glt `He put the sword into the scabbard.' 

\z
\z



Accusative and ergative markers  can  also develop from the
  reanalysis of possessor or oblique markers used on the notional A or
  P arguments of various types of source constructions, for example,
  `X is occupied with the Verbing  {\bf of} Y $>$ `X
        is Verbing Y {\bf ACC}', `{\bf To} X it will be the Verbing
        of Y' $>$ `X {\bf ERG} 
        will Verb Y',  `Y is X{\bf 's} Verbed thing', `Y is
        Verbed {\bf by} X'
        $>$ `X {\bf ERG} Verbed Y'. These processes have been described for
        a wide variety of languages (see, for example,
     \citealt{HarrisCampbell1995,Bubenik1998,Gildea1998,Creissels2008}). In such cases
        too,  the relevant A and P arguments can be nominal,
        pronominal, animate or inanimate NPs (e.g. `The Verbing of
        you, of it, of the house'; `You are Verbed, the house is
        Verbed'). The  markers used for these arguments, then,  will
        be used with all of these NPs, and the resulting accusative or
        ergative markers are used with all of these NPs too. This is
        illustrated in \refP{wayana} and \refP{carina}, where accusative and
        ergative markers derived in this way are used, respectively,
        with nominal inanimate and pronominal animate arguments.

%\EX{Wayana (Carib)}
\ea\label{wayana}
Wayana (Carib;  \fatcitNP{Gildea1998}{201})\\
\gll ​ɨ-pakoro-\textbf{{n}} iri pək wai.\\
1-house-\textsc{acc} make occupied.with 1.be\\
\glt `I'm (occupied with) making my house.' (originally `I am occupied with my house's making.')

\z

%\EX{Cari\~na (Carib)}
\ea\label{carina}
Cari\~na (Carib;  \fatcitNP{Gildea1998}{169})\\
\gll A-eena-r​ɨ ​ɨ-\textbf{{'wa}}-ma.\\
2-have-\textsc{nmlz} 1-\textsc{erg}-3.be\\
\glt `I will have you.' (from a nominalized construction `To me it will be your having.')

\z
% \nomenclature{NOMLZR}{nominalizer}
% \nomenclature{DAT}{dative}

On a similar note, loss of inflectional distinctions through sound
change,
leading to the loss of particular alignment patterns, targets specific
forms because of their phonological properties. This process, then, can affect different NP types cross-linguistically,
provided that the relevant forms have specific phonological
properties. This leads to different distributional
restrictions for particular alignment patterns. In English, as
detailed earlier, the process affected nouns as
opposed to pronouns, leading to accusative alignment becoming
restricted to pronouns. In  Nganasan, however, a combination of sound change and
analogical levelling led to a  loss of inflectional
distinctions for pronouns, but not for nouns
\fatcit{Filimonova2005}{94--98}. As a result, as can be seen from
\refP{nganasan}, accusative alignment became
restricted to nouns, even though this configuration should be
disfavored in terms of economy, because nominal P arguments are less in need
of disambiguation than pronominal ones.

%\EX{Nganasan (Uralic)}
\ea\label{nganasan}
Nganasan (Uralic; \fatcitNP{Filimonova2005}{94})\\
  \ea
    \gll \textbf{{Mənə}} nanuntə m​ɨntəl'i-ʔə-ŋ.\\
    1\textsc{sg} 2\textsc{sg.loc-instr} take-\textsc{indef}-2\textsc{sg}\\
    \glt `You have taken me with you.' (pronominals originally had dedicated accusative forms, e.g. {\em mənə-m} `1\textsc{sg-acc}')
    
  \ex
    \gll ŋül{\oe}ȝə tund​ɨ-\textbf{{m}} tandarku-čü.\\
    wolf fox-\textsc{acc} chase-3\textsc{sg.a}\\
    \glt `The wolf is chasing the fox.'
    
  \z
\z

If there were principles that favor or disfavor particular distributional
restrictions for accusative and ergative aligment because of properties of the
resulting grammatical configurations, we would not expect the development of
these restrictions to be tied to specific source
constructions and developmental mechanisms. 

Finally, individual distributional restrictions develop through several distinct
processes, which are rather different in nature and provide
independent motivations for the restriction. In some cases, particular
restrictions arise as accusative and ergative case markers develop
through processes of context-driven reinterpretation of various types
of source elements, which, for different reasons, are restricted in the same way. In other cases, the restrictions reflect the domain of
application of different phonological processes. To the extent that different diachronic processes provide different
motivations for particular distributional restrictions, 
explanations for these restrictions cannot be read off from the
restrictions in themselves, because these can originate differently in
different cases.

\section{Some possible origins of zero vs. overt marking for singular
  and plural}\label{number}
Another well-known typological
universal pertains to the use of zero vs. overt
marking for singular and plural. Languages can use overt marking for plural and zero marking for
  singular, but usually not the other way round. A classical,
  result-oriented explanation for this pattern, as
  mentioned in Section \ref{intro},  is in terms of economy. Speakers tend to use overt marking
  only for meanings that are more in need of disambiguation, and plural is more in need of disambiguation than singular due to its
  lower discourse frequency. As a result, overt marking can be limited
  to plural, whereas it will not be limited to singular (\citealt{Greenberg1966,TU2};
  \citealt{Haspelmath2008}).  This explanation, however, is not
supported by a number of diachronic processes that lead languages to
have zero marked singulars and overtly marked plurals.


Often, in languages which make no distinction between singular
and plural, an overt plural marker  evolves through the reinterpretation of
pre-existing expressions, whereas
    singulars retain zero marking. Sometimes, some expression  takes on a plural meaning originally associated with
      a co-occurring expression. For example, in partitive
      constructions involving plural quantifiers (`many
      {\bf of} them' $>$ `they  {\bf \textsc{pl}}'), the partitive marker can take on the meaning of
      plurality associated with the quantifier as the latter is
      lost. This process took place in  Bengali, as illustrated in \refP{bengali}.



%\EX{Bengali (Indo-European)}
\ea\label{bengali}
Bengali (Indo-Aryan; \fatcitNP{Chatterji}{735--736})\\
\ea
\gll āmhā-\textbf{{rā}} s{\r{a}}b{\r{a}}\\
we-\textsc{gen} all\\
\glt `all of us' (14th century)


\ex
\gll chēlē-\textbf{{rā}} \\
child-\textsc{gen}\\
\glt `children' (15th century)

\z
\z


In other cases, plurality becomes the central meaning of expressions
inherently or contextually
associated with this notion but originally used to encode other meanings, for example, distributive expressions
(`house here and there') or expressions of multitude (`all', `people'). This is illustrated in \refP{paiute} and \refP{maithili}.

%\EX{Southern Paiute (Uto-Aztecan)}
\ea\label{paiute}
Southern Paiute (Uto-Aztecan; \citealt[258]{Sapir1930})\\
\gll qa'nɪ​ / \textbf{{qaŋqa'nɪ​}}\\
house / house.\textsc{distr}\\
\glt `house, houses'
\nocite{Paiute}

\z
%\nomenclature{DISTR}{distributive}

%\EX{Maithili (Indo-Aryan)}
\ea\label{maithili}
Maithili (Indo-Aryan: \fatcitNP{Maithili1997}{69})\\
\gll jən \textbf{{səb}}\\
laborer all\\
\glt `laborers'

\z

Another process that leads languages to have zero marking for singular
and overt marking for plural is the elimination of an overt singular
marker through regular sound change in a situation where both singular
and plural are originally overtly marked. This was, for example, the case in
English, where singular and plural were both originally overtly
  marked in most cases. The current
configuration with zero marked singulars and {\em -s} marked plurals
resulted from a series of sound changes that led to the
elimination of all inflectional endings except genitive singular {\em
  -s} and plural {\em -es} (\citealt{Mosse}). 


These various processes do not appear to be triggered by the higher need to
disambiguate plural as opposed to singular. In some cases, an overt
plural marker arises as a result of a metonymization
process whereby plural meaning is transfered from a quantifier to some
other component of a complex expression. This is plausibly due to the
co-occurrence of the two. In other cases, some
expressions evolves into a plural marker because it is contextually or
inherently associated with the notion of plurality, and this notion becomes the
central meaning of the expression as some other meaning component is
bleached.  In yet
other cases, a pre-existing overt singular marker is eliminated due to
regular sound changes driven by the phonological  properties of the
marker. 

The end result of the various processes, the use of  overt marking for
plural rather than singular, is  directly motivated in
terms of the properties of particular source constructions and
developmental mechanisms.  In many cases,  an overt
marker is  used for plural because the source construction is one associated with the notion of
plurality. Alternatively, sound changes leading to the elimination of
an overt marker target singular rather than plural markers due to the
phonological properties of the former. The fact that overt marking
is restricted to plural, then, cannot be taken as evidence for
principles that favor this particular configuration independently of
particular source constructions and developmental mechanisms. As in
the case of accusative and ergative case marking alignment, this point
is further supported by the fact that other source
constructions and developmental mechanisms give rise to different
configurations, that is, overt marking for both singular and plural,
or just for singular.

A case in point is provided by  Kxoe,
illustrated in Table \ref{kxoe} below. This language has gender
markers derived from third person pronouns \cite{Heine1982}. As the
pronouns have overt singular and plural forms, the resulting gender
markers also encode singular and plural, so that the language has
overt marking not only for plural, but also for singular.


\begin{table}
\begin{tabular}{lllll}
\lsptoprule
&& SG &PL &\\
\midrule
Nouns &M &{\em /õ{\acm{a}}-}\textbf{\textit{mà}}&{\em /õ{\acm{a}}-}\textbf{\textit{//u`a}}& `boy'\\
&F &{\em /õ{\acm{a}}-}\textbf{\textit{h\`ε}}&{\em /õ{\acm{a}}-}\textbf{\textit{djì}}&`girl\\
&C &{\em /õ{\acm{a}}-}\textbf{\textit{('à)}}, {\em
     /õ{\acm{a}}-}\textbf{\textit{djì}} &{\em õ{\acm{a}}-}\textbf{\textit{nà}}& `child'\\
Pronouns &M &{\em xà-}\textbf{\textit{á}}, {\em
              á-}\textbf{\textit{mà}}, {\em
              i-}\textbf{\textit{mà}}&{\em
                                         xà-}\textbf{\textit{//u̯á}},
                                         {\em
                                         á-}\textbf{\textit{//u̯á}},
                                         {\em {í}--}\textbf{\textit{//u̯á}}& `he'\\
&F &{\em xà-}\textbf{\textit{h\`ε}}, {\em
     á-}\textbf{\textit{h\`ε}}, {\em
     i--}\textbf{\textit{h\`ε}}&{\em
                                              xà-}\textbf{\textit{dj{í}}},
                                              {\em
                                              á-}\textbf{\textit{dj{í}}},
                                              {\em {í}-}\textbf{\textit{dj{í}}}& `she'\\
&C &({\em xa-'}\textbf{\textit{à}})&{\em
                                       xà-}\textbf{\textit{nà}},
                                       {\em
                                       á-}\textbf{\textit{nà}},
                                       {\em {í}-}\textbf{\textit{nà}}& `it'\\
%Demonstratives &M &&&& `that'\\
%&F &&&& `that'\\
%&C &&&& `that'\\}
\lspbottomrule
\end{tabular}
\caption{Gender/number markers and third person pronouns in Kxoe (Khoisan; \citealt[211]{Heine1982})}\label{kxoe}
\nocite{Heine1982}
\end{table}
\nomenclature{C}{common}%!


Also, as described above, partitive case markers can evolve into plural markers by
taking on the plural meaning associated with a co-occurring plural
quantifier. In expressions where the quantifier is singular (`one of
them'), however, this same process can lead to the development of
singular markers, sometimes leading to a configuration where only singular
is overtly marked. This was the case in Imonda, which has zero marked plurals, but developed
an overt non-plural
(singular and dual) marker
 from a source case marker used in partitive constructions \fatcit{Imonda}{38--39}.

%\EX{Imonda (Border)}
\ea\label{imonda2}
Imonda (Border; \fatcitNP{Imonda}{194, 219})\\
\ea
\gll Agõ-\textbf{{ianèi}}-m ainam fa-i-kõhõ.\\
women-\textsc{nonpl-gl} quickly \textsc{clf-lnk}-go\\
\glt `He grabbed the woman.' 

\ex
\gll mag-m ad-\textbf{{ianèi}}-m \\
one-\textsc{gl} boy-\textsc{src-gl} \\
\glt `to one of the boys'
\nomenclature{NONPL}{non-plural}%!
\nomenclature{GL}{goal}%!
%\nomenclature{CL}{classifier}
\nomenclature{LNK}{linker}%!
\nomenclature{SRC}{source}%!


\z
\z


Similar observations apply to loss of number markers through sound
change. This process can affect
either singular or plural markers depending on the phonological
properties of the marker. From one language to another, then, the process may lead either to  zero marked singulars and
overtly marked plurals, as detailed above for
English, or to the opposite configuration. In the
Indo-Aryan language Sinhala,  for example, some inanimate nouns have
overtly marked singulars and
zero marked plurals
 (e.g. {\em pot-\textbf{\textit {a}}/ pot} `book-\textsc{sg}/ book.\textsc{pl}'). This was a result of
 sound changes leading to the loss of the plural ending of a specific inflectional class in the ancestor language
% specific inflectional class in middle indica (declension ii, which is the ancestor of today's inanimate classes, e.g. at-a SG at PL 'hand, 256)
 \citealt[250--256]{NitzNordhoff2010}. Similarly, in Nchanti, a Beboid
 language, nouns in classes 3/4 have overt
  marking in the singular and zero marking in the plural, e.g. {\em
    k{\bf ʷ}\=əŋ}/ {\em
    k\=əŋ} `firewood.\textsc{sg}/ firewood.\textsc{pl}, {\em k{\bf
      ʷ}ēē/ kēē} `moon.\textsc{sg}/ moon.\textsc{pl}'.
  Originally, both singular and plural were marked overtly through the
  two prefixes {\em *u-} and {\em *i-} respectively. As these were
  eliminated, the singular prefix led to the labialization of the
  initial consonant of the stem, while the plural prefix left no trace
  \cite{Hombert1980}.

Finally, just like distributional restrictions for accusative and
ergative case marking alignment, the fact that a language uses zero marking for singular
and overt marking for plural can be a result of a variety of
diachronic processes,
 which lead to this particular configuration for different
reasons.  
In many cases, both singular and plural are originally zero
marked (i.e., the language makes no distinction between the two), but
zero marking becomes restricted to singular because different expressions, for different reasons, evolve into plural
markers.  In other cases, both
singular and plural are originally overtly marked, and sound change
leads to the loss of singular markers due to their phonological
properties. Explanations for why overt marking is restricted to plural, then,  cannot be read off from this configuration in itself,
because it can originate differently in different cases.


\section{Rare grammatical configurations and result-oriented explanations}\label{martin}

Result-oriented explanations of typological universals are crucially
 based on
the fact that certain logically possible grammatical configurations
are significantly rarer than
others in the world's languages. This is usually accounted for by
postulating  principles that both disfavor those configurations
and favor some of the other configurations. For
example, the rarity of configurations where singular is
overtly marked and plural is zero marked is assumed to be due to the
fact that these configurations are disfavored by economy, and hence
will usually not occur in  the world's languages. This same
principle is assumed to also favor the opposite configuration, zero marking for
singular and overt marking for plural, thus providing a motivation for
the occurrence of this configuration.

\citet{Haspelmath2019tv} uses this line of reasoning to claim that
result-oriented explanations  should be invoked even in cases where
the development of some grammatical configuration is accounted
for by the properties of particular source constructions or developmental
mechanisms, rather than synchronic properties of the configuration in itself.  Haspelmath concedes that, in such cases,
there is no direct evidence that the occurrence of the
configuration is motivated by principles pertaining to its synchronic
properties (functional-adaptive principles, in his terminology). He argues, however, that this
hypothesis is supported by two types of indirect evidence: the fact
that other logically possible configurations are significantly rarer, and what he calls
multi-convergence, the fact that different  diachronic processes all
lead to that particular configuration. According to Haspelmath, these facts
can only be accounted for by assuming that the occurrence of
the configuration is ultimately motivated by
principles that favor that configuration independently of the
diachronic processes that give rise to it.
 Haspelmath draws a parallel with the notion
of adaptiveness in evolutionary biology (and other domains):
The development of particular traits is independent of the fact that
those traits are adaptive to the environment, in the sense of conferring an evolutionary advantage to the organisms
carrying them, but adaptiveness provides the ultimate explanation for their
spread and survival in a population.

There is, however, a logical fallacy in the idea that, if some
 principle motivates the  non-occurrence of some
configuration (and hence its rarity),
then the occurrence of some other configuration is motivated by the
same principle. The fact that some principle A provides the motivation
for some phenomenon X can can be framed as a logical implication,  X
$\rightarrow$ A (because X will always involve A, unless other
motivations for X are also postulated).  This
implication means, however, that the absence of A will lead to phenomena different
than X, that is, $\sim$ A $\rightarrow$ $\sim$ X, not that phenomena
different from X are also motivated by A. This would be a distinct logical
implication, $\sim$ X $\rightarrow$ A, with a different truth table.
For example, if the non-occurrence
of configurations where singular is
overtly marked and plural is zero marked (X)  is assumed to be due to
economy (A), this means that principles other than economy ($\sim$ A) will lead to the occurrence of other configurations ($\sim$ X), not that
the latter phenomenon is also due to economy. This undermines the general
logic of result-oriented explanations, including Haspelmath's
argument: From the fact that some principle provides 
a motivation for the non-occurrence of some
configuration, we cannot conclude that it also provides a
motivation for the occurrence of other configurations.

As for the multi-convergence argument, this
ignores the fact that different
diachronic processes can all lead to the same synchronic output for
different reasons, as detailed in Sections \ref{alignment} and
\ref{number}. If the same synchronic output is motivated
differently in different cases, multi-convergence cannot be taken as
evidence for principles that favor that output
independently of the individual processes that give rise to
it. Instead, to the extent that the various processes recurrently take place in
different languages, the cross-linguistic distribution of the output
will be a combined result of the effects of each process. 

From a logical point of view, source-oriented explanations do not rule
out that the cross-linguistic distribution of particular grammatical
configurations may ultimately also be determined by properties
pertaining to the synchronic properties of the configuration, as
assumed by Haspelmath. For example,  these factors could play a role in
the transmission of the configuration from one speaker to another, or
its retention across different generations of speakers. This would be
the equivalent of the notion of adaptive evolution through natural
selection in evolutionary
biology: particular genetic traits do not develop because they
they confer an evolutionary advantage to the organisms
carrying them, but this provides the ultimate explanation for their
distribution in a population.\footnote{A referee suggests that this is
  similar to Lass's \citeyear{Lass1990} use of the notion of
  exaptation: particular grammatical traits may lose their original
  function, but they are retained in the language because they are
  deployed for novel functions.  This, however, is meant to account for why
  particular traits  survive in a language despite losing their
  original function, not why they are
  selected over others, as is the case with  result-oriented
  explanations of typological universals and explanations of
  biological evolution in terms of adaptiveness through natural selection.}

In evolutionary biology, however, there is direct evidence for
adaptiveness, in that particular genetic traits make it  demonstrably more likely for the organisms
carrying them to survive and pass them on to their
descendants. For languages, on the other hand, there is generally is no evidence that the
fact that some grammatical configuration conforms to the principles
postulated in result-oriented explanations, for example economy, makes 
it more likely for that configuration to spread and survive in a
speech community. In fact, there is a long tradition of
linguistic thought in which the propagation of individual
constructions within  a speech community is entirely determined by
social factors independent of particular inherent properties of the
construction (see, for example, \citealt{McMahon1994}
and \citealt{BillLC} for reviews of the relevant issues and
literature). 

In principle, there is another sense in which particular
grammatical configurations could be adaptive. While individual
configurations directly reflect the properties of particular
source constructions or developmental mechanisms, it could be the case
that the specific diachronic processes that give rise to the configuration are ultimately
favored by principles pertaining to its synchronic properties. For example, different
processes of context-driven reinterpretation leading to overt marking for less frequent
types of argument roles could be favored by the need to
give overt expression to these roles. Similarly, different processes leading to zero
marking for singulars (zero marking
becoming restricted to singular due to the development of an overt plural
marker, phonological erosion of an existing overt singular
marker) could be favored by the lower need to give overt
expression to singular as opposed to plural.

These assumptions, however, 
are not part of any standard account of the relevant processes in
studies of language change (see \citealt{BybeeEtAl1994}: 298--300 and  \fatcitNP{Slobin2002}{381} for an explicit rejection of this view in
regard to grammaticalization, as well as \citealt{Otadependencies} for
more discussion).
In fact, diachrony provides specific evidence against the
idea that particular grammatical configurations develop both
because of properties of particular source constructions or developmental
mechanisms and because of
principles that favor the configuration in iself. As detailed in Sections
\ref{alignment} and \ref{number}, different source constructions and developmental mechanisms
give rise to different grammatical configurations, even when this goes
against some postulated principle that favors some of these
configurations as opposed to the others.
This is not what one would expect if there
were principles favoring particular grammatical
configurations independently of the specific source constructions or
developmental mechanisms that give rise to them.


All this means that, to the extent that a principled source-oriented
explanation is available for the occurrence of particular grammatical
configurations, explanations in terms of the synchronic properties of
the configuration are redundant, because we don't have either direct
or indirect evidence for these explanations (see \citealt{Blevins2004} for
similar arguments in phonology, and \citealt{Newmeyer2002,Newmeyer2004} for an application of
this line of reasoning to Optimality Theory models of typological
universals). Of course, one still needs to account for the fact that
certain logically possible grammatical configurations
are significantly rarer than others. This phenomenon, however, is logically
independent of the possible motivations for the occurrence of the more
frequent configurations, as
detailed above. To the extent that
individual grammatical configurations arise due to properties of
particular source constructions or developmental mechanisms, any differences in the frequency of particular
configurations will reflect differences in the frequency of
the source constructions or developmental mechanisms that give
rise to those configurations. The higher
frequency of particular configurations will then  be a result of the higher frequency of the  source
constructions and developmental mechanisms that give rise to them,
while the rarity of other configurations
will  due to the rarity of possible
source constructions or developmental mechanisms for those
configurations (see \citealt{Harris2008} for an earlier formulation of this point
in regard to tripartite case marking alignment). Frequency differences in the occurrence of particular source
constructions or developmental mechanisms need to be
accounted for, but they need not be related to any properties of the
resulting configurations, so they should be investigated independently.

\section{Concluding remarks}
Source-oriented explanations of typological universals are  in line with classical
views of language change held within
grammaticalization studies and historical linguistics in
general. These views are manifested, for example, in accounts of the
development of tense, aspect and mood systems, or alignment
patterns (\citealt{BybeeEtAl1994,HarrisCampbell1995,Gildea1998,TraugottDasher2005}, among others). In these accounts, grammatical change
is usually not related to synchronic properties of the resulting
constructions, for example the fact that the use of these
constructions complies with some postulated principle of optimization
of grammatical structure. Rather, grammatical change is usually a result of the properties of particular
source constructions and the contexts in which they are used. In
particular, new grammatical constructions recurrently emerge through
processes of context-induced reintepretation of pre-existing
ones, and their distribution originally reflects the distribution of the
source constructions. 

In source-oriented explanations,  the patterns
captured by typological universals originate from several distinct diachronic processes, which involve
different source constructions and developmental mechanisms. These
processes recurrently take place in different languages, and are
plausibly motivated by the same factors from one language to another. Individual patterns, however, are a combined
result of the cross-linguistic frequencies of the various processes, rather than a result
of some 
overarching principle independent of these processes.

While
this scenario is 
more complex and less homogeneous than those assumed in result-oriented explanations, it is consistent with what is known about the actual origins of
the relevant grammatical configurations in individual languages, and it
makes it possible to address several facts  not accounted for in
these explanations. 

For example, the patterns captured by typological universals usually
have exceptions. This is
in contrast with the assumption that these patterns reflect principles of optimization of
grammatical structure that are valid for all languages, because in
this case one has to account for why these principles are violated in
some languages. Also, individual principles invoked in result-oriented
explanations are often in contrast with some of the
grammatical configurations captured by individual universals. For
example, the idea that zero marking of more frequent meanings is
motivated by economy is in constrast with the fact that these meanings
are overtly marked in many languages. 

These facts have sometimes been
dealt with in terms of competing motivations models, but a general
problem with this approach is that it may lead to a
proliferation of explanatory principles for which no independent
evidence is available (\fatcitNP{Newmeyer1998}{145--153}, \citealt{Otacompetingmotivations}, among
others). If the patterns captured by typological universals reflect the properties of different
source constructions and developmental mechanisms, however,  then it is
natural that they should have exceptions, because not all
languages will have the same source constructions, nor will
particular developmental mechanisms be activated in all
languages. Principles pertaining to the synchronic properties of the
pattern will fail to account for all of the relevant grammatical
configurations because the pattern is not actually motivated by those principles.

Over the past decades, several linguists have emphasized the need for
source-oriented explanations of typological universals (\citealt{Bybee1988,Bybee2006,Bybee2008,Aristar1991,Gildea1998,Creissels2008,Otareferential,Otacompetingmotivations,Otadependencies,Anderson2016}). This
view, however, has not really made its way into the actual typological
practice, despite the close integration between typology and studies
of language change (a fully fledged research approach along these lines is, on
the other hand, the Evolutionary Phonology framework developed in
\citealt{Blevins2004}). While diachronic evidence about the origins
of the patterns captured by individual universals is much scantier and
less systematic than the synchronic evidence about these patterns, 
 it poses specific foundational problems for existing result-oriented explanations of these universals. These problems point to a
new research agenda for typology, one  focusing on what source constructions and developmental
mechanisms play a role in the shaping of individual cross-linguistic
patterns, as well as why certain source
constructions or developmental mechanisms are rarer than others.

\section*{Abbreviations}

The paper conforms to the Leipzig Glossing Rules. Additional abbreviations include:
  
% {\footnotesize 
% \printnomenclature } 
\sloppy
\printbibliography[heading=subbibliography,notkeyword=this] 


\end{document} 
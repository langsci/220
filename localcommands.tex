\newcommand{\citeapo}[1]{\citeauthor{#1}' (\citeyear*{#1})}

 
\newcommand{\lspxcol}{}
\newenvironment{langscibars}[1][]{\begin{axis}[ybar,xtick=data, xticklabels from table={\mydata}{\lspxcol}, 
        width  = \textwidth,
	height = .3\textheight,
    	nodes near coords, 
	xtick=data,
	x tick label style={},  
	ymin=0,
	cycle list name=langscicolors,
	#1
        ]}{\end{axis}}
        
\newcommand{\langscibar}[1]{\addplot+ table [x=i, y=#1] {\mydata};\addlegendentry{#1};}

\newcommand{\langscidata}[1]{\pgfplotstableread{#1}\mydata;}
 
\newcounter{lsConnectTempGroup}
\NewDocumentCommand\ConnectTail{m O{\thelsConnectTempGroup}}{%read: mandatory arg #1, optional argument #2 with the current group counter as its default value.
    \edef\lsConnectTempPosition{#2}%\edef expands the argument, which means reading the current value of the counter.
    {\tikz[remember picture,
           anchor=base, baseline,
           inner xsep=0pt, 
           inner ysep=-.5ex]\node (ConnectTempTail\lsConnectTempPosition) {\strut{}#1};}%\strut for baseline
}
\NewDocumentCommand\ConnectHead{s O{1ex} m O{\thelsConnectTempGroup} s}{%read: star #1, optional argument (distance of arrow from text= std. one x-height), mand. arg. (node text), optional argument #2, the group specifier
    \edef\lsConnectTempPosition{#4}%
    \stepcounter{lsConnectTempGroup}%We have a match, let's update the group counter   
    {\tikz[remember picture,
           anchor=base, baseline, 
           inner xsep=0pt, 
           inner ysep=-.5ex] \node (ConnectTempHead\lsConnectTempPosition) {\strut{}#3};%
     \tikz[remember picture] \draw[% we have a tail and a head, let's bring them together
                                \IfBooleanTF#1{{Triangle[]}-}{-{Triangle[]}},% Check if the starred version is used. The starred version is right->left, the normal version left->right
                                overlay,dashed] (ConnectTempTail\lsConnectTempPosition.south) -- ++(0,-#2) -| (ConnectTempHead\lsConnectTempPosition.south);%
    }\IfBooleanTF#5{\vspace{#2}}{}%
}

\NewDocumentCommand\NodeConnectHead{O{-{Triangle[]}} O{1ex} m O{\thelsConnectTempGroup} m s}{%read: star #1, optional argument (distance of arrow from text= std. one x-height), mand. arg. (node text), optional argument #2, the group specifier
    \edef\lsConnectTempPosition{#4}%
    \stepcounter{lsConnectTempGroup}%We have a match, let's update the group counter   
    {\tikz[remember picture,
           anchor=base, baseline, 
           inner xsep=0pt, 
           inner ysep=-.5ex] \node (ConnectTempHead\lsConnectTempPosition) {\strut{}#3};%
     \tikz[remember picture] \draw[% we have a tail and a head, let's bring them together
                                #1,overlay] (ConnectTempTail\lsConnectTempPosition.south) -- ++(0,-#2) -| node[near start,fill=white] {#5} (ConnectTempHead\lsConnectTempPosition.south);%
    }\IfBooleanTF#6{\vspace{#2}}{}% Automatically make space below 
} 


% Some commands and definitions.
\newcommand{\secn}[1]{Section~\ref{#1}}
\newcommand{\tbl}[1]{Table~\ref{#1}}
\newcommand{\fig}[1]{Fig.~\ref{#1}}
\newcommand{\sect}[1]{\setcounter{equation}{0}\section{#1}}
% \newcommand{\fatcit}[2]{(\citeNP{#1}: {#2})} % (Croft 1990: 24)
% \newcommand{\fatcitNP}[2]{\citeNP{#1}: {#2}} % Croft 1990: 24
% \newcommand{\fatcitA}[2]{\citeANP{#1} (\citeyearNP{#1}: {#2})}%Croft (1990: 24)
\newcommand{\fatcit}[2]{\citep[#2]{#1}} % (Croft 1990: 24)
\newcommand{\fatcitNP}[2]{\citealt[#2]{#1}} % Croft 1990: 24
\newcommand{\fatcitA}[2]{\citet[#2]{#1}}%Croft (1990: 24)


\newcommand{\mr}[1]{$\rm {#1}$}
\newcounter{lang} 
\newcommand{\exn}[2]{
\vspace{-.1cm}
\begin{tabular}{*{#1}{l}} #2
\end{tabular}}
\newcommand{\exNG}[2]{
  \ea
  {\label{#1}} #2
  \z
}

\newcommand{\trad}[1]{ 
  \begin{quote} #1
  \end{quote}
}
\newcommand{\refP}[1]{(\ref{#1})}
\newcounter{app}
\setcounter{app}{0}
\newcommand{\app}[2]{
  \begin{quote}
  \refstepcounter{app}{\label{#1}}
  {\bf \large \arabic{app}}{\hspace{.5cm}}
  {\bf \large #2}
  \end{quote}
  \addtocontents{toc}
  {\protect\contentsline{section}{\theapp{\hspace{.5cm} #2}}{\thepage}}
}
 

\newcommand{\acm}[1]{\'\=#1}
% \newcommand{\textsubbar}[1]{#1̱ }
% \newcommand{\textsubdot}[1]{\d{#1}}
% \newcommand{\textsubarch}[1]{̯#1}